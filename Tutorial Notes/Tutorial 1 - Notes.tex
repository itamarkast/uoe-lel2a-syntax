\documentclass{article}
\usepackage{xr-hyper} %Adds referencing between handouts and the Skills.tex document to avoid typos (req. latexmkrc)
\externaldocument{Skills} %where to look for labels
\usepackage[hidelinks]{hyperref} %links and URLS
\usepackage[linguistics]{forest} %needs tikz, draws trees
\usepackage[margin=1in]{geometry} %page layout
\usepackage{graphicx} % Required for inserting images
\usepackage[T1]{fontenc} %Make sure to be able to get accented characters etc
\usepackage[utf8]{inputenc}
\usepackage[normalem]{ulem} %adds strikethrough and other commands
\setlength{\parindent}{0pt}%don't indent paragraphs...
\setlength{\parskip}{1ex plus 0.5ex minus 0.2ex} 
\usepackage{multicol} %adds columns
\usepackage{gb4e} %for formatting examples, works with leipzig and multicol
\primebars %setting for gb4e, adds bars for X-bar notation, allows switch between bar or %'%
\noautomath
\usepackage{tabto}
\usepackage{amssymb}
\usepackage{fancyhdr}
\usepackage{setspace}
\usepackage{pifont} %allows dingbats to be called (for the "crosses" and "ticks" defined below)
\usepackage{tipa} % IK


\usepackage{leipzig}%primarily used for the abbreviations

\usepackage[backend=biber,
            style=unified,
            natbib,
            maxcitenames=3,
            maxbibnames=99]{biblatex}
\addbibresource{references.bib}
\usepackage{attrib}%allows authors next to quote environments

\makeatletter
\def\@maketitle{%I guessed from the commenting out of the author below that you don't want an author, this just gets rid of the space associated with the author field
  \newpage
  \null
%  \vskip 2em%
  \begin{center}%
  \let \footnote \thanks
    {\LARGE {\@title}\par}
%    \vskip 1.5em%
%    {\large
%      \lineskip .5em%
%      \begin{tabular}[t]{c}%
%        \@author
%      \end{tabular}\par}%
    \vskip 1em%
    {\large \@date}%
  \end{center}%
  \par
%  \vskip 1.5em
}
\makeatother

\title{LEL2A: Syntax}
%\author{Instructor: Itamar Kastner}
\date{Semester 1, 2024-25}%changed to current academic year

\newcommand*{\sqb}[1]{\lbrack{#1}\rbrack}
\newcommand*{\fn}[1]{\footnote{#1}}
\newcommand{\keyword}[1]{\textsc{#1}}
\newcommand{\cmark}{\ding{51}}
\newcommand{\xmark}{\ding{55}}
\newcommand{\subtitle}[1]{\maketitle\begin{center}{\Large #1}\end{center}}
\makeatletter
\newcommand*{\addFileDependency}[1]{% argument=file name and extension
\typeout{(#1)}% latexmk will find this if $recorder=0
% however, in that case, it will ignore #1 if it is a .aux or 
% .pdf file etc and it exists! If it doesn't exist, it will appear 
% in the list of dependents regardless)
%
% Write the following if you want it to appear in \listfiles 
% --- although not really necessary and latexmk doesn't use this
%
\@addtofilelist{#1}
%
% latexmk will find this message if #1 doesn't exist (yet)
\IfFileExists{#1}{}{\typeout{No file #1.}}
}\makeatother

\newcommand*{\myexternaldocument}[1]{%
\externaldocument{#1}%
\addFileDependency{#1.tex}%
\addFileDependency{#1.aux}%
}
\myexternaldocument{Skills} %also necessary for cross referencing, to reference other documents duplicate with name of document


%% Trigger answers and notes. Package by Byron Ahn with edits by Craig Sailor
\usepackage[solution]{myProbSol} %   Optional arguments:
%           'solution': reveals solutions
%           'spaces': leaves whitespace corresponding to the size of the solutions (has no effect if you also pass 'solution')

\begin{document}
\maketitle
\subtitle{Tutorial Notes Week 5: Topics 1, 2 \& 3}

\paragraph{Explanation on accumulating Syntax Skills:} Let me stress that the point of the syntax block is to learn how to think like a syntactician, learn about English, and try to have fun. But marks are involved, and here's how we mark Syntax Skills: generally speaking, it's enough to show them once. This means that if we have questions Q1A and Q2B in a given tutorial sheet (for example), both of which target skill 6b, then getting one of them right is enough. That said, it's best to attempt all questions.

\section*{Question 1}
\hfill{}
%\ref{whatissyntaxA} $\Box$,
\ref{whatissyntaxB} $\Box$
%\ref{whatissyntaxC} $\Box$

\paragraph{Q1A} Given the following list of constituents, visually represent the structure of the sentence \emph{My cousin gave the dog a juicy bone}:
\begin{itemize}
    \item my cousin
    \item the dog
    \item a juicy bone
    \item gave the dog a juicy bone
\end{itemize}
Don’t worry too much about whether your representation is \emph{right} or not in other aspects, just as long as it
shows those four strings as constituents. For example, it doesn't need to have any labels for the constituents (like VP or NP).

\begin{answer}
{
Here are three possible trees that all meet the requirements set out in the question concerning what must be represented as a constituent. In this case, none of them make any claims about the \emph{categories} of the constituents (all the constituents are unlabelled). The trees differ from each other in terms of what \emph{additional} claims they embody about constituents within the sentence.  For example the trees in II and III both embody a claim that \emph{the dog a juicy bone} is also a constituent).\\
\begin{quote}
        \ex[I]{
    \small \begin{forest}
for tree={fit=band, parent anchor=north,},
	[\phantom{}
 [\phantom{} [my] [cousin]][\phantom{}
 [gave] [\phantom{} [the] [dog]] [\phantom{}[a] [juicy] [bone] ]]]
    \end{forest}}
    \ex[II]{
    \small \begin{forest}
for tree={%calign=fixed edge angles,
fit=band, parent anchor=north,},
	[[[my] [cousin]][[ gave] [[[the] [dog]] [[a] [juicy] [bone] ]]]]
    \end{forest}}
    \ex[III]{
    \small \begin{forest}
for tree={%calign=fixed edge angles,
fit=band, parent anchor=north,},
	[[[my] [cousin]][[gave]
 [[the] [dog]]
 [[a]
 [[juicy] [bone]] ]]]
    \end{forest}}
\end{quote}
}
\end{answer}

\hfill{}\ref{constituencytestVP} $\Box$
%, M5 $\Box$

\paragraph{For discussion} Above, you were told that \emph{gave the dog a juicy bone} was a constituent.
However, if we apply an \emph{it-cleft} test of constituency to this string we get the following results:
\begin{exe}
    \ex[*]{It is gave the dog a juicy bone that my sister}
\end{exe}
What does this mean for the representation you gave above?
Can you use other diagnostics to resolve this issue (one is enough but if you can give more that's even better)?

\begin{answer}
{
Clefts are not a good test for VP constituency, and even if they were, we might still get a false negative. We can try other tests instead and see that the string passes them, indicating that it is a constituent (of type VP) after all: \emph{My cousin gave the dog a juicy bone and I did so too}, \emph{What did my cousin do?}
}
\end{answer}

\section*{Question 2}
\hfill{}
\ref{constituencytestNP} $\Box$,
\ref{constituencytestVP} $\Box$,
\ref{constituencytestother} $\Box$
%, M3 $\Box$, M4 $\Box$

\paragraph{Q2A} Which of the following trees is the best representation of the structure for the sentence \emph{She donated that dress to the shop}, based on what you can determine about the constituent
structure of the sentence? For each of the trees that you didn’t choose in (\ref{Q2_trees}), what is the evidence against it?
\begin{exe}
\ex{
\begin{xlist}
    \ex{
    \small \begin{forest}
for tree={s=2mm, l-=5mm},
	[S [NP[Pronoun\\\emph{she}]] [VP [V\\\emph{donated}] [NP [Det\\\emph{that}] [NP [N\\\emph{dress}] [PP [P\\\emph{to}] [NP [Det\\\emph{the}] [N\\\emph{shop}]]]]]]]
    \end{forest}}
    \label{Q2_treesA}
    \ex{
    \small \begin{forest}
for tree={s=2mm, l-=5mm},
	[S [NP[Pronoun\\\emph{she}]] [VP [V\\\emph{donated}]
    [NP 
        [Det\\\emph{that}]
        [N\\\emph{dress}]
        [P\\\emph{to}]
        [NP [Det\\\emph{the}] [N\\\emph{shop}]]]]]
    \end{forest}}
    \label{Q2_treesB}
\ex{
    \small \begin{forest}
for tree={s=2mm, l-=5mm},
	[S [NP[Pronoun\\\emph{she}]] [VP [V\\\emph{donated}] [NP [Det\\\emph{that}] [NP [N\\\emph{dress}]]] [PP [P\\\emph{to}] [NP [Det\\\emph{the}] [N\\\emph{shop}]]]]]
    \end{forest}}
    \label{Q2_treesC}
    \end{xlist}}
    \label{Q2_trees}
\end{exe}

\begin{answer}
{
(\ref{Q2_treesC}) is the most adequate.\\
Evidence against both (\ref{Q2_treesA}) and (\ref{Q2_treesB}) is that they both represent \emph{that dress to the shop} as a constituent here, but there is no constituency test that it passes (\emph{*That dress to the shop, she donated; Q: *What did she donate? A: that dress to the shop}).\\
Even more convincingly, there \textbf{is} evidence that \emph{that dress} is a constituent (e.g. \emph{She donated it to the shop; That dress, she donated to the shop; Q: What did she donate to the shop? A: That dress}; etc).\\
In both (\ref{Q2_treesA}) and (\ref{Q2_treesB}), that string is \textbf{not} represented as a constituent, while in (\ref{Q2_treesC}) it is.\\
Why is that more convincing? Because we know that sometimes even constituents can fail constituency tests (``False negatives''); so conceivably \emph{that vase on the table} might somehow be failing constituency tests for some extraneous reason.
But we don't have an equivalent problem of ``False positives'' (you can think about why there might be such an asymmetry), so we can be more confident that \emph{that vase} really \textbf{is} a constituent.\\
Finally, the tree in~(\ref{Q2_treesB}) misses the fact that \emph{to the shop} is a PP constituent. For some users of English, the following is grammatical: \emph{Where did she donate that dress? To the shop.}
}
\end{answer}


\section*{Question 3}
\hfill{}
\ref{syn_sem_roles} $\Box$,
\ref{sem_roles} $\Box$

As we know, some verbs in English can appear either with two syntactic arguments, or just one:
\begin{exe}
    \ex{
    \begin{xlist}
        \ex{This film shocks.\tabto{10em} This film shocks a lot of people.}
        \ex{The donkey kicked.\tabto{10em} The donkey kicked the door.}
        \ex{The canary drank.\tabto{10em} The canary drank some water.}
    \end{xlist}}
    \label{semantic_rolesA}
    \ex{
    \begin{xlist}
        \ex{The branch broke.\tabto{10em} The branch broke the window.}
        \ex{The cup dropped.\tabto{10em} The child dropped the bag.}
        \ex{The lake thawed.\tabto{10em} The sun thawed the ice.}
    \end{xlist}}
    \label{semantic_rolesB}
\end{exe}
Thinking of the ways in which participant roles (like AGENT or THEME) are mapped onto syntactic functions (like SUBJECT or OBJECT):

\paragraph{Q3A} What is the difference between the verbs in (\ref{semantic_rolesA}) and those in (\ref{semantic_rolesB})? Describe it in terms of the syntactic and semantic arguments of the predicates.

\begin{answer}
{
The point here is that all the verbs in (\ref{semantic_rolesA}) consistently assign the same thematic/participant role to their subject, regardless of whether they additionally select another argument or not.\\
But the verbs in (\ref{semantic_rolesB}) are such that the subject of the intransitive verb gets the same thematic/participant role as the object of the transitive version.\\
Verbs in the class in (\ref{semantic_rolesB}) are sometimes described as participating in the \emph{causative alternation}, which has been a major research focus since the 60s.\\
The alternation in (\ref{semantic_rolesA}) is less thoroughly studied, but sometimes called \emph{unspecified object deletion}.
}
\end{answer}

\paragraph{For discussion:} Can you think of other English verbs that behave like the verbs in (\ref{semantic_rolesA})? And like the verbs in (\ref{semantic_rolesB})?

\paragraph{For discussion:} If you speak or are familiar with another language, are there verbs in that language that behave to some extent (or exactly) like the verbs in (\ref{semantic_rolesB})? If you find a pattern that is similar but not identical, can you describe it?


\end{document}
