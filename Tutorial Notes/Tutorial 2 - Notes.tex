\documentclass{article}
\usepackage{xr-hyper} %Adds referencing between handouts and the Skills.tex document to avoid typos (req. latexmkrc)
\externaldocument{Skills} %where to look for labels
\usepackage[hidelinks]{hyperref} %links and URLS
\usepackage[linguistics]{forest} %needs tikz, draws trees
\usepackage[margin=1in]{geometry} %page layout
\usepackage{graphicx} % Required for inserting images
\usepackage[T1]{fontenc} %Make sure to be able to get accented characters etc
\usepackage[utf8]{inputenc}
\usepackage[normalem]{ulem} %adds strikethrough and other commands
\setlength{\parindent}{0pt}%don't indent paragraphs...
\setlength{\parskip}{1ex plus 0.5ex minus 0.2ex} 
\usepackage{multicol} %adds columns
\usepackage{gb4e} %for formatting examples, works with leipzig and multicol
\primebars %setting for gb4e, adds bars for X-bar notation, allows switch between bar or %'%
\noautomath
\usepackage{tabto}
\usepackage{amssymb}
\usepackage{fancyhdr}
\usepackage{setspace}
\usepackage{pifont} %allows dingbats to be called (for the "crosses" and "ticks" defined below)
\usepackage{tipa} % IK


\usepackage{leipzig}%primarily used for the abbreviations

\usepackage[backend=biber,
            style=unified,
            natbib,
            maxcitenames=3,
            maxbibnames=99]{biblatex}
\addbibresource{references.bib}
\usepackage{attrib}%allows authors next to quote environments

\makeatletter
\def\@maketitle{%I guessed from the commenting out of the author below that you don't want an author, this just gets rid of the space associated with the author field
  \newpage
  \null
%  \vskip 2em%
  \begin{center}%
  \let \footnote \thanks
    {\LARGE {\@title}\par}
%    \vskip 1.5em%
%    {\large
%      \lineskip .5em%
%      \begin{tabular}[t]{c}%
%        \@author
%      \end{tabular}\par}%
    \vskip 1em%
    {\large \@date}%
  \end{center}%
  \par
%  \vskip 1.5em
}
\makeatother

\title{LEL2A: Syntax}
%\author{Instructor: Itamar Kastner}
\date{Semester 1, 2024-25}%changed to current academic year

\newcommand*{\sqb}[1]{\lbrack{#1}\rbrack}
\newcommand*{\fn}[1]{\footnote{#1}}
\newcommand{\keyword}[1]{\textsc{#1}}
\newcommand{\cmark}{\ding{51}}
\newcommand{\xmark}{\ding{55}}
\newcommand{\subtitle}[1]{\maketitle\begin{center}{\Large #1}\end{center}}
\makeatletter
\newcommand*{\addFileDependency}[1]{% argument=file name and extension
\typeout{(#1)}% latexmk will find this if $recorder=0
% however, in that case, it will ignore #1 if it is a .aux or 
% .pdf file etc and it exists! If it doesn't exist, it will appear 
% in the list of dependents regardless)
%
% Write the following if you want it to appear in \listfiles 
% --- although not really necessary and latexmk doesn't use this
%
\@addtofilelist{#1}
%
% latexmk will find this message if #1 doesn't exist (yet)
\IfFileExists{#1}{}{\typeout{No file #1.}}
}\makeatother

\newcommand*{\myexternaldocument}[1]{%
\externaldocument{#1}%
\addFileDependency{#1.tex}%
\addFileDependency{#1.aux}%
}
\myexternaldocument{Skills} %also necessary for cross referencing, to reference other documents duplicate with name of document

\forestset{
  nice nodes/.style={
    for tree={
      inner sep=0pt,
      fit=band,
    },
  },
  default preamble=nice nodes,
}
\begin{document}
\maketitle
\subtitle{Tutorial Notes Week 6: Topics 4, 5 \& 6}

\paragraph{Explanation on accumulating Syntax Skills:} Let me stress that the point of the syntax block is to learn how to think like a syntactician, learn about English, and try to have fun. But marks are involved, and here's how we mark Syntax Skills: generally speaking, it's enough to show them once. This means that if we have questions Q1A and Q2B in a given tutorial sheet (for example), both of which target skill 6b, then getting one of them right is enough. That said, it's best to attempt all questions.

Some tutorial questions are primarily \textbf{for discussion}. This means that they can count towards your skill tally, but don't worry if you don't get them right -- their goal is to allow for extended discussion in the tutorial itself and some will go beyond the discussion in class.

(Please remember that we're developing this approach to tutorials and marking together, so your feedback is welcome -- keep it specific and try to give it directly to Itamar.)

\section*{Question 1}
\hfill{} \ref{c_selection} $\Box$, \ref{adjunct_complement} $\Box$, \ref{sem_roles} $\Box$

\paragraph{Q1A} For each of the examples in~(\ref{ungrammatical1}), explain why the sentence is not grammatical using the framework developed so far in the course.
Though you may find it useful, it is not necessary to give corrected versions. The first one has been done for you -- longer answers are not necessary, and you don't have to identify the individual skills yourselves either, as long as you engage with the concepts properly.

\begin{exe}
    \ex{
    \begin{xlist}
        \ex[*]{We think his address\\
            \textcolor{red}{This sentence is ungrammatical because even though we correctly have a Theme argument for the verb \emph{think} (skill \ref{sem_roles}), we don't have the correct selection for a CP (skill \ref{c_selection}), e.g.~\emph{We think that his address is wrong}.}
        }
        % \ex[*]{John gave Bill}
        % \label{ungrammatical_gave}
        \ex[*]{Susan laughed the joke\\
        \textcolor{red}{The verb does not select for an NP (skill \ref{c_selection}); compare~(\ref{intransitive_laugh}) with \emph{Susan laughed}.
        We might be tempted to say that \emph{laugh} selects a PP complement which might be a Theme (skill \ref{sem_roles}, though we haven't discussed whether \emph{the joke} would count as a Theme) as in \emph{Susan laughed at the joke}, but this doesn't appear to hold up: we can see that PPs like \emph{at the joke} are adjuncts in \emph{Susan laughed quietly at the joke} and \emph{Susan laughed at her reflection and Bill at his.} (skill \ref{adjunct_complement})}\label{intransitive_laugh}
        }
        % \ex[*]{Bill liked}
        % \label{intransitive_like}
        % \ex[*]{Bill liked Anna the book}
        % \label{ditransitive_like}
        \ex[*]{Bill liked on Tuesday\\
        \textcolor{red}{This example feels incomplete, \emph{Bill liked what on Tuesday}? The \keyword{theme} role is unassigned here (skill~\ref{adjunct_complement}); we could also say that we need selection for an NP (skill~\ref{c_selection}). As part of the answer, we could also show that \emph{on Tuesday} is an adjunct and not an argument (skill~\ref{adjunct_complement}).}
        }
        \label{intransitive_like_adjunct}
    \end{xlist}
    }
    \label{ungrammatical1}
\end{exe}

\paragraph{For discussion} Here are a few more examples, similar to those Q1A, which you might want to attempt.
\ea
        \ea[*]{John gave Bill\\
        \textcolor{red}{The problem with (\ref{ungrammatical_gave}) is that \emph{give} must select three NP arguments (skill~\ref{c_selection}). A CP argument does not improve anything, *\emph{John gave that Bill was wrong}.
        Here, it's clear that we are missing an argument, e.g. \emph{John gave Bill the book} is grammatical.
        In this case, the verb selects three NP arguments with the following roles (skill~\ref{sem_roles}):
        \begin{itemize}
            \item \emph{John} the \keyword{agent}
            \item Something that is given, e.g. \emph{the book}, the \keyword{theme}
            \item A recipient of the thing that's given, e.g. \emph{Bill}, the \keyword{goal}
        \end{itemize}
        We can see that all three are \textbf{obligatorily} as *\emph{John gave the book} is also incomplete.}}
        \label{ungrammatical_gave}
        \ex[*]{Bill liked\\
        \textcolor{red}{In (\ref{intransitive_like}), we see again that verbs must have certain arguments: \emph{like} is transitive so requires a complement Theme (skill~\ref{sem_roles}) which is an NP (skill~\ref{c_selection}), e.g.~\emph{Bill likes the book}.}}
        \label{intransitive_like}
        \ex[*]{Bill liked Anna the book\\
        \textcolor{red}{In (\ref{ditransitive_like}), we see that verbs cannot take \emph{extra} arguments.
        That is, \emph{like} selects one internal argument, the \keyword{theme}, and one external argument, the \keyword{agent} (skill~\ref{sem_roles}). We are beginning to approach a generalisation that verbs have a set number of arguments that they must select. If we compare~(\ref{intransitive_like_adjunct}), (\ref{intransitive_like}) and now~\ref{ditransitive_like}), we see that \emph{like} must select one internal argument, and that this argument must be an NP and not a PP (skill~\ref{c_selection}).}
        }
        \label{ditransitive_like}
        \z
        \label{ungrammatical2}
\z

% \begin{exe}
%     \ex{
%     \begin{xlist}
%         \ex[*]{We think his address}\label{ungrammatical_think} % 4a (3b)
%         \ex[*]{John gave Bill}
%         \label{ungrammatical_gave} % 4a (3b)
%         \ex[*]{Susan laughed the joke}\label{intransitive_laugh} % 4a, 3b
%         \ex[*]{Bill liked}
%         \label{intransitive_like} % 4a, 3b
%         \ex[*]{Bill liked Anna the book}
%         \label{ditransitive_like} % 4a (3b)
%         \ex[*]{Bill liked on Tuesday}
%         \label{intransitive_like_adjunct} % 4a, 3b (4b)
%     \end{xlist}
%     }
%     \label{ungrammatical}
% \end{exe}

% \textcolor{red}{Both of the first two examples, (\ref{ungrammatical_think}-\ref{ungrammatical_gave}) seem incomplete.
% In the first example, we know that it's not because of a missing argument.
% In an example without \emph{his address}, but with \emph{it} instead the results are grammatical, \emph{we think it}.
% But, here \emph{it} doesn't refer to an NP, it has to refer to a \keyword{clause}, e.g. \emph{we think that his address is wrong}.
% So, the problem with (\ref{ungrammatical_think}), is \emph{think} selects CP complements for its internal argument.}

% \textcolor{red}{We can see a similar problem with the number of arguments in (\ref{intransitive_laugh}-\ref{ditransitive_like}).


\paragraph{Q1B} In contrast to (\ref{ungrammatical_gave}), (\ref{grammatical_gave}) is judged as grammatical.
For each of the constituents \sqb{\emph{quickly}}, \sqb{\emph{Bill}}, \sqb{\emph{the book}}, \sqb{\emph{in the street}}, and \sqb{\emph{on Tuesday}}, say if it is a complement of the verb or an adjunct. Use diagnostics to support your argument. The first one has been done for you. Get at least two out of the four right to obtain Syntax Skill \ref{adjunct_complement}.
\begin{exe}
    \ex{John gave Bill the book quickly in the street on Tuesday}
    \label{grammatical_gave}
\end{exe}

\textcolor{red}{\textbf{Quickly} is an adjunct, since it allows \emph{do-so} substitution: \emph{John gave Bill the book quickly in the street on Tuesday, and Paul did so slowly (at work on Wednesday}. To give another example, \emph{quickly} cannot appear before the argument \emph{Bill}: *\emph{John gave quickly Bill the book.}}

\textcolor{red}{There are multiple possible ways to get this answer, one is given below:
\begin{exe}
    \sn{John gave Bill the book quickly in the street on Tuesday
    \begin{xlist}
        \ex[]{and Paul did so on Thursday}
        \ex[]{and Paul did so at work (on Monday)}
        \ex[]{and Paul did so slowly (at work on Wednesday)}
        \ex[*]{and Paul did so the mug (quickly in the street on Tuesday)}
        \ex[*]{and Paul did so Mary (the mug quickly in the street on Tuesday)}
    \end{xlist}
    }
\end{exe}
We can see that \sqb{\emph{quickly}}, \sqb{\emph{in the street}}, and \sqb{\emph{on Tuesday}} all behave like adjuncts as they allow \emph{do so} substitution.
However, \sqb{\emph{Bill}} and \sqb{\emph{the book}} both behave like complements as they cannot be substituted by \emph{do so}.}

\paragraph{For discussion:} Based on the data in (\ref{ungrammatical1})--(\ref{grammatical_gave}) above, and supplementing with any grammatical counterparts you feel necessary, what can we say about the relationship between c-selection/complementation and semantic roles?

\textcolor{red}{So far, we have seen that different verbs have selection requirements and that they \textbf{must} select the correct number of arguments.
An intransitive verb only has one argument, a transitive verb must have two arguments, and a ditransitive verb must have three arguments.
We also saw that \keyword{adjuncts} can appear in any number.
This is important as it means that cannot satisfy a verb's selection requirements (\keyword{adjuncts} don't count).
In the examples in (\ref{ungrammatical1})--(\ref{grammatical_gave}) above, NPs are \keyword{complements} and \keyword{adjuncts} are PPs and adverbs.
It also appears that certain semantic roles, \textsc{agent}, \textsc{theme}, \& \textsc{goal}, are associated with NP complements and others, \textsc{location} etc., are associated with adjuncts.}

\paragraph{For discussion:} For (\ref{grammatical_put}), are the strings, \emph{the book} and \emph{on the table}, complements or adjuncts? 
Do you need to reformulate your answer to the previous question in order to accommodate (\ref{grammatical_put})?
\begin{exe}
    \ex{John won't put the book on the table}
    \label{grammatical_put}
\end{exe}


\textcolor{red}{As above, there are multiple ways to get the right answers here, but here is one possible way:
\begin{exe}
    \sn{John won't put the book on the table
    \begin{xlist}
        \ex[*]{and Mary won't do so on the shelf}
        \ex[*]{and Mary won't do so the pen (on the shelf)}
    \end{xlist}
    }
\end{exe}
Both \emph{the book} and \emph{on the table} appear to be \keyword{complements}.
If above we said that PPs were uniformly \keyword{adjuncts}, we now need to revise our generalisation.
The verb \emph{put} selects an NP \keyword{theme} and a PP \keyword{goal} as \keyword{complements}.
Notice that an NP \keyword{goal} doesn't work *\emph{John won't put the book the table}, so this must be more than $\theta{}$-roles, and that changing the order doesn't change the roles assigned \emph{it is on the table that John won't put the book}.
Our generalisation may now be something like:
\begin{exe}
    \sn{Generalisation:
    \begin{xlist}
        \ex[]{Each verb is associated with a set of selection requirements, that selects the \emph{number} and \emph{category} of \keyword{complements} (\keyword{c-selection})}
        \ex[]{Each verb is associated with a set of $\theta{}$-roles, that are assigned to \keyword{complements}}
        \ex[]{\keyword{adjuncts} do not count towards satisfying \keyword{c-selection} or assignment of $\theta{}$-roles}
    \end{xlist}
    }
\end{exe}}

\section*{Question 2}
\hfill{} \ref{projection} $\Box$, \ref{functional_heads} $\Box$, \ref{VPinternal_subjects} $\Box$

\paragraph{Q2A} The tree in (\ref{spicy_food}) fails to account for one or more acceptability judgments which would typically be held by speakers of English.
Think about the way the tree in (\ref{spicy_food}) represents constituency (skills~\ref{projection}, \ref{functional_heads}).
What claims does it make about the constituents in the sentence \emph{the guests may like spicy food}? Identify one incorrect claim, and use at least one constituency test to exemplify it.

If you applied different constituency tests, how do the results of these diverge from those claims?
% If you are not sure about the data, find a language consultant (either among your peers or consult one of the course tutors).

Remember that the point of this question is thinking about constituency. It asks a specific question about what is empirically wrong with the tree, not whether it matches our theory.
\begin{exe}
    \ex{
    \begin{forest}
for tree={s=2mm, l-=5mm},
	[IP 
 [NP[the guests, roof]] [VP 
 [V [V\\may] [V\\like]] [NP [very spicy food, roof]] ]
 ]
    \end{forest}
    }
    \label{spicy_food}
\end{exe}

\textcolor{red}{
The point of this question is thinking about constituency.
It asks you a fairly specific question (what is empirically wrong with the trees).
It \textbf{doesn't} ask you whether the trees match our theory (e.g.\ X$'$-theory).
There multiple ways to answer this question, but here is one obvious acceptability judgment for which the tree make the wrong predictions:
\begin{itemize}
    \item Tree (\ref{spicy_food}) represents the hypothesis (among other things) that \emph{may like} is a constituent.
    \item That \emph{like very spicy food} is not.
    \item This predicts that \emph{like very spicy food} can not be substituted for \emph{do so}.
    \item But this is not the case: \emph{The guests may like very spicy food and the staff may do so too} is perfectly acceptable.
    \item So the tree fails to account for the judgment that \emph{The guests may like very spicy food and the staff may do so too} is fully acceptable.
\end{itemize}
}

\paragraph{Q2B} Draw an elementary tree for \emph{like}. Which positions in the tree do the roles of \emph{agent} and \emph{theme} merge into (skill~\ref{projection})? 

\textcolor{red}{
As we have a better idea of the constituent structure, we are in a better position to start building the tree and we can do this in stages, starting with an elementary tree of the verb:
\begin{center}
    \begin{forest}
    [VP, nice empty nodes
    [NP\textsubscript{\textsc{agent}}][V$'$
    [V\\like][NP\textsubscript{\textsc{theme}}]]
    ]
\end{forest}
\end{center}
}

For now, do not worry about the internal structure of the arguments, add \emph{the guests} and \emph{very spicy food} to the tree for \emph{like} and label them as NPs.

\textcolor{red}{
As we have a better idea of the constituent structure, we are in a better position to start building the tree and we can do this in stages, starting with an elementary tree of the verb:
\begin{center}
    \begin{forest}
    [VP, nice empty nodes
    [NP [the guests, roof]][V$'$
    [V\\like][NP [very spicy food, roof]]]
    ]
\end{forest}
\end{center}
}

Now draw the elementary tree for \emph{may} (recall that we are assuming that modals like \emph{may} and auxiliaries like \emph{be} all occupy the category I, part of skill~\ref{functional_heads}).
Which type of phrase(s) does \emph{may} take as a complement?

\textcolor{red}{
So, what about the \keyword{auxiliary}?
We said in the Course Notes that, this should go in an I head and take a VP complement.
\begin{center}
    \begin{forest}
    [IP, nice empty nodes
    [][I$'$
    [I\\may][VP]]
    ]
\end{forest}
\end{center}
}

Finally, combine the elementary trees to derive a tree for the sentence \emph{the guests may like very spicy food} consistent with the framework developed so far (skills \ref{functional_heads} and \ref{VPinternal_subjects}).

\textcolor{red}{
At this stage we have an IP tree that takes a VP complement and VP with two arguments.
We can combine these quite straightforwardly:
\begin{center}
    \begin{forest}
    [IP, nice empty nodes
    [][I$'$
    [I\\may][VP
    [NP [the guests, roof]][V$'$
    [V\\like][NP [very spicy food, roof]]]]]
    ]
\end{forest}
\end{center}
The VP-internal subject then raises to the Specifier of IP, as explained next.
}

\paragraph{For discussion:} How is the idiom \emph{the cat may be out of the bag} (which means roughly `the secret has been revealed') relevant to determining the structure projected by \emph{may}? 

\textcolor{red}{
Our current tree for Q2B has the wrong surface order, \emph{may the guests like very spicy food}.
We used idioms like \emph{the cat may be out of the bag}, to show that auxiliaries preserve idiomatic meanings.
We probably don't want a lexical entry for every idiomatic phrase with every possible combination of auxiliary.
For this reason, we said that \emph{the cat is out of the bag} is stored and the external argument moves to \keyword{spec} IP:
\begin{center}
    \begin{forest}
    [IP, nice empty nodes
    [NP [the guests, roof, name=copy]][I$'$
    [I\\may][VP
    [NP [\sout{the guests}, roof, name=trace]][V$'$
    [V\\like][NP [very spicy food, roof]]]]]
    ]
    \draw[->,dotted] (trace) to[out=south west,in=south] (copy);
\end{forest}
\end{center}
}

\section*{Question 3}
\hfill{} \ref{functional_heads} $\Box$, \ref{V_adjunction} $\Box$

\paragraph{Q3}Returning to the sentences given in (\ref{grammatical_gave}) and (\ref{grammatical_put}), repeated below, draw trees to represent them. One tree would be enough for both Syntax Skills (if you get them both right).
\begin{exe}
    \exr{grammatical_gave}{John gave Bill the book quickly in the street on Tuesday.}
    \exr{grammatical_put}{John will not put the book on the table.}
\end{exe}

\textcolor{red}{We know that in (\ref{grammatical_gave}), \emph{Bill} and \emph{the book} are complements.
This means they should be sisters to \obar{V}.
We also know that \emph{quickly}, \emph{in the street}, and \emph{on Tuesday} are adjuncts.
This means they should be sisters to \ibar{V} levels.
We also know (see above) that it's possible to substitute \emph{do so} for the verb and no adjuncts, the verb plus one adjunct, the verb plus two adjuncts, \dots{}.
This means the adjuncts are not sisters to each other but to \ibar{V} levels, and the other adjuncts are daughters of those \ibar{V} levels.
In (\ref{grammatical_put}), we have a negative auxiliary, \emph{won't}.
In the notes we proposed a head above VP, Neg, to put \emph{not} and this is what's shown in the tree.
At this point, we don't have enough evidence to work out how \emph{will not} becomes \emph{won't} or if this is even a question for syntax.
So, these have been separated in the tree.
Putting this all together gives us the two trees below.}
\begin{center}
    \begin{forest}
    [IP%, nice empty nodes
    [NP [John, roof, name=copy]][I$'$
    [I\\\lbrack{}\textsc{+pst}\rbrack{}][VP
    [NP [\sout{John}, roof, name=trace]][V$'$ [V$'$[V$'$ [V$'$
    [V\\gave][NP, [Bill, roof]][NP [the book, roof]]][AdjP [quickly, roof]]][PP [in the street, roof]]][PP [on Tuesday, roof]]]]]
    ]
    \draw[->,dotted] (trace) to[out=south west,in=south] (copy);
\end{forest}
\end{center}

\begin{center}
    \begin{forest}
    [IP%, nice empty nodes
    [NP [John, roof, name=copy]][I$'$
    [I\\will][NegP [NegP$'$ [Neg\\not][VP
    [\sout{NP} [\sout{John}, roof, name=trace]] [V$'$
    [V\\put][NP, [the book, roof]][PP [P$'$ [P\\on][NP [the table, roof]]]]]]]]]]
    \draw[->,dotted] (trace) to[out=south west,in=south] (copy);
\end{forest}
\end{center}

\textcolor{red}{Both of these examples deal with something we haven't come across in the notes yet, ditransitive verbs.
That these verbs have two internal arguments means that V in our trees has two sisters.
This fits with out constituency tests but violates \keyword{binary branching}, a principle that's gone otherwise unstated in the notes so far.
At this stage, we are not in a position to motivate an analysis that preserves \keyword{binary branching}, but below are versions of the trees that do.
In the first example (\ref{grammatical_gave}), this involves decomposing the verb \emph{give} into two parts, \textsc{cause} and \textsc{have}, and associating these with separate verbal heads.
The higher verbal head is labelled \emph{v} (little-\emph{v}) and selects a VP complement.
The verb in V then undergoes \keyword{head-movement} to adjoin to \emph{v}.
The key text on this is \citet{larson_double_1988} but relies on ideas not covered yet in notes.
We will return to this in Topic 10.
}
\begin{center}
    \begin{forest}
    [IP%, nice empty nodes
    [NP [John, roof, name=copy]][I$'$
    [I\\\lbrack{}\textsc{+pst}\rbrack{}][\emph{v}P
    [NP [\sout{John}, roof, name=trace]][\emph{v}$'$ [\emph{v}$'$[\emph{v}$'$ [\emph{v}$'$
    [\emph{v} [V\textsubscript{\textsc{have}}\\gave, name=vcopy][\emph{v}\textsubscript{\textsc{cause}}]][VP [NP, [Bill, roof]] [V$'$ [V\textsubscript{\textsc{have}}\\\sout{gave}, name=vtrace] [NP [the book, roof]]]]][AdjP [quickly, roof]]][PP [in the street, roof]]][PP [on Tuesday, roof]]]]]
    ]
    \draw[->,dotted] (trace) to[out=south west,in=south] (copy);
    \draw[->,dotted] (vtrace) to[out=south west,in=south] (vcopy);
\end{forest}
\end{center}

\begin{center}
    \begin{forest}
    [IP%, nice empty nodes
    [NP [John, roof, name=copy]][I$'$
    [I\\will][NegP [NegP$'$ [Neg\\not][\emph{v}P
    [\sout{NP} [\sout{John}, roof, name=trace]] [\emph{v}$'$
    [\emph{v} [V\\put, name=vcopy][\emph{v}]][VP [NP, [the book, roof]][V$'$ [V\\\sout{put}, name=vtrace] [PP [P$'$ [P\\on][NP [the table, roof]]]]]]]]]]]]
    \draw[->,dotted] (trace) to[out=south west,in=south] (copy);
    \draw[->,dotted] (vtrace) to[out=south west,in=south] (vcopy);
\end{forest}
\end{center}

\paragraph{For discussion:} What changes would you need to make to your tree to accommodate the changes shown in (\ref{cp_put})?
\begin{exe}
    \ex{}{It seems that John won't put the book on the table}
    \label{cp_put}
\end{exe}
\textcolor{red}{The example in (\ref{cp_put}) has an embedded clause:
\begin{exe}
    \sn \lbrack{}\textsubscript{CP} It seems \lbrack{}\textsubscript{CP} that \lbrack{}\textsubscript{IP} John won't put the book on the table\rbrack{}\rbrack{}\rbrack{}
\end{exe}
The embedded clause in (\ref{cp_put}) is (\ref{grammatical_put}) from above.
In the Topic 6 notes, we saw that some verbs select clausal complements and that these are headed by a \keyword{complementizer}.
In (\ref{cp_put}), \emph{that} is the \keyword{complementizer} selected by \emph{seem} and this in turn selects the IP.
This is represented below with \keyword{binary branching} as above but \keyword{ternary branching} for embedded \emph{put} and its complements would be acceptable here too.}
\begin{center}\small
    \begin{forest}
    [IP
    [NP [it, roof, name=copy1]][I$'$
    [I\\\lbrack{}\textsc{-pst}\rbrack{}][VP
    [\sout{NP} [\sout{it}, roof, name=trace1]][V$'$
    [V\\seems][CP[C$'$
    [C\\that][IP%, nice empty nodes
    [NP [John, roof, name=copy]][I$'$
    [I\\will][NegP [NegP$'$ [Neg\\not][\emph{v}P
    [\sout{NP} [\sout{John}, roof, name=trace]] [\emph{v}$'$
    [\emph{v} [V\\put, name=vcopy][\emph{v}]][VP [NP, [the book, roof]][V$'$ [V\\\sout{put}, name=vtrace] [PP [P$'$ [P\\on][NP [the table, roof]]]]]]]]]]]]
    ]]]]]]
    \draw[->,dotted] (trace) to[out=south west,in=south] (copy);
    \draw[->,dotted] (trace1) to[out=south west,in=south] (copy1);
    \draw[->,dotted] (vtrace) to[out=south west,in=south] (vcopy);
\end{forest}
\end{center}

\printbibliography
\end{document}