\documentclass{article}
\usepackage{xr-hyper} %Adds referencing between handouts and the Skills.tex document to avoid typos (req. latexmkrc)
\externaldocument{Skills} %where to look for labels
\usepackage[hidelinks]{hyperref} %links and URLS
\usepackage[linguistics]{forest} %needs tikz, draws trees
\usepackage[margin=1in]{geometry} %page layout
\usepackage{graphicx} % Required for inserting images
\usepackage[T1]{fontenc} %Make sure to be able to get accented characters etc
\usepackage[utf8]{inputenc}
\usepackage[normalem]{ulem} %adds strikethrough and other commands
\setlength{\parindent}{0pt}%don't indent paragraphs...
\setlength{\parskip}{1ex plus 0.5ex minus 0.2ex} 
\usepackage{multicol} %adds columns
\usepackage{gb4e} %for formatting examples, works with leipzig and multicol
\primebars %setting for gb4e, adds bars for X-bar notation, allows switch between bar or %'%
\noautomath
\usepackage{tabto}
\usepackage{amssymb}
\usepackage{fancyhdr}
\usepackage{setspace}
\usepackage{pifont} %allows dingbats to be called (for the "crosses" and "ticks" defined below)
\usepackage{tipa} % IK


\usepackage{leipzig}%primarily used for the abbreviations

\usepackage[backend=biber,
            style=unified,
            natbib,
            maxcitenames=3,
            maxbibnames=99]{biblatex}
\addbibresource{references.bib}
\usepackage{attrib}%allows authors next to quote environments

\makeatletter
\def\@maketitle{%I guessed from the commenting out of the author below that you don't want an author, this just gets rid of the space associated with the author field
  \newpage
  \null
%  \vskip 2em%
  \begin{center}%
  \let \footnote \thanks
    {\LARGE {\@title}\par}
%    \vskip 1.5em%
%    {\large
%      \lineskip .5em%
%      \begin{tabular}[t]{c}%
%        \@author
%      \end{tabular}\par}%
    \vskip 1em%
    {\large \@date}%
  \end{center}%
  \par
%  \vskip 1.5em
}
\makeatother

\title{LEL2A: Syntax}
%\author{Instructor: Itamar Kastner}
\date{Semester 1, 2025--26}%changed to current academic year

\newcommand*{\sqb}[1]{\lbrack{#1}\rbrack}
\newcommand*{\fn}[1]{\footnote{#1}}
\newcommand{\keyword}[1]{\textsc{#1}}
\newcommand{\cmark}{\ding{51}}
\newcommand{\xmark}{\ding{55}}
\newcommand{\subtitle}[1]{\maketitle\begin{center}{\Large #1}\end{center}}
\newcommand\blue[1]{\textcolor{blue}{#1}} % Itamar is lazy (I am Itamar)
\makeatletter
\newcommand*{\addFileDependency}[1]{% argument=file name and extension
\typeout{(#1)}% latexmk will find this if $recorder=0
% however, in that case, it will ignore #1 if it is a .aux or 
% .pdf file etc and it exists! If it doesn't exist, it will appear 
% in the list of dependents regardless)
%
% Write the following if you want it to appear in \listfiles 
% --- although not really necessary and latexmk doesn't use this
%
\@addtofilelist{#1}
%
% latexmk will find this message if #1 doesn't exist (yet)
\IfFileExists{#1}{}{\typeout{No file #1.}}
}\makeatother

\newcommand*{\myexternaldocument}[1]{%
\externaldocument{#1}%
\addFileDependency{#1.tex}%
\addFileDependency{#1.aux}%
}
\myexternaldocument{Skills} %also necessary for cross referencing, to reference other documents duplicate with name of document

\begin{document}

Opening puzzles for each topic:

\section{Topic 1 (what is syntax)}

Why is this tweet funny? What does it reveal about the grammar of the original headline?

\begin{center}
\includegraphics[width=0.6\linewidth, alt={Screenshot of a Tweet. A user posted a story with a photo of a sea cucumber and the headline: ``Oceans: Scientists find 30 potential new species at bottom of deep sea using robots''. The tweet reads: ``This is worrying. We've only started using robots ourselves.''}]{Images/deep-sea-robots.png}
\end{center}

If you enjoyed that, you can try also the following:

\begin{center}
\includegraphics[width=0.8\linewidth, alt={Screenshot of a Facebook post. A user posted a story with two close-up photos of ants carrying drops of water. The headline is: ``Woman Takes Photos Of Ants In Her GArden Using A Smartphone And They're Award-Winning''. A comment reads: ``it does not look like they are using a smartphone. It looks like the ants are carrying drops of water.'', accompanied by many `laugh', `like' and `love' reactions.}]{Images/award-winning-ants.jpg}
\end{center}

\section{Topic 2 (constituents)}
The sentences in the~(a) examples are fine but the ones in the~(b) examples aren't. How come?

\ea \label{ex:top2a}
    \ea[] {We mashed the potatoes but roasted the swedes.}  \label{ex:2aok}
    \ex[*] {We mashed the but roasted the swedes.} \label{ex:2abad}
    \z
\ex \label{ex:top2b}
    \ea[] {You take the high road and I'll take the low road.}  \label{ex:2bok}
    \ex[*] {You take and I'll take the low road.} \label{ex:2bbad}
    \z
\z

\section{Topic 3 (arguments)}
Different languages have different ways of saying that it's raining. What do these all have in common?

\ea \label{ex:top3}
    \ea \textsc{English}\\
        It's raining.
    \ex \textsc{Standard German}
        \gll Es regnet. \\
        it rain.\textsc{pres.3sg} \\
    \ex \textsc{Gaelic} 
        \gll Tha an t-uisge ann. \\
         is the \textsc{DEF}-water in.it \\
    \ex \textsc{Modern Hebrew}
        \gll jored ge\textipa{S}em. \\
        descend.\textsc{pres.3sg.m} rain\textsc{<M>} \\
    \ex \textsc{Palestinian Arabic}
        \gll ad-dunja b-ti\textipa{S}ti \\
        the-world\textsc{<F>} \textsc{pres}-rain.\textsc{3.f} \\
    \z
\z

\section{Topic 4 (modifiers)}
In~(\ref{ex:top4a}) we have a basic clause. Let's think about what we can add to it.

\ea Kim bought shoes. \label{ex:top4a}
\z

What more can we add? How long can this this clause get?

\ea
    \ea Kim bought shoes online.
    \ex Kim bought shoes before noon.
    \ex Kim bought shoes online before noon.
    \ex Eager to fill a gap in their enormous collection, Kim quickly bought a new pair of shoes online before noon to without realising they already owned the same pair.
    \ex \dots{}
    \z
\z

What \emph{couldn't} we add?


\section{Topic 5 (X-bar)}
Here are some examples from Japanese (simplified for our purposes; some are taken from acquisition studies with children). What is the basic word order in the language? Think about the order of subject, object and verb. How is it different to that of English?

\ea
    \ea \gll Taro-ga ringo-o tabeta\\
            Taro-SUBJ apple-OBJ	ate\\
        \glt `Taro ate an apple.'
    \ex \gll kame-san-ga ahiru-san-o osimasita\\
            turtle-SAN-SUBJ duck-SAN-OBJ pushed\\
        \glt `Mr Turtle pushed Mr Duck.'
    \ex \gll gorira-ga (raion-ni) tegami-o kaku\\
            gorilla-SUBJ (lion-DATIVE) letter-OBJ writes\\
        \glt `A gorilla is writing a letter (to a lion).'
    \z
\z


\section{Topic 6 (IP)}
Consider where the subjects appear in the following examples, acceptable to at least some users of Belfast English:

\ea
    \ea There should some students get distinctions.
    \ex There have lots of students missed the classes.
    \z
\z

How would we say this in ``standard'' English? Why are the subjects where they are?

\section{Topic 7 (NP)}
Just as verbs can have arguments (complements) and adjuncts (modifiers), we'll see that so can nouns. Consider the examples in~(\ref{ex:top7a})--(\ref{ex:top7b}). There is reason to think that in~(\ref{ex:top7a}) the PPs are arguments, whereas in~(\ref{ex:top7b}) the PPs are adjuncts. Can you come up with syntactic tests to show this?

\ea \label{ex:top7a}
    \ea That student [of linguistics].
    \ex That student [of physics].
    \z
\ex \label{ex:top7b}
    \ea That student [from Belgium].
    \ex That student [with the long hair].
    \z
\z

\section{Topic 8 (nonverbal XPs)}
In the puzzle for Topic 5, we saw that the basic word order of Japanese is SOV. Let's now look at other constituents. What's the generalization about the order of words within a phrase in Japanese?
\ea
    \ea \gll Taro-ga kuruma	da Kobe ni itta\\
            Taro-SUBJ car by Kobe to went\\
        \glt `Taro went to Kobe by car'
    \ex \gll Taro-ga Hanako ni yasashii\\
            Taro-SUBJ Hanako to kind\\
        \glt `Taro is kind to Hanako'
    \ex \gll Taro-ga Hanako no shashin-o mita\\
            Taro-SUBJ Hanako of picture-OBJ saw\\
        \glt `Taro saw a picture of Hanako'
    \z
\z


\section{Topic 9 (control and raising)}
In this Topic we will look at some advanced issues in the study of non-finite clauses. We can start by considering the verbs \emph{seem} and \emph{want}. When they embed an active non-finite clause, the meaning is roughly similar in that Doctor Jex-Blake seems to or wants to examine every patient - she's still the subject in both examples:
\ea
    \ea Doctor Jex-Blake seemed [to examine every patient].
    \ex Doctor Jex-Blake wanted [to examine every patient].
    \z
\z

But if the embedded non-finite clause is passive, we find another difference between the examples. Who's the subject of the embedded clause now in each case? Why might that be?

\ea 
    \ea Every patient seemed [to be examined by Doctor Jex-Blake].
    \ex Every patient wanted [to be examined by Doctor Jex-Blake].
    \z
\z


\section{Topic 10 (object control, why raise)}
If we look at sentences with verbs like \emph{seem}, we find that certain elements can appear as the matrix subject. On closer inspection, these aren't the kinds of subjects we usually run into. In~(\ref{ex:top10a}) Jenny doesn't do any ``seeming'', and in~(\ref{ex:top10a}b) there is no referent \emph{it} that does the ``seeming''.
\ea \label{ex:top10a}
    \ea Jenny seems to enjoy politics.
    \ex It seems that Jenny enjoys politics.
    \z
\z

In addition, we can also get~(\ref{ex:top10b}), where \emph{there} does the ``seeming''.
\ea \label{ex:top10b} There seem to be bunnies in the shop.
\z

What is going on here? What do these clauses look like in other languages you might be familiar with?


\section{Topic 11 (passives)}
Here's a classic contrast from the literature on the syntax of passives:
\ea[]{\label{ex:top11a}The ship was sunk to collect the insurance.}
\z
\ea[*]{\label{ex:top11b}The ship sank to collect the insurance.}
\z

What does this contrast tell us about the agents of the passive clause in~(\ref{ex:top11a}) and in the anticausative clause in~(\ref{ex:top11b})? What might their syntax and/or semantics be like?


\section{Topic 12 (wh)}

Here's a \emph{wh}-question in Bulgarian. What's remarkable about it when compared to its English translation?
\ea[]{\gll Koj kude udari Ivan?\\
         who where hit Ivan \\
         \trans `Who hit Ivan where?'}
\z

Interestingly, we can't swap the order of the two \emph{wh}-phrases, even though the meaning would've still been the same. Why might that be?
\ea[*]{\gll Kude koj kude udari Ivan?\\
         where who hit Ivan \\
         \trans (intended: `Who hit Ivan where?')}
\z




\end{document}

begin{exe}
    \ex[]{
    \begin{xlist}
        \ex[]{Anna read a book.}
        \ex[]{Which book did Anna read?}
        \ex[]{Anna read a book and the newspaper.}
        \ex[*]{Which book did Anna read and the newspaper?}
    \end{xlist}
    }
\end{exe}