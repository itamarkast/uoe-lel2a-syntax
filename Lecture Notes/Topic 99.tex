\documentclass{article}
\usepackage{xr-hyper} %Adds referencing between handouts and the Skills.tex document to avoid typos (req. latexmkrc)
\externaldocument{Skills} %where to look for labels
\usepackage[hidelinks]{hyperref} %links and URLS
\usepackage[linguistics]{forest} %needs tikz, draws trees
\usepackage[margin=1in]{geometry} %page layout
\usepackage{graphicx} % Required for inserting images
\usepackage[T1]{fontenc} %Make sure to be able to get accented characters etc
\usepackage[utf8]{inputenc}
\usepackage[normalem]{ulem} %adds strikethrough and other commands
\setlength{\parindent}{0pt}%don't indent paragraphs...
\setlength{\parskip}{1ex plus 0.5ex minus 0.2ex} 
\usepackage{multicol} %adds columns
\usepackage{gb4e} %for formatting examples, works with leipzig and multicol
\primebars %setting for gb4e, adds bars for X-bar notation, allows switch between bar or %'%
\noautomath
\usepackage{tabto}
\usepackage{amssymb}
\usepackage{fancyhdr}
\usepackage{setspace}
\usepackage{pifont} %allows dingbats to be called (for the "crosses" and "ticks" defined below)
\usepackage{tipa} % IK


\usepackage{leipzig}%primarily used for the abbreviations

\usepackage[backend=biber,
            style=unified,
            natbib,
            maxcitenames=3,
            maxbibnames=99]{biblatex}
\addbibresource{references.bib}
\usepackage{attrib}%allows authors next to quote environments

\makeatletter
\def\@maketitle{%I guessed from the commenting out of the author below that you don't want an author, this just gets rid of the space associated with the author field
  \newpage
  \null
%  \vskip 2em%
  \begin{center}%
  \let \footnote \thanks
    {\LARGE {\@title}\par}
%    \vskip 1.5em%
%    {\large
%      \lineskip .5em%
%      \begin{tabular}[t]{c}%
%        \@author
%      \end{tabular}\par}%
    \vskip 1em%
    {\large \@date}%
  \end{center}%
  \par
%  \vskip 1.5em
}
\makeatother

\title{LEL2A: Syntax}
%\author{Instructor: Itamar Kastner}
\date{Semester 1, 2024-25}%changed to current academic year

\newcommand*{\sqb}[1]{\lbrack{#1}\rbrack}
\newcommand*{\fn}[1]{\footnote{#1}}
\newcommand{\keyword}[1]{\textsc{#1}}
\newcommand{\cmark}{\ding{51}}
\newcommand{\xmark}{\ding{55}}
\newcommand{\subtitle}[1]{\maketitle\begin{center}{\Large #1}\end{center}}
\makeatletter
\newcommand*{\addFileDependency}[1]{% argument=file name and extension
\typeout{(#1)}% latexmk will find this if $recorder=0
% however, in that case, it will ignore #1 if it is a .aux or 
% .pdf file etc and it exists! If it doesn't exist, it will appear 
% in the list of dependents regardless)
%
% Write the following if you want it to appear in \listfiles 
% --- although not really necessary and latexmk doesn't use this
%
\@addtofilelist{#1}
%
% latexmk will find this message if #1 doesn't exist (yet)
\IfFileExists{#1}{}{\typeout{No file #1.}}
}\makeatother

\newcommand*{\myexternaldocument}[1]{%
\externaldocument{#1}%
\addFileDependency{#1.tex}%
\addFileDependency{#1.aux}%
}
\myexternaldocument{Skills} %also necessary for cross referencing, to reference other documents duplicate with name of document

\begin{document}
\maketitle
\subtitle{Topic 8 Course Notes: Nonverbal XP's Part 2\\
AdjP, PP, \& Headedness}
\hfill{}\textbf{Skills:}~\ref{adjp_pp},
\ref{xp_structure}

\section{Adjective Phrases}
\hfill{}\textbf{Skills:}~\ref{adjp_pp},
\ref{xp_structure}

We have seen the parallels between clauses and noun phrases that have led to the postulation of the general X$'$ schema for phrase structure. Now we'll briefly consider how phrases built from the two remaining main lexical categories, \keyword{adjectives} and \keyword{prepositions}, fit into this general scheme.

Looking at AdjPs, let us first try to answer the question whether an \obar{Adj} can have a complement. That does seem to be the case: some adjectives are optionally transitive:
\begin{exe}
    \ex[]{
    \begin{xlist}
        \ex[]{They are proud (of their daughter).}
        \ex[]{They are keen (on pineapple).}
    \end{xlist}
    }
    \label{adj_of_on}
\end{exe}
And \emph{fond} is even obligatorily transitive:
\begin{exe}
    \ex[]{They are fond *(of their daughter).}
    \label{fond}
\end{exe}

For speakers who allow \emph{so} substitution for adjectival phrases (I personally don't really have this as part of my grammar), it can be shown that the \emph{of-} and \emph{on-}phrases in (\ref{adj_of_on}) are complements, rather than modifiers, on the basis of the constituent replacement test.
(Note that the \emph{of}-phrase in (\ref{fond}) must be regarded as a complement anyway if it is true that modifiers are never obligatory; see earlier notes, and Chapter 2 of S\&K).
For such speakers, the Adj$'$ constituent inside an AdjP can be replaced by the word \emph{so}.
An adjective and its complement form an Adj$'$ constituent that excludes modifiers of Adj.
Hence, it should be possible to replace the constituent containing an adjective and its complement by \emph{so}, while leaving a modifier unaffected by this replacement.
It should not be possible to replace the adjective while leaving the complement unaffected by the replacement (if indeed \emph{so} can only replace an Adj$'$ constituent, not an Adj).
The sentences and judgements in (\ref{proud_sub})--(\ref{keen_sub}) show, then, that \emph{of their daughter} and \emph{on pineapple} are complements:
\begin{exe}
    \ex[]{
    \begin{xlist}
        \ex[]{She was proud of her daughter and he was so, too.}
        \ex[*]{She was proud of her daughter and he was so of his son.}
    \end{xlist}
    }
    \label{proud_sub}
    \ex[]{
    \begin{xlist}
        \ex[]{They are keen on pineapple and we are so, too.}
        \ex[*]{They are keen on pineapple, and we are so on pomegranates.}
    \end{xlist}
    }
    \label{keen_sub}
\end{exe}
\newpage
\begin{exe}
    \ex{
    \begin{multicols}{4}
    \begin{xlist}
            \ex{\small
            \begin{forest}
                [
                \iibar{Adj}
                [\ibar{Adj}
                [\obar{Adj}\\proud][\iibar{P}/\iibar{C}]]
                ]
            \end{forest}
            }%\columnbreak
            \ex{\small
            \begin{forest}
                [
                \iibar{Adj}
                [\ibar{Adj}
                [\obar{Adj}\\fond][\iibar{P}]]
                ]
            \end{forest}
            }%\columnbreak
            \ex{\small
            \begin{forest}
                [
                \iibar{Adj}
                [\ibar{Adj}
                [\obar{Adj}\\keen][\iibar{P}/\iibar{C}]]
                ]
            \end{forest}
            }%\columnbreak
            \ex{\small
            \begin{forest}
                [
                \iibar{Adj}
                [\ibar{Adj}
                [\obar{Adj}\\happy][\iibar{P}]]
                ]
            \end{forest}
            }
    \end{xlist}
    \end{multicols}
    }
\end{exe}

So, adjectives can have complements.
Can they have specifiers as well?
We can observe that, within an AdjP, the adjective can indeed be preceded by specific elements, especially by degree words such as \emph{very}, \emph{rather}, \emph{extremely}, or \emph{too}:
\begin{exe}
    \ex[]{
    \begin{xlist}
        \ex[]{That shirt is [\textsubscript{AdjP} too red to go in the washing machine with the white shirt]}
        \ex[]{In winter the city is [\textsubscript{AdjP} very cold]}
        \ex[]{Sam is [\textsubscript{AdjP} rather fond of dogs]}
    \end{xlist}
    }
\end{exe}

The problem with treating these as specifiers is similar to the problem with treating determiners as specifiers of NPs, discussed above.
Namely, most of these degree expressions are not complete phrases themselves, but rather single words.
Therefore, parallel to the DP hypothesis, it has been proposed that these degree words are heads of a phrase in their own right.
In that case, AdjPs that include such degree words are really DegPs (Degree Phrases) in which the Deg head takes an AdjP as its complement.
But we won't pursue this question any further here -- we've mostly established that adjectives can have complements within their AdjP. 

\section{Preposition/postposition phrases}
\hfill{}\textbf{Skills:}~\ref{adjp_pp},
\ref{xp_structure}

Next we consider PPs, the phrases built from prepositions or postpositions.
Can we discern complements and specifiers in the structure of PPs?
It certainly seems plausible to say that a P can take a complement.
Many prepositions are in fact obligatorily transitive (that is, they obligatorily take a complement):
\begin{exe}
    \ex[]{
    \begin{xlist}
        \ex[]{This will last [\textsubscript{PP} until *(Doomsday)]}
        \ex[]{a piece [\textsubscript{PP} of *(cake)]}
        \ex[]{We’ll go on [\textsubscript{PP} from *(this page)]}
    \end{xlist}
    }
\end{exe}
\begin{exe}
    \ex[]{
    \begin{multicols}{3}
        \begin{xlist}
            \ex[]{
            \begin{forest}
                [
                \iibar{P}
                [\ibar{P}
                [\obar{P}\\until][\iibar{D}]]
                ]
            \end{forest}
            }
            \ex[]{
            \begin{forest}
                [
                \iibar{P}
                [\ibar{P}
                [\obar{P}\\of][\iibar{D}]]
                ]
            \end{forest}
            }
            \ex[]{
            \begin{forest}
                [
                \iibar{P}
                [\ibar{P}
                [\obar{P}\\from][\iibar{D}]]
                ]
            \end{forest}
            }
        \end{xlist}
    \end{multicols}
    }
\end{exe}

As with the other lexical categories, some members of the category do not or do not necessarily take a complement, as discussed in S\&K.
Examples of optionally intransitive prepositions include:
\begin{exe}
    \ex[]{
    \begin{xlist}
        \ex[]{I’ve never met him [\textsubscript{PP} before (today)]}
        \ex[]{The paint came [\textsubscript{PP} off (the wall)]}
    \end{xlist}
    }
\end{exe}
\newpage
\begin{exe}
    \ex[]{
    \begin{multicols}{2}
        \begin{xlist}
            \ex[]{
            \begin{forest}
                [
                \iibar{P}
                [\ibar{P}
                [\obar{P}\\of][(\iibar{D})]]
                ]
            \end{forest}
            }
            \ex[]{
            \begin{forest}
                [
                \iibar{V}
                [\ibar{V}
                [\obar{V}\\eat][(\iibar{D})]]
                ]
            \end{forest}
            }
        \end{xlist}
    \end{multicols}
    }
\end{exe}

The complement of a P need not always be a DP, it can sometimes be a \keyword{clause} or even another PP:\footnote{Where the complement of the preposition is a clause, we still have to decide: is it a CP, or just an IP?  For a discussion of this question, see the textbook.}
\begin{exe}
    \ex[]{
    \begin{xlist}
        \ex[]{\lbrack{}\textsubscript{PP} after \lbrack{}\textsubscript{Clause} they went to America\rbrack{}\rbrack{} they started a new business.}
        \ex[]{The play will last \lbrack{}\textsubscript{PP} until \lbrack{}\textsubscript{PP} after midnight\rbrack{}\rbrack{}}
    \end{xlist}
    }
\end{exe}
As for \iibar{P}s, it is not so obvious what might occupy the specifier position of \iibar{P}s.
For now we will leave this question open.

\section*{Bonus material: Headedness}
\hfill{}\textbf{(No Skills)}

The examples above illustrate that in a typical English phrase, the head first combines with something to its right, the complement, after which the head--complement combination combines with something to its left, the specifier.
This results in general specifier--head--complement order in English phrases.

This order is not universal.
There is language variation, in particular with respect to the order between head and complement: in some languages the head follows rather than precedes the complement.
Languages with complement--head order are often called \keyword{head-final}, and languages with head--complement order \keyword{head-medial} (also often called `head-initial').\footnote{Part of the difficulty with terminology here (and something to watch out for) is that, as indicated here, the terms ``head-inital''/``head-final'' are typically used to refer  to the order of the head with respect to its \emph{complement}.
But that means, for example, that a VP might be ``head-final'' in this sense while still allowing for adjuncts to follow the verb; similarly, a VP could be described as head-initial if the object follows the verb, even if the specifier precedes it.}
For some reason, there seems to be less language variation with respect to the position of specifiers, which tend to be phrase-initial.
It is also important to note that while languages \emph{tend} to be either head-initial or head-final in all their phrases, but this is not necessary; it is possible for a language to have some phrases that are head-initial and some that are head-final.
It even happens that a head of the same lexical category sometimes occurs before and sometimes after its complement. Such behaviour is shown, for instance, by the category P in Dutch (note that (\ref{dutch_p_headedness_A}) and (\ref{dutch_p_headedness_B}) have different meanings):
\begin{exe}
   \ex[]{
   \begin{xlist}
       \ex[]{\gll Ze zwommen \lbrack{}\textsubscript{PP} in het kanaal\rbrack{}\\
       they swam in the canal\\
       \trans `They were swimming in the canal.'}
       \label{dutch_p_headedness_A}
       \ex[]{\gll Ze zwommen \lbrack{}\textsubscript{PP} het kanaal in\rbrack{}.\\
       they swam the canal in\\
       \trans `They swam into the canal'}
       \label{dutch_p_headedness_B}
    \end{xlist}
    }
\end{exe}
Interestingly, the headedness of (phrases in) a language can change during its history.
For example, Old and Middle English showed both complement--verb and verb--complement orders, rather than the consistent verb--complement order we see today.
English is therefore said to have undergone a change `from OV to VO' (where O stands for Object). 

%\printbibliography
\end{document}