\documentclass{article}
\usepackage{xr-hyper} %Adds referencing between handouts and the Skills.tex document to avoid typos (req. latexmkrc)
\externaldocument{Skills} %where to look for labels
\usepackage[hidelinks]{hyperref} %links and URLS
\usepackage[linguistics]{forest} %needs tikz, draws trees
\usepackage[margin=1in]{geometry} %page layout
\usepackage{graphicx} % Required for inserting images
\usepackage[T1]{fontenc} %Make sure to be able to get accented characters etc
\usepackage[utf8]{inputenc}
\usepackage[normalem]{ulem} %adds strikethrough and other commands
\setlength{\parindent}{0pt}%don't indent paragraphs...
\setlength{\parskip}{1ex plus 0.5ex minus 0.2ex} 
\usepackage{multicol} %adds columns
\usepackage{gb4e} %for formatting examples, works with leipzig and multicol
\primebars %setting for gb4e, adds bars for X-bar notation, allows switch between bar or %'%
\noautomath
\usepackage{tabto}
\usepackage{amssymb}
\usepackage{fancyhdr}
\usepackage{setspace}
\usepackage{pifont} %allows dingbats to be called (for the "crosses" and "ticks" defined below)
\usepackage{tipa} % IK


\usepackage{leipzig}%primarily used for the abbreviations

\usepackage[backend=biber,
            style=unified,
            natbib,
            maxcitenames=3,
            maxbibnames=99]{biblatex}
\addbibresource{references.bib}
\usepackage{attrib}%allows authors next to quote environments

\makeatletter
\def\@maketitle{%I guessed from the commenting out of the author below that you don't want an author, this just gets rid of the space associated with the author field
  \newpage
  \null
%  \vskip 2em%
  \begin{center}%
  \let \footnote \thanks
    {\LARGE {\@title}\par}
%    \vskip 1.5em%
%    {\large
%      \lineskip .5em%
%      \begin{tabular}[t]{c}%
%        \@author
%      \end{tabular}\par}%
    \vskip 1em%
    {\large \@date}%
  \end{center}%
  \par
%  \vskip 1.5em
}
\makeatother

\title{LEL2A: Syntax}
%\author{Instructor: Itamar Kastner}
\date{Semester 1, 2024-25}%changed to current academic year

\newcommand*{\sqb}[1]{\lbrack{#1}\rbrack}
\newcommand*{\fn}[1]{\footnote{#1}}
\newcommand{\keyword}[1]{\textsc{#1}}
\newcommand{\cmark}{\ding{51}}
\newcommand{\xmark}{\ding{55}}
\newcommand{\subtitle}[1]{\maketitle\begin{center}{\Large #1}\end{center}}
\makeatletter
\newcommand*{\addFileDependency}[1]{% argument=file name and extension
\typeout{(#1)}% latexmk will find this if $recorder=0
% however, in that case, it will ignore #1 if it is a .aux or 
% .pdf file etc and it exists! If it doesn't exist, it will appear 
% in the list of dependents regardless)
%
% Write the following if you want it to appear in \listfiles 
% --- although not really necessary and latexmk doesn't use this
%
\@addtofilelist{#1}
%
% latexmk will find this message if #1 doesn't exist (yet)
\IfFileExists{#1}{}{\typeout{No file #1.}}
}\makeatother

\newcommand*{\myexternaldocument}[1]{%
\externaldocument{#1}%
\addFileDependency{#1.tex}%
\addFileDependency{#1.aux}%
}
\myexternaldocument{Skills} %also necessary for cross referencing, to reference other documents duplicate with name of document

\begin{document}
\maketitle
\subtitle{Topic 11 Course Notes: Passives}
\hfill{}\textbf{Skills:}~\ref{passives}

In our last few topics we saw that in a clause containing a verb like \emph{seem} and a non-finite complement clause to this verb, the subject of the non-finite complement clause moves to the subject position of the clause containing seem, a process we called \keyword{raising}; another name for this sort of movement is \keyword{A-movement}. We then saw that this kind of movement to subject position (Specifier of IP) happens in a number of constructions. In this topic we will discuss one more construction, called the \keyword{passive}, in which a constituent moves to the subject position of the clause. For further reading and detail, there will also be sections dealing with \keyword{locality} restrictions on movement to subject position, in that the DP moving to subject position must be close enough to it in a sense to be made specific, and on additional investigations of the subject requirement.

\section{Properties of the passive}
\hfill{}\textbf{Skill:}~\ref{passives}

Many languages (though not all) have a construction known as the \keyword{passive}. In English, the alternation between an active clause and its passive counterpart looks like in the following pairs (where parentheses show optionality):
\begin{exe}
\begin{multicols}{2}
    \ex{
    \begin{xlist}
        \ex{Pavel invited Itamar to the party.}
        \ex{Itamar was invited to the party (by Pavel).}
    \end{xlist}
    }\label{passive_examples_1}
    \ex{
    \begin{xlist}
        \ex{Joan is feeding the elephants.}
        \ex{The elephants are being fed (by Joan).}
    \end{xlist}
    }\label{passive_examples_2}
\end{multicols}
\end{exe}

We can identify a number of systematic similarities differences between the active and passive clauses. Let's list them here, and then go through them one at a time:
\begin{enumerate}
    \item The active and passive clauses share selectional restrictions. This also means that the Theme of the active clause remains the Theme of the passive clause.
    \item The object of the active clause becomes the subject of the passive clause.
    \item The Agent can be left out altogether, or reintroduced in a \emph{by}-phrase.
    \item The verb and auxiliaries (if any) receive special morphological marking.
\end{enumerate}

    \subsection{Selectional restrictions}
We have seen that a transitive verb combines with two syntactic arguments, a subject that frequently realizes an Agent role and an object that frequently realizes a Theme role.
Moreover, we have seen that (at least in English) there is a particular order in which the verb combines with these two arguments.
First the verb combines with the object within the VP, then the subject combines with the VP as a whole.

We saw previously that a verb can impose selectional restrictions on its arguments.
A verb like \emph{invite}, for example, wants its direct object to refer to a human, while a verb like \emph{feed} only goes together felicitously with objects that refer to things that can take food.
It turns out that the selectional restrictions on the object in an active clause are exactly the same as the selectional restrictions on the subject in the passive counterpart of that clause.
This is illustrated in (\ref{passive_selection_1}--\ref{passive_selection_2}):
\begin{exe}
    \ex{
    \begin{xlist}
        \ex[]{Itamar invited his parents / \#only public transport / \#three bottles of whisky.}
        \ex[]{His parents / \#Only public transport / \#Three bottles of whisky were invited, too.}
    \end{xlist}
    }\label{passive_selection_1}
    \ex{
    \begin{xlist}
        \ex[]{Joan feeds  her cat /\#her bike / \#her shoe.}
        \ex[]{Her cat / \#her bike / \#her shoe  is regularly fed by Joan.}
    \end{xlist}
    }\label{passive_selection_2}
\end{exe}

This means that we're probably dealing with the same verb, and potentially the same elementary tree, in both the active and passive versions. The Theme of the active clause remains the Theme even in the passive clause.

    \subsection{Object-to-subject raising}
Back in Topic 3a we discussed cases in which syntactic functions (like subject and object) don't align with thematic roles (like Agent and Theme). The passive examples show us that the subject is a Theme, just like the original Theme object of the active sentence. We can tell that it's the subject with our usual tool of agreement:
\ea
    \ea The child was invited to the party.
    \ex The children were invited to the party.
    \z
\z

This is starting to look a bit like an instance of raising, where we had an object Theme which remains the Theme semantically but is now the subject. Why this happens is something we'll need to return to after we consider the other properties of the passive.

    \subsection{The Agent and \emph{by}-phrases}
While the Theme is preserved in the passive, the status of the Agent is less obvious. We can leave it out completely:
\ea Her parents were invited.
\z

But we can also re-introduce it in a PP headed by the preposition \emph{by}, often simply called a \keyword{\emph{by}-phrase}:
\ea Her parents were invited \textbf{by the First Minister}.
\z

There are two things worth keeping in mind in our analysis. The first is that because the Agent is now optional, we don't want to include it in our elementary tree for the verb - we don't have a way in our theory to chuck out phrases that we don't want anymore during the derivation!

The second is that the \emph{by}-phrase is limited to Agents, i.e.~animate beings that can do something more or less on purpose. Other Causers are incompatible with the \emph{by}-phrase and require some other kind of wording. And other Causers are also incompatible with some verbs, too.

Some verbs require an Agent subject, others don't, but \emph{by}-phrases do:
\ea
    \ea[]{The First Minister invited her parents.}
    \ex[*]{The wind invited her parents.}
    \z
\ex
    \ea[]{The first minister knocked over the glass.}
    \ex[]{The wind knocked over the glass.}
    \z
\ex
    \ea[]{Her parents were invited by the First Minister.}
    \ex[*]{Her parents were invited by the wind.}
    \z
\z

    \subsection{Passive morphology}
In English, the passive is formed by using a combination of the \textbf{past participle} of the main verb and the auxiliary \uline{\emph{be}}, plus any other auxiliaries which might also be involved:
\ea
    \ea Her parents \uline{were} \textbf{invited}.
    \ex Her parents will have \uline{been} \textbf{invited}.
    \z
\z

While we would want to capture the systematic aspects of the morphology in our theory, that's not something we'll be able to develop in LEL2A.

\section{Formal analysis}
\hfill{}\textbf{Skill:}~\ref{passives}

Let's recap what we've established so far.
\ea Properties of the passive:
    \ea The active and passive clauses share selectional restrictions.
    \ex The object of the active clause becomes the subject of the passive clause.
    \ex The Agent can be left out altogether, or reintroduced in a \emph{by}-phrase.
    \ex The verb and auxiliaries (if any) receive special morphological marking.
    \z
\z

Our solution would be to treat passives as essentially another instance of raising to subject (or ``A-movement''). The Agent will not be introduced in the Specifier of NP, but it can be added as a modifier \emph{by}-phrase PP (adjunction at the V-bar level).

Let's put this into practice. Assume that the constituent in the subject position of the passive clause actually is the object argument of the verb.
What happens in a passive, then, is that this object argument is moved to the subject position in the clause:
\begin{exe}
    \ex{
    \begin{forest}
        [
        \iibar{I}
        [\iibar{D} [her parents, roof, name=copy]][\ibar{I}
        [\obar{I}\\were][\iibar{V} [\phantom{X} ]
        [\ibar{V}
        [\obar{V}\\invited][\iibar{D}\\t, name=trace]]]]
        ]
        \draw[->,dotted] (trace) to[out=south west,in=south] (copy);
    \end{forest}
    }\label{passive-tree}
\end{exe}
\vspace{-2em}
As with all traces left by moved constituents, the trace in (\ref{passive-tree}) shares all properties with its antecedent (\emph{her parents} in this case).
The trace can therefore function as the object of the verb \emph{invited} and satisfy the selectional restrictions of that verb.
Note that no other element can occupy the object position, which can be taken as an indication that this position is indeed filled:
\begin{exe}
    \ex[*]{Her parents were invited her cousins.}\label{her_parents}
\end{exe}

The DP \emph{her parents} in (\ref{her_parents}) now functions as the grammatical subject of the clause (which lets it trigger the correct agreement, as we saw above).
% This is shown, amongst other things, by the fact that it shows agreement with the finite verb:
% \begin{exe}
%     \ex{
%     \begin{xlist}
%         \ex[]{Her father was/*were invited.}
%         \ex[]{Her parents *was/were invited.}
%     \end{xlist}
%     }
% \end{exe}

When the Agent does appear, it's as an adjunct:
\begin{exe}
    \ex{
    \begin{forest}
    [\iibar{I}
        [\iibar{D} [her parents, roof, name=copy]]
        [\ibar{I}
            [\obar{I}\\were]
            [\iibar{V}
                [{\phantom{X}} ]
                [\ibar{V}
                    [\ibar{V}
                        [\obar{V}\\invited][\iibar{D}\\t, name=trace]
                    ]
                    [\iibar{P}
                        [\phantom{X} ]
                        [\ibar{P}
                            [P\\by]
                            [DP [the First Minister, roof]]
                        ]
                    ]
                ]
            ]
        ]            
    ]
        \draw[->,dotted] (trace) to[out=south west,in=south west] (copy);
    \end{forest}
    }\label{passive-tree}
\end{exe}


\section{Extension: more on A-movement}
The type of grammatical subject we see in passives, and in sentences with a raising verb (as discussed in the last set of notes), which starts life as an object or as the subject of a lower clause and is then `promoted' to a higher subject position, is called a \keyword{derived} subject.

The type of object-to-subject movement we see in passives must be distinguished from other types of movement, such as the movement process that puts questioned phrases in the first position of a clause in English, called \keyword{wh-movement}.
In contrast to movement that makes a phrase the grammatical subject of a clause, Wh-movement does not change the grammatical function of the moved phrase: just like \emph{her parents} in (\ref{marya}), \emph{whose parents} is the direct object of \emph{invite} in (\ref{maryb}), not the subject.
It is just in a different position than the usual one for objects, namely in the first position in the clause, in front of the subject position which is occupied by \emph{Mary} in both (\ref{marya}) and (\ref{maryb}).
(You can see that \emph{Mary} is still the subject in (\ref{maryb}), not \emph{whose parents}, because it is the former rather than the latter that determines the agreement on the finite verb.)
\begin{exe}
    \ex{
    \begin{xlist}
        \ex[]{Mary has invited her parents.}\label{marya}
        \ex[]{Whose parents$_{i}$ has Mary invited \emph{t}$_i$?}\label{maryb}
    \end{xlist}
    }
\end{exe}
The type of movement we see in passives and in raising constructions, which moves a constituent to the subject position of the clause, is called \keyword{A-movement}.
This term is used because the type of position to which this process moves a constituent is called an \keyword{A-position}, for historical reasons that need not concern us here.
In contrast, the type of movement that does not change the grammatical function of the moving constituent, such as \keyword{Wh-movement}, is called \keyword{A$’$-movement} (pronounced `A-bar movement'), and the type of position that this movement targets is called an \keyword{A$'$-position} (pronounced `A-bar position').
We will see in the next topic which position in the clause is the landing site for Wh-movement.

In the examples of passive above, such as (\ref{passive-tree}), there was A-movement from direct object position to subject position.
In some contexts, another constituent than the direct object is `promoted' to subject by this movement.
This happens, for instance, in the passive of a \keyword{double object} construction.
The double object construction is a sentence type containing a ditransitive verb, in which both the direct object and the indirect object are expressed by Noun Phrases, as in (\ref{igave}) for example.\footnote{Again this raises the question of the structure for a double object construction is, particularly since we have been assuming that all phrase structure trees are only binary branching.
See the item for the tutorial exercises for Topic 2 in Learn for some notes about ditransitives and further reading about this, if you are interested. }
In many varieties of English, it is the indirect object that is chosen to be promoted to the subject position in the passive counterpart of the double object construction, rather than the direct object, see (\ref{marywas}--\ref{thosebooks}):
\begin{exe}
    \ex[]{
    \begin{xlist}
        \ex[]{I gave Mary those books.}\label{igave}
        \ex[]{Mary$_i$ was given \emph{t}$_i$ those books.}\label{marywas}
        \ex[*]{Those books$_i$ were given Mary \emph{t}$_i$.}\label{thosebooks}
    \end{xlist}
    }
\end{exe}

\section{Extension: Locality}
% \hfill{}\textbf{Skills:}~\ref{passives},
% \ref{locality_constraints}

As it turns out, it is not possible to move a constituent across arbitrarily long distances by A-movement.
In particular, it turns out that:
\begin{exe}
    \ex[]{
    \begin{xlist}
        \ex[]{You cannot move a constituent to a subject position across another subject position.}\label{generalisation_A}
        \ex[]{You cannot move a constituent to a subject position X if there is another constituent that could also move to X and that is closer to X.}\label{generalisation_B}
    \end{xlist}
    }
\end{exe}
The restriction in (\ref{generalisation_A}) is known as the impossibility of \keyword{superraising}.
Its effect is shown by an example like (\ref{locality_D}), where movement of \emph{Sean} to the subject position in the highest clause skips the subject position of the intermediate clause, which is filled by \emph{it}.
This can be compared to the grammatical examples in (\ref{locality_B}) and (\ref{locality_C}), where movement to a higher subject position does not skip another subject (in (\ref{locality_C}) this is because there first is A-movement to the subject position of the passive intermediate clause, and then to the subject position of the highest clause, which contains a raising verb).
\begin{exe}
    \ex[]{
    \begin{xlist}
        \ex[]{It seems [that it was believed [that Sean plays the piano]].}\label{locality_A}
        \ex[]{It seems [that Sean$_i$ was believed [\emph{t}$_i$ to play the piano]].}\label{locality_B}
        \ex[]{Sean$_i$ seems [\emph{t}$_i$ to have been believed [\emph{t}$_i$ to play the piano]].}\label{locality_C}
        \ex[*]{Sean$_i$ seems [that it was believed [\emph{t}$_i$ to play the piano]].}\label{locality_D}
    \end{xlist}
    }\label{locality}
\end{exe}

The restriction in (\ref{generalisation_B}) is known as \keyword{superiority}.
Superiority is discussed most often in connection to A$'$-movement (on which more in a later lecture), but A-movement is subject to it, too.
Consider (\ref{superiority_example}).
There is evidence (that we cannot discuss here) that in a sentence like (\ref{superiority_example_A}) the direct object DP is in a higher position in the structure than the indirect object PP.
It is therefore closer to the subject position.
Superiority then demands that in the passive counterpart to (\ref{superiority_example_A}) it is the direct object argument that moves to subject position, rather than the indirect object argument.
This is correct, as (\ref{superiority_example_B}--\ref{superiority_example_D}) show.
Possibly, (\ref{superiority_example_C}) is ungrammatical for an independent reason, as English does not like PPs as subject very much in any case, but this does not hold for (\ref{superiority_example_D}) (note that the `preposition stranding' that occurs in (\ref{superiority_example_D}) is perfectly fine in other contexts in English, such as  \emph{Paula is never listened to}.
Therefore, that cannot be the source of the ungrammaticality of this example). 
\begin{exe}
    \ex{
    \begin{xlist}
        \ex[]{Mary showed some books to Paula.}\label{superiority_example_A}
        \ex[]{\lbrack{}Some books\rbrack{}$_i$ were shown \emph{t}$_i$ to Paula by Mary.}\label{superiority_example_B}
        \ex[*]{\lbrack{}To Paula\rbrack{}$_i$ was shown some books \emph{t}$_i$ by Mary.}\label{superiority_example_C}
        \ex[*]{\lbrack{}Paula\rbrack{}$_i$ was shown some books to \emph{t}$_i$ by Mary.}\label{superiority_example_D}
    \end{xlist}
    }\label{superiority_example}
\end{exe}

\section{Extension: Burzio’s generalization}
% \hfill{}\textbf{Skill:}~\ref{passives}

We have now seen a number of instances of A-movement in which a constituent is moved to the subject position of a finite clause, which we have assumed to be the spec-IP position.
So far, however, we have been silent on the question why A-movement takes place.
One area where people have a looked for an answer to this is \keyword{Case theory}. 

In languages with visible case morphology, we see that the different arguments of a verb can carry different case endings.
Typically, subjects show up in what is called a \keyword{nominative} case form, direct objects in an \keyword{accusative} case form, and indirect objects in a \keyword{dative} case form.\footnote{Notice, however, that in the course of its history English has lost the distinction between accusative and dative case, even on pronouns, which appear in the accusative case even when they are functioning as indirect objects.}
Apparently, if some DP receives a thematic role of the verb, it carries a particular case.
Case is therefore sometimes said to make the thematic role of an DP ‘visible’ for the interpretative component of the grammar (the semantic component):
\begin{exe}
    \ex[]{
    \begin{xlist}
        \ex[]{\keyword{Visibility condition}\\
		A DP that receives a thematic role from a verb must be assigned case.}
    \end{xlist}
    }\label{visibility_condition}
\end{exe}
There is no reason to assume that in languages without overt case morphology (such as---with the exception of the pronominal system---modern English) the requirement in (\ref{visibility_condition}) is just void.
The difference with languages showing morphological case is that the case-marking must involve abstract, nonvisible, case.
To indicate the difference with morphological case, this abstract case usually gets the distinction of being written with a capital C, i.e.\ as `Case'.

There is a tendency that languages without morphological case have a somewhat stricter word order than languages that have it (though it should be emphasised that this is, indeed, a tendency).
This may indicate that there are conditions on the assignment of abstract Case, such that a particular Case can only be assigned in a particular structural configuration.
The following restrictions on structural Case assignment have been proposed:
\begin{exe}
    \ex[]{
    \begin{xlist}
        \ex[]{Verbs and prepositions can assign abstract structural Case, but nouns and adjectives cannot.}\label{generalisations_2A}
        \ex[]{Verbs and prepositions can assign structural Accusative to a DP that they govern (where, roughly, V or P governs a DP if the DP is lower in the structure than V or P and there is no other Case-assigner intervening between the V or P and the DP)}\label{generalisations_2B}
        \ex[]{The inflection I on finite verbs can assign structural Nominative to the element in the associated spec-IP position.}\label{generalisations_2C}
        \end{xlist}
    }
\end{exe}
Returning now to the question of why there is A-movement, let us first consider what passivization does to a verb.
How does making a passive participle out of a verb affect the properties of this verb?
For one, this process must involve suppression of the assignment of the regular external theta-role of the verb, as the subject that gets this role in the active clause does not appear as an argument in the passive (but at most inside an optional PP adjunct).
Suppose now that another effect is that the Case-assigning capabilities of the verb are lost under passive participle formation.
In that case, if the internal argument would remain in the complement position to the verb, it would remain Caseless---which would violate the Visibility condition in (\ref{visibility_condition}).
Given that passive clauses do contain a verb that carries finite inflection (namely the finite auxiliary verb), Nominative Case is available in the spec-IP position.
Therefore, the object moves to spec-IP to receive this Case and thereby satisfy (\ref{visibility_condition}).

If so, we know what a defining characteristic of raising verbs must be.
Despite being active verbs, they, too, must have deficient Case-properties: they are not able to assign Accusative to the subject of their non-finite complement clause.
Nor can the subject of a non-finite clause get Nominative, because finite inflection is needed to assign that.
Like the object in a passive, then, the subject in the complement to a verb like \emph{seem} raises to the spec-IP position of the finite main clause in order to get Case (namely Nominative) there.

We see that, if this approach to the rationale of A-movement is correct, it must be the case in general that verbs that do not assign a thematic role to their subject position (such as passive participles and raising verbs) lack the capacity to assign Accusative Case to a constituent in (or inside) their complement position.
This correlation is known as \keyword{Burzio’s Generalization}, after the Italian syntactician Luigi Burzio.
\begin{exe}
    \ex[]{\keyword{Burzio’s Generalization}\\
		If a verb does not assign a thematic role to its subject position, it does not assign Accusative Case to its complement position, and vice versa.}\label{burzios_gen}
\end{exe}
It should be noted that the account outlined above is not wholly unproblematic, however.
People, including Burzio himself, have pointed out that (\ref{burzios_gen}) is quite stipulative, as it is not so clear why a verb’s capacity to assign abstract Case to its complement position should be related to its assigning a thematic role to its subject position or not.
Also, consider what happens with other elements that cannot assign abstract Case to their complements, namely adjectives and nouns (see (\ref{generalisations_2A})).
That these don't assign abstract Case to their complement is shown by the ungrammaticality of examples like those in (\ref{n_a_dont_assign_case}), where an N and an A, respectively, take a DP complement.
(Apparently, there is no abstract Case counterpart to the genitive case we typically see showing up on complements of N and A in languages that have morphological case.)
\begin{exe}
    \ex[]{
    \begin{xlist}
        \ex[*]{the destruction the city}
        \ex[*]{afraid dogs}
    \end{xlist}
    }\label{n_a_dont_assign_case}
\end{exe}
As and Ns also do not carry finite inflection, so any potential subject position in the NP or AP will not receive Nominative either.
Does this mean that As and Ns cannot select for a DP argument in languages without morphological case at all? Clearly not.
But what happens is that a meaningless preposition \emph{of} is plugged into the structure, as in (\ref{p_saves_the_day}).
The only function of this P seems to be to assign abstract Case to the DP in the complement.
\begin{exe}
    \ex[]{
    \begin{xlist}
        \ex[]{the destruction of the city}
        \ex[]{afraid of dogs}
    \end{xlist}
    }\label{p_saves_the_day}
\end{exe}
However, if this preposition is available to save DPs in the complement of an N or A from being without Case, and if lack of Case is the crucial property of complements to passive and raising verbs, we may wonder why it is not possible to use this preposition in passives and with raising verbs as well (rather than having the complement undergo A-movement).
That this is indeed impossible is shown by the examples in (\ref{p_passives}):
\begin{exe}
    \ex[]{
    \begin{xlist}
        \ex[*]{It has been invited of Mary.}
        \ex[*]{It seems of John to have left.}
    \end{xlist}
    }\label{p_passives}
\end{exe}
Instead of appealing to Case, there are other accounts (including one by Burzio) which make use of the fact that, quite independently of A-movement, we already need a principle in English that states that every clause must have a subject, namely the Subject requirement we have encountered before.
Perhaps that requirement is what triggers A-movement to subject position, too, but we will not pursue this here any further.

\section*{Further reading}

If you are interested in finding out more about the passive, a good place to start is the classic article \citet{jaeggli_passive_1986}.

\printbibliography
\end{document}