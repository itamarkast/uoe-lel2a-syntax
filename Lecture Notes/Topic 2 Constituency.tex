\documentclass{article}
\usepackage{xr-hyper} %Adds referencing between handouts and the Skills.tex document to avoid typos (req. latexmkrc)
\externaldocument{Skills} %where to look for labels
\usepackage[hidelinks]{hyperref} %links and URLS
\usepackage[linguistics]{forest} %needs tikz, draws trees
\usepackage[margin=1in]{geometry} %page layout
\usepackage{graphicx} % Required for inserting images
\usepackage[T1]{fontenc} %Make sure to be able to get accented characters etc
\usepackage[utf8]{inputenc}
\usepackage[normalem]{ulem} %adds strikethrough and other commands
\setlength{\parindent}{0pt}%don't indent paragraphs...
\setlength{\parskip}{1ex plus 0.5ex minus 0.2ex} 
\usepackage{multicol} %adds columns
\usepackage{gb4e} %for formatting examples, works with leipzig and multicol
\primebars %setting for gb4e, adds bars for X-bar notation, allows switch between bar or %'%
\noautomath
\usepackage{tabto}
\usepackage{amssymb}
\usepackage{fancyhdr}
\usepackage{setspace}
\usepackage{pifont} %allows dingbats to be called (for the "crosses" and "ticks" defined below)
\usepackage{tipa} % IK


\usepackage{leipzig}%primarily used for the abbreviations

\usepackage[backend=biber,
            style=unified,
            natbib,
            maxcitenames=3,
            maxbibnames=99]{biblatex}
\addbibresource{references.bib}
\usepackage{attrib}%allows authors next to quote environments

\makeatletter
\def\@maketitle{%I guessed from the commenting out of the author below that you don't want an author, this just gets rid of the space associated with the author field
  \newpage
  \null
%  \vskip 2em%
  \begin{center}%
  \let \footnote \thanks
    {\LARGE {\@title}\par}
%    \vskip 1.5em%
%    {\large
%      \lineskip .5em%
%      \begin{tabular}[t]{c}%
%        \@author
%      \end{tabular}\par}%
    \vskip 1em%
    {\large \@date}%
  \end{center}%
  \par
%  \vskip 1.5em
}
\makeatother

\title{LEL2A: Syntax}
%\author{Instructor: Itamar Kastner}
\date{Semester 1, 2024-25}%changed to current academic year

\newcommand*{\sqb}[1]{\lbrack{#1}\rbrack}
\newcommand*{\fn}[1]{\footnote{#1}}
\newcommand{\keyword}[1]{\textsc{#1}}
\newcommand{\cmark}{\ding{51}}
\newcommand{\xmark}{\ding{55}}
\newcommand{\subtitle}[1]{\maketitle\begin{center}{\Large #1}\end{center}}
\makeatletter
\newcommand*{\addFileDependency}[1]{% argument=file name and extension
\typeout{(#1)}% latexmk will find this if $recorder=0
% however, in that case, it will ignore #1 if it is a .aux or 
% .pdf file etc and it exists! If it doesn't exist, it will appear 
% in the list of dependents regardless)
%
% Write the following if you want it to appear in \listfiles 
% --- although not really necessary and latexmk doesn't use this
%
\@addtofilelist{#1}
%
% latexmk will find this message if #1 doesn't exist (yet)
\IfFileExists{#1}{}{\typeout{No file #1.}}
}\makeatother

\newcommand*{\myexternaldocument}[1]{%
\externaldocument{#1}%
\addFileDependency{#1.tex}%
\addFileDependency{#1.aux}%
}
\myexternaldocument{Skills} %also necessary for cross referencing, to reference other documents duplicate with name of document

\begin{document}
\maketitle
\subtitle{Topic 2 Course Notes: Constituent Structure and Constituency Tests}
\hfill{}\textbf{Skills:}~\ref{constituencytestNP},
\ref{constituencytestVP},
\ref{constituencytestother}

\section{Diagnostics for constituency}

We have already seen that sentences have an internal structure: they consist of constituents, which in turn may consist of smaller constituents, and so on. 
These notes go over how we can determine whether a string of words in a sentence is a constituent or not.
That is, we are going to discuss a number of \keyword{diagnostics} or \keyword{tests} for sentence structure, so that we can be more precise about what we mean when we say that some words in the sentence `belong together'.

The two most basic tests to determine whether a sequence of words (also called a \keyword{string}) is a constituent are:

\begin{enumerate}
	\item A constituent can be \emph{replaced} by a single word.
	\item A constituent can, in its entirety, be \emph{placed in a different position} in the sentence.
\end{enumerate}

We will discuss a third test as well, the \emph{cleft}, after we have seen what these two amount to. It's important to remember that constituency tests can confirm, but not disprove: if a string passes a test, it's likely to be a constituent. But there are various reasons why it might not pass a test even if it is a constituent, so we need to be careful of these \textsc{false negatives} (see the discussion in Chapter 2 of the Santorini and Kroch reading).

\section{Substitution}

\subsection{Noun phrases}
\hfill{}\textbf{Skill:}~\ref{constituencytestNP}

A constituent that is built around a \keyword{noun} is called a \keyword{noun phrase}, or \keyword{np} for short.
That the noun is the most important element, called the \keyword{head}, of a noun phrase is suggested by the fact that you generally cannot leave it out:
\begin{exe}
    \ex[]{
    \begin{xlist}
        \ex[]{{[}Grandmothers] wear hats.}
        \ex[]{{[}Scottish grandmothers] wear hats.}
        \ex[]{{[}Grandmothers from Ayrshire] wear hats.}
        \ex[]{{[}Scottish grandmothers from Ayrshire] wear hats.}
        \ex[*]{{[}Scottish] wear hats.}
        \ex[*]{{[}From Ayrshire] wear hats}
        \ex[*]{{[}Scottish from Ayrshire] wear hats.}
    \end{xlist}
    }
\end{exe}

A noun phrase can be recognized as such because it can be replaced by a \keyword{pronoun}.
Thus, we can determine that in (\ref{np_subA}) the string \emph{the boy with a bad haircut who fed the cat} is one constituent (functioning as the direct object of the verb \emph{saw}), because it can be replaced by the pronoun \emph{him}, as in (\ref{np_subB}).
By the same means, we can determine that just the string \emph{the boy} is not a constituent of (\ref{np_subA}), since it cannot be replaced in this way: see (\ref{np_subC}).
Of course, \emph{the boy} can be a constituent in a different sentence, such as (\ref{np_subD}), as we can infer from the fact that (\ref{np_subE}) is well-formed.
\begin{exe}
    \ex[]{
    \begin{xlist}
        \ex[]{Joan saw [the boy with a bad haircut who fed the cat].}
        \label{np_subA}
        \ex[]{Joan saw [him].}
        \label{np_subB}
        \ex[*]{Joan saw [[him] with a bad haircut] who fed the cat.}
        \label{np_subC}
        \ex[]{Joan saw [the boy] while he fed the cat.}
        \label{np_subD}
        \ex[]{Joan saw [him] while he fed the cat}
        \label{np_subE}
    \end{xlist}
    }
\end{exe}

Similarly, (\ref{recursion_NP_subB}) shows that the entire string \emph{the woman \ldots\ shop} is a constituent (namely the object) of (\ref{recursion_NP_subA}).
That this constituent in turn consists of smaller constituents is shown by (\ref{recursion_NP_subC}), where replacement with pronominal \emph{them} indicates that \emph{the shoes that I had donated to the Oxfam shop} is one of the constituents within the object constituent of (\ref{recursion_NP_subA}) - recursion! 
\begin{exe}
    \ex[]{
    \begin{xlist}
        \ex[]{Yesterday I met [the woman who you told me you noticed on the corner wearing [the shoes that I had donated to the Oxfam shop]].}
        \label{recursion_NP_subA}
        \ex[]{I met [her].}
        \label{recursion_NP_subB}
        \ex[]{I met [the woman who you told me you noticed on the corner wearing [them]].}
        \label{recursion_NP_subC}
    \end{xlist}
    }
\end{exe}

\subsection{Preposition phrases}
\hfill{}\textbf{Skill:}~\ref{constituencytestother}

\keyword{Preposition phrases} or \keyword{pp}s are constituents that have a \keyword{preposition} as their head.
That the preposition is the head is again suggested by the fact that it cannot be left out:
\begin{exe}
    \ex[]{
    \begin{xlist}
        \ex[]{the man [in the room]}
        \ex[*]{the man [the room]}
    \end{xlist}
    }
\end{exe}

An indication that something is a constituent of the PP type is that it can be replaced by certain \keyword{adverbs}, although this only really works for locative PPs with \emph{there} and temporal PPs with \emph{then} (can you figure out why?):
\begin{exe}
    \ex[]{
    \begin{xlist}
        \ex[]{She went \lbrack{}\textsubscript{PP} to the port of Rotterdam\rbrack{}.}
        \ex[]{She went [there].}
        \ex[]{I have never seen that \lbrack{}\textsubscript{PP} in a place like Edinburgh\rbrack{}.}
        \ex[]{I have never seen that [here].}
    \end{xlist}
    }
\end{exe}

Thus, (\ref{pp_subB}--\ref{pp_subC}) show that \emph{in the theatre with the painted ceiling}, but not just \emph{in the theatre}, is a constituent of (\ref{pp_subA}), while (\ref{pp_subE}) shows that in (\ref{pp_subD}) \emph{in the theatre} is a constituent.
We can also determine that \emph{in the theatre with a friend of mine} is not a constituent of (\ref{pp_subE}), because \emph{there} in (\ref{pp_subC}) cannot stand in for this entire string.
\begin{exe}
    \ex[]{
    \begin{xlist}
        \ex[]{He was [in the theatre with the painted ceiling].}
        \label{pp_subA}
        \ex[*]{He was there with the painted ceiling.}
        \label{pp_subB}
        \ex[]{He was [there].}
        \label{pp_subC}
        \ex[]{He was [in the theatre] with a friend of mine.}
        \label{pp_subD}
        \ex[]{He was [there] with a friend of mine.}
        \label{pp_subE}
    \end{xlist}
    }
\end{exe}

\subsection{Adjective phrases}
\hfill{}\textbf{Skill:}~\ref{constituencytestother}

The phrases in (\ref{adjectivesA}--\ref{adjectivesB}) have an \keyword{adjective} as their head; this head cannot be left out, as shown by (\ref{adjectivesC}--\ref{adjectivesD}).
\begin{exe}
    \ex[]{
    \begin{xlist}
        \ex[]{Marian seems \lbrack{}\textsubscript{AP} very ill\rbrack{}.}
        \label{adjectivesA}
        \ex[]{That music is \lbrack{}\textsubscript{AP} too loud to be bearable\rbrack{}.}
        \label{adjectivesB}
        \ex[*]{Marian seems \lbrack{}very\rbrack{}.}
        \label{adjectivesC}
        \ex[*]{That music is \lbrack{}too to be bearable\rbrack{}.}
        \label{adjectivesD}
    \end{xlist}
    }
\end{exe}

An \keyword{adjective phrase}, or \keyword{ap}, can for some speakers be replaced by the word \emph{so} (although this is not felicitous in just any context; adding the word \emph{too} to the sentence often helps):
\begin{exe}
    \ex[]{Marian seems ill and Harriet seems so, too.}
\end{exe}

This test indicates that in (\ref{ap_subA}) \emph{rather weird} is an AP, while just \emph{weird} is not.
\begin{exe}
    \ex[]{
    \begin{xlist}
        \ex[]{Harry is rather weird, and Bill is so, too.}
        \label{ap_subA}
        \ex[*]{Harry is rather weird, and Bill is very so, too.}
        \label{ap_subB}
    \end{xlist}
    }
\end{exe}
Note that the test does not work when the AP is used \keyword{attributively}, that is, as a \keyword{prenominal modifier}:
\begin{exe}
\ex{\begin{xlist}
    \ex[]{a rather weird man}
    \ex[*]{a so woman, too}
\end{xlist}}
\end{exe}

\subsection{Verb phrases}
\hfill{}\textbf{Skill:}~\ref{constituencytestVP}

Finally, there are phrases that have a verb as their head, \keyword{verb phrases} or \keyword{VP}s, such as the bracketed phrases in (\ref{vp}).
\begin{exe}
    \ex[]{
    \begin{xlist}
        \ex[]{To \lbrack{}\textsubscript{VP} work in the garden\rbrack{} is a nice thing to do.}
        \ex[]{Mary will \lbrack{}\textsubscript{VP} rent some films tonight\rbrack{}.}
    \end{xlist}
    }
    \label{vp}
\end{exe}

A VP constituent can be recognized by its being replaceable by the expression \emph{do so}.
This is, admittedly, not a single word, but since arguably neither the part \emph{do} nor the part \emph{so} on its own stands in for a constituent, we are going to assume for now that this is a fixed expression that substitutes as a whole for a VP:
\begin{exe}
    \ex[]{
    \begin{xlist}
        \ex[]{Mary will \lbrack{}\textsubscript{VP} rent some films tonight\rbrack{} and Bill will \emph{do so} as well.}
        \ex[]{Shane has \lbrack{}\textsubscript{VP} given money to that charity\rbrack{} and Jane has \emph{done so}, too.}
        \ex[]{To \lbrack{}\textsubscript{VP} work in the garden\rbrack{} is a nice thing to do, and to \emph{do so} in the morning is especially relaxing.}
    \end{xlist}
    } \label{vp-do-so}
\end{exe}

Notice, for example, that in~(\ref{vp-do-so}c), what's relaxing to do in the morning is not just to \emph{work}, but to \emph{work in the garden}.

    \subsection{Summary of substitutions}
Let's summarise the substitutions we've discussed so far:

\begin{center}
\begin{tabular}{ll}
    Constituent type & Substitution \\\hline
    NP & pronoun \\
    PP & \emph{here/there/then} \\
    AdjP & \emph{so (too)} \\
    VP & \emph{do so} \\
\end{tabular}
\end{center}

In a sense, the substitutions give us the simplest, least informative forms of the constituents being substituted. You can try this with other languages you're familiar with and share what you've found out.

\section{Movement}
\hfill{}\textbf{Skills:}~\ref{constituencytestNP},
\ref{constituencytestVP},
\ref{constituencytestother}

The \keyword{word order} of a sentence can be altered in certain ways such that the result is still a possible sentence.
You can recognize a constituent by its staying together as a unit in such a `displacement' process.
We will refer to such processes with the term \keyword{movement}.
So, a constituent can be recognized as such because it can move to another position in the sentence---typically, to the first position (note that it is not the case that any constituent can move to any position in a sentence; this depends on the exact rules of syntax of a particular language).
A string of words that do not form a constituent cannot move like this.
\begin{exe}
    \ex[]{
    \begin{xlist}
        \ex[]{Mary will never read \textbf{that novel by Robinson}.}
        \label{CHEx13a}
        \ex[]{\textbf{That novel by Robinson}, Mary will never read (although she's a keen reader in general).}
        \ex[*]{\textbf{That novel}, Mary will never read \textbf{by Robinson}.}
    \end{xlist}
    }
    \ex[]{
    \begin{xlist}
        \ex[]{John never will \textbf{read a thousand novels}.}
        \label{vp_move}
        \ex[]{\textbf{Read a thousand novels}, John never will.}
        \ex[*]{\textbf{Read}, John will never \textbf{a thousand novels}.}
    \end{xlist}
    }
    \ex[]{
    \begin{xlist}
        \ex[]{Shelly was reading a book \textbf{in the garden}.}
        \label{pp_move}
        \ex[]{\textbf{In the garden}, Shelly was reading a book.}
        \ex[*]{A book \textbf{in the garden}, Shelly was reading.} \label{object_adjunct_move}
    \end{xlist}
    }
\end{exe}

We can conclude from these examples that in (\ref{CHEx13a}) \emph{that novel by Robinson} is a constituent (an NP), in (\ref{vp_move}) \emph{read a thousand novels} is a constituent (a VP), and in (\ref{pp_move}) \emph{in the garden} is a constituent (a PP).
We also do not have evidence that \emph{a book in the garden} is a constituent of (\ref{pp_move}), as shown by the ungrammaticality of (\ref{object_adjunct_move}).

    \subsection{Questions}
A combination of the replacement test and the movement test involves trying to \keyword{question} the string of words for which you want to determine if it is a constituent.
If you can replace the string by a \keyword{question word} or \keyword{wh-word} like \emph{who} or \emph{what} and move this question word to the first position of the sentence, then you are dealing with a constituent:
\begin{exe}
    \ex[]{
    \begin{xlist}
        \ex[]{Kim ate \textbf{a sandwich with veggie haggis}.}
        \ex[]{What did Kim eat?   Answer: \textbf{A sandwich with veggie haggis}.\hfill{}(NP)}
        \ex[*]{What did Kmim eat \textbf{veggie haggis}?  Answer: \textbf{A sandwich with}.\hfill{}(not a constituent)}
    \end{xlist}
    }
    \ex[]{
    \begin{xlist}
        \ex[]{Ruby was \textbf{in the large garden}.}
        \ex[]{Where was Ruby?    Answer: \textbf{In the large garden}.\hfill{}(PP)}
        \ex[*]{Where was Ruby \textbf{garden}?  Answer: \textbf{In the large}.\hfill{}(not a constituent)}
    \end{xlist}
    }
\end{exe}

In the case of questioned VPs, a form of \emph{do} must appear in addition to the moved question word, if there is no other auxiliary in the sentence:
\begin{exe}
    \ex[]{
    \begin{xlist}
        \ex[]{Beatrice will [read that book].}
        \ex[]{\emph{What} will Beatrice \emph{do}?}
    \end{xlist}
    }
\end{exe}

\section{\emph{It}-clefts}
\hfill{}\textbf{Skills:}~\ref{constituencytestNP},
\ref{constituencytestVP},
\ref{constituencytestother}

English has a construction known as the \keyword{cleft} construction.
Schematically, a cleft sentence looks as follows:
\begin{exe}
\ex[]{It --- form of \emph{to be} --- \textbf{[X]} --- \emph{that/who/which} --- Y}
\label{it_cleft_schema}
\end{exe}

In this scheme, X is something that represents the answer to an implicit question; it is the new information in the sentence, and there is also frequently some kind of contrast involved.
In brief, it is the \keyword{focus} of the sentence.
Here are some examples of clefts; what is special about the ``X'' or the ``Y'' part?
\begin{exe}
\ex[]{
\begin{xlist}
\ex[]{It was McTominay that scored the goal (not another player).}
\ex[]{It is the female bird that has yellow feathers (not the male).}
\ex[]{It is a strong tree that can carry such a heavy burden (not a weak tree).}
\end{xlist}
}
\end{exe}

Crucially, only constituents can be the focus of a cleft, i.e. the ``X'' part in the scheme in (\ref{it_cleft_schema}).
Hence, something is a constituent of a sentence if we can turn the sentence into a cleft that has the relevant string as its focus.
So for example, if you want to know whether the string \emph{with two sugars} is a constituent in the sentence \emph{She usually drinks her coffee with two sugars}, you would form the relevant cleft by placing the string \emph{with two sugars} in the focus position, and omitting it from the `tail' of the cleft (here we've indicated this omission by striking through the phrase in its original position).
\begin{exe}
    \ex[]{It is \textbf{with two sugars} that she usually drinks her coffee \sout{with two sugars}.}
\end{exe}
So this shows that in the sentence \emph{It is with two sugars that she usually drinks her coffee}, the string \emph{with two sugars} is indeed a constituent.
If you wanted to test whether \emph{her coffee} is a constituent in this same sentence, you would need instead to consider whether the following is grammatical:
\begin{exe}
    \ex[]{It is \textbf{her coffee} that she usually drinks \sout{her coffee} with two sugars.}    
\end{exe}

It is, so we know that \emph{her coffee} is a constituent in the sentence \emph{She usually drinks her coffee with two sugars}. 

Thus, from the examples in (\ref{it_cleft}) it can be deduced that in (\ref{it_cleftA}) \emph{the old lady with the grey coat} and \emph{on the steep stairs} are constituents, but \emph{the old lady} and \emph{on the steep} are not.
You will need to look at the `tails' of the clefts carefully to verify this---take your time!
\begin{exe}
    \ex[]{
    \begin{xlist}
        \ex[]{The old lady with the grey coat fell on the steep stairs}
        \label{it_cleftA}
        \ex[]{It was the old lady with the grey coat that fell on the steep stairs}
        \label{it_cleftB}
        \ex[]{It was on the steep stairs that the old lady with the grey coat fell}
        \label{it_cleftC}
        \ex[*]{It was the old lady that with the grey coat fell on the steep stairs}
        \label{it_cleftD}
        \ex[*]{It was on the steep that the old lady with the grey coat fell stairs}
        \label{it_cleftE}
    \end{xlist}
    }
    \label{it_cleft}
\end{exe}

This concludes our discussion of different constituency tests. The first tries to substitute a constituent with a simpler replacement (in some syntactic and semantic sense); the second tries to move the constituent around; and the third is a bit of a combination of the first two. Play around with these, and remember that if a string passes a test it's probably a constituent, whereas if it doesn't we still can't be sure that it isn't - can you come up with some examples of such ``false negatives''?

\end{document}
