\documentclass{article}
\usepackage{xr-hyper} %Adds referencing between handouts and the Skills.tex document to avoid typos (req. latexmkrc)
\externaldocument{Skills} %where to look for labels
\usepackage[hidelinks]{hyperref} %links and URLS
\usepackage[linguistics]{forest} %needs tikz, draws trees
\usepackage[margin=1in]{geometry} %page layout
\usepackage{graphicx} % Required for inserting images
\usepackage[T1]{fontenc} %Make sure to be able to get accented characters etc
\usepackage[utf8]{inputenc}
\usepackage[normalem]{ulem} %adds strikethrough and other commands
\setlength{\parindent}{0pt}%don't indent paragraphs...
\setlength{\parskip}{1ex plus 0.5ex minus 0.2ex} 
\usepackage{multicol} %adds columns
\usepackage{gb4e} %for formatting examples, works with leipzig and multicol
\primebars %setting for gb4e, adds bars for X-bar notation, allows switch between bar or %'%
\noautomath
\usepackage{tabto}
\usepackage{amssymb}
\usepackage{fancyhdr}
\usepackage{setspace}
\usepackage{pifont} %allows dingbats to be called (for the "crosses" and "ticks" defined below)
\usepackage{tipa} % IK


\usepackage{leipzig}%primarily used for the abbreviations

\usepackage[backend=biber,
            style=unified,
            natbib,
            maxcitenames=3,
            maxbibnames=99]{biblatex}
\addbibresource{references.bib}
\usepackage{attrib}%allows authors next to quote environments

\makeatletter
\def\@maketitle{%I guessed from the commenting out of the author below that you don't want an author, this just gets rid of the space associated with the author field
  \newpage
  \null
%  \vskip 2em%
  \begin{center}%
  \let \footnote \thanks
    {\LARGE {\@title}\par}
%    \vskip 1.5em%
%    {\large
%      \lineskip .5em%
%      \begin{tabular}[t]{c}%
%        \@author
%      \end{tabular}\par}%
    \vskip 1em%
    {\large \@date}%
  \end{center}%
  \par
%  \vskip 1.5em
}
\makeatother

\title{LEL2A: Syntax}
%\author{Instructor: Itamar Kastner}
\date{Semester 1, 2025--26}%changed to current academic year

\newcommand*{\sqb}[1]{\lbrack{#1}\rbrack}
\newcommand*{\fn}[1]{\footnote{#1}}
\newcommand{\keyword}[1]{\textsc{#1}}
\newcommand{\cmark}{\ding{51}}
\newcommand{\xmark}{\ding{55}}
\newcommand{\subtitle}[1]{\maketitle\begin{center}{\Large #1}\end{center}}
\newcommand\blue[1]{\textcolor{blue}{#1}} % Itamar is lazy (I am Itamar)
\makeatletter
\newcommand*{\addFileDependency}[1]{% argument=file name and extension
\typeout{(#1)}% latexmk will find this if $recorder=0
% however, in that case, it will ignore #1 if it is a .aux or 
% .pdf file etc and it exists! If it doesn't exist, it will appear 
% in the list of dependents regardless)
%
% Write the following if you want it to appear in \listfiles 
% --- although not really necessary and latexmk doesn't use this
%
\@addtofilelist{#1}
%
% latexmk will find this message if #1 doesn't exist (yet)
\IfFileExists{#1}{}{\typeout{No file #1.}}
}\makeatother

\newcommand*{\myexternaldocument}[1]{%
\externaldocument{#1}%
\addFileDependency{#1.tex}%
\addFileDependency{#1.aux}%
}
\myexternaldocument{Skills} %also necessary for cross referencing, to reference other documents duplicate with name of document

\usepackage{comment}
\begin{document}
\maketitle
\subtitle{Topic 10 Course Notes: Non-finite clauses Part 2\\
Raising \& Control Revisited}
\hfill{}\textbf{Skills:}~\ref{A_movement} %,\ref{object_control}

\begin{comment}
\section{Making sense of some supporting data on non-finite clauses}
\hfill{}\textbf{Skill:}~\ref{A_movement}

As has been mentioned already, unstressed \emph{there} has a very restricted distribution:
\begin{exe}
\begin{multicols}{2}
    \ex{
    \begin{xlist}
        \ex[]{There is a problem.}
        \ex[]{Is there a problem?}
    \end{xlist}
    }\label{thereis}
    \ex{
    \begin{xlist}
        \ex[]{There are many solutions.}
        \ex[]{Are there many solutions?}
    \end{xlist}
    }
    \ex{
    \begin{xlist}
        \ex[*]{There robbed the bank.}
        \ex[*]{There admired my book.}
    \end{xlist}
    }
    \ex{
    \begin{xlist}
        \ex[]{There arrived several penguins.}
        \ex[]{There suddenly appeared a killer whale.}
    \end{xlist}
    }\label{therearrived}
    \end{multicols}
\end{exe}
Essentially, expletive, unstressed \emph{there} only appears as the subject of verbs of existence, or coming into existence.
By far its most frequent use is with \emph{be}, to express the existence of the entity corresponding to the noun phrase in post-verbal position, but it can appear also with a small set of intransitive verbs like \emph{arrive} and \emph{appear} (but not \emph{disappear}!), as in (\ref{therearrived}).
Notice that the $\theta$-role of \emph{arrive} and \emph{appear} does not seem to be assigned to \emph{there}, but rather to the postverbal nominal. 
Given this limitation of where unstressed \emph{there} is licensed, it is not surprising, given what we have said about \emph{try}, that \emph{there} cannot appear as its subject:
\begin{exe}
    \ex{
    \begin{multicols}{2}
        \begin{xlist}
            \ex[*]{There tried to be a problem.}
            \ex[*]{There tried to be many solutions.}
            \ex[*]{There tried to rob the bank.}
            \ex[*]{There tried to admire my book.}
        \end{xlist}
        \end{multicols}
        }\label{theretried}
\end{exe}
\vspace{-1em}
\emph{Try} is not a verb of existence or coming into existence, rather it has a Agent $\theta$-role to assign to its subject, and as we have seen unstressed \emph{there}, as a kind of expletive, cannot receive a $\theta$-role.
So the ungrammaticality of the examples in (\ref{theretried}) is just as predicted.

What is more interesting is the pattern we observe with \emph{seem}, as illustrated in (\ref{thereseemed}):
\begin{exe}
    \ex[]{
    \begin{xlist}
    \begin{multicols}{2}
        \ex[]{There seemed to be a problem.}
        \label{thereseemed_A}
        \ex[]{There seemed to be many solutions.}
        \label{thereseemed_B}
        \ex[*]{There seemed to rob the bank.}
        \label{thereseemed_C}
        \ex[*]{There seemed to admire my book.}
        \label{thereseemed_D}
    \end{multicols}
    \end{xlist}
    }\label{thereseemed}
\end{exe}
\vspace{-1em}
Here we can't say that unstressed \emph{there} cannot appear in subject position of a clause with \emph{seem} as its verb, as (\ref{thereseemed_A}) \& (\ref{thereseemed_B}) are grammatical, in contrast to (\ref{thereseemed_C}) \& (\ref{thereseemed_D}).
What distinguishes the two sets of cases here is that the verb in the \textbf{infinitival} clause in the grammatical cases (\ref{thereseemed_A}) \& (\ref{thereseemed_B}) licenses \emph{there}, while the verb in the \textbf{infinitival} clause in the ungrammatical cases does not.
Now, since we have argued that \emph{seem} is a \keyword{raising verb}, which assigns no $\theta$-role itself, this is also as predicted.
\emph{There} in (\ref{thereseemed_A}) \& (\ref{thereseemed_B}) is licensed \textbf{in its initial position in the infinitival clause}; it then moves (``raises'') into the matrix clause containing \emph{seem}, which does not assign it a $\theta$-role as it has none to assign.
But in (\ref{thereseemed_C}) \& (\ref{thereseemed_D}) \emph{there} is not licensed in the infinitival clause (nor in the matrix clause), so the result is simply ungrammatical.
\end{comment}

\section{The subject requirement: Case and/or EPP}
\hfill{}\textbf{Skill:}~\ref{A_movement}

One question we can ask about raising constructions is why the DP that raises \emph{must} do so.
That is, why do we get (\ref{why_raise_A}) rather than (\ref{why_raise_B}--c)?
\begin{exe}
    \ex[]{
    \begin{xlist}
        \ex[]{Jenny seems to enjoy politics.}\label{why_raise_A}
        \ex[*]{Seems to Jenny enjoy politics.}\label{why_raise_B}
        \ex[*]{Seems Jenny to enjoy politics.}\label{why_raise_C}
    \end{xlist}
    }\label{why_raise}
\end{exe}

We've pointed out a number of times that English finite clauses need to have a subject. This is why, instead of raising the embedded subject in~(\ref{why_raise_A}), we could also use an expletive matrix subject:
\ea[]{There seems to be enjoyment of politics.}
\z

In fact this is really essentially the same question that arises even in simple sentences, given that we have assumed that subjects originate inside the VP.
That is, why do we get (\ref{jennymay}) rather than (\ref{mayjenny}) as the declarative clause?\footnote{(\ref{mayjenny}) is of course fine as a question, but we'll see in a future topic that questions actually involve \emph{more} movement, rather than less.}
\begin{exe}
    \ex[]{
    \begin{xlist}
        \ex[]{Jenny may enjoy politics.}\label{jennymay}
        \ex[*]{May Jenny enjoy politics.}\label{mayjenny}
    \end{xlist}
    }
\end{exe}
There are two possible answers.
I'll outline them briefly here, but we don't have time in this course to go into them in great depth; instead we can refer to a general \keyword{subject requirement} in English, which can be viewed either in terms of \keyword{Case} or the \keyword{EPP}. Some more background on both of these follows below, if you're interested.

    \subsection{The subject requirement as Case}
First, one possible answer that has been given is that all DPs require \keyword{case}.
\keyword{Accusative}/objective case is assigned by transitive verbs; but it seems that \keyword{nominative} case is assigned by finite \obar{I}.
If we assume that \keyword{nominative} case in English is assigned in a \keyword{Specifier--Head} configuration (sometimes called `Spec--Head agreement'), that will force the subject argument in the specifier of VP to move in order to get the case it requires:
\begin{exe}
\small
\ex{
    \begin{forest}
        [
        \iibar{I}
        [\iibar{D}\textsubscript{\textsc{nom}} [\ibar{D} [\obar{D}\\she, name=copy]]] [\ibar{I}
        [\obar{I}\textsubscript{\textsc{nom}}\\may][\iibar{V}
        [$\langle$\sout{\iibar{D}}$\rangle$ [$\langle$\sout{she}$\rangle$, roof, name=trace]][\ibar{V} 
        [\obar{V}\textsubscript{\textsc{acc}}\\enjoy][\iibar{D}\textsubscript{\textsc{acc}} [\ibar{D} [\obar{D}\\them]]]]
        ]]
        ]
        \draw[->,dotted] (trace) to[out=south west,in=south] (copy);
    \end{forest}}
    \label{case_tree}
\end{exe}
 In (\ref{case_tree}), the closest \obar{I} above the VP was finite, so the subject DP only has to move to its specifier.
 In the case of a raising verb that selects an infinitival IP, however, the closest \obar{I} (\emph{to}) is not finite, and therefore not a source of \keyword{nominative} case.
 So in this case the subject DP would be forced to move up until it arrives at the specifier position of a finite \obar{I}, just as we have seen.

    \subsection{The subject requirement as an EPP feature on \obar{I}}
A second possible answer is that there is some property of \obar{I} (whether finite or not) that requires that its specifier be filled by some nominal element, regardless of case.
That essentially amounts to saying that the \keyword{Subject Requirement} (historically called the `EPP', though that term is now completely opaque) operates at the IP level.
This would, equally, force a DP to appear in its specifier position (the `subject position' of the clause).  

As mentioned above, if we assume that IP \textit{always} has to have a specifier, because of the subject requirement, we have to propose structures for raising and control where this movement takes places also in the infinitival clauses, even though this has no audible effect.
\newpage
\begin{exe}
    \ex{Control:\\
    \begin{forest}
        [
        \iibar{I}
        [\iibar{D}\textsubscript{\emph{i}} [\ibar{D} [\obar{D}\\they, name=copy1]]] [\ibar{I}
        [\obar{I}\\\lbrack{}\textsc{past}\rbrack{}] [\iibar{V}
        [$\langle$\sout{\iibar{D}}$\rangle$ [$\langle$\sout{they}$\rangle$, roof, name=trace1]] [\ibar{V}
        [\obar{V}\\tried] [\iibar{I}
        [\iibar{D}\textsubscript{\emph{i}} [\ibar{D} [\obar{D}\\\textsc{pro}, name=copy]]] [\ibar{I}
        [\obar{I}\\to][\iibar{V}
        [$\langle$\sout{\iibar{D}}$\rangle$ [$\langle$\sout{\textsc{pro}}$\rangle$, roof, name=trace]] [\ibar{V}
        [\obar{V}\\complete][\iibar{D} [this task, roof]]]]]]]
        ]]
        ]
        \draw[->,dotted] (trace) to[out=south west,in=south] (copy);
        \draw[->,dotted] (trace1) to[out=south west,in=south] (copy1);
    \end{forest}
    }
    \ex{Raising:\\
    \begin{forest}
        [
        \iibar{I}
        [\iibar{D} [\ibar{D} [\obar{D}\\you, name=copy]]][\ibar{I}
        [\obar{I}\\\lbrack{}\textsc{pres}\rbrack{}][\iibar{V}
        [$\langle$\sout{\iibar{D}}$\rangle$ [$\langle$\sout{you}$\rangle$, roof]][\ibar{V}
        [\obar{V}\\seem][\iibar{I}
        [$\langle$\sout{\iibar{D}}$\rangle$ [$\langle$\sout{you}$\rangle$, roof, name=trace1]][\ibar{I}
        [\obar{I}\\to][\iibar{V}
        [$\langle$\sout{\iibar{D}}$\rangle$ [$\langle$\sout{you}$\rangle$, roof, name=trace]][\ibar{V}
        [\obar{V}\\enjoy][\iibar{D} [those old movies, roof]]]]]]]
        ]]
        ]
        \draw[->,dotted] (trace) to[out=south west,in=south] (trace1);
        \draw[->,dotted] (trace1) to[out=south west,in=south] (copy1);
    \end{forest}
    }
\end{exe}

\section{More infinitival complements}
\hfill{}\textbf{Skill:}~\ref{A_movement}%, \ref{object_control}

We've seen that we need to work out what verb/adjective selects (or assigns a $\theta$-role to) what DP(s).
On the basis of evidence from selection / $\theta$-role assignment we have postulated the existence of a phonetically null DP (\textsc{pro}), which appears in some, but not all, infinitival clauses.

Bearing that in mind, let's look at some more infinitival complement clauses.
The questions here are: what are the arguments of the matrix verbs? And relatedly, what is assigning a $\theta$-role to the italicized DPs, the verb in the matrix clause (\emph{believed}, \emph{consider}, \emph{expect}, \emph{persuaded}, \emph{ordered}) or the verb in the infinitival clause (\emph{have}, \emph{lie}, \emph{know}, \emph{sign}, \emph{answer})?
\begin{exe}
    \ex{
    \begin{xlist}
        \ex[]{In the past, people believed \emph{women} to have a lower tolerance for pain than men.}
        \ex[]{They consider \emph{the witnesses} to be lying.}
        \ex[]{I expect \emph{everyone} to know the answer.}
    \end{xlist}
    }
    \ex{
    \begin{xlist}
        \ex[]{The lawyer persuaded \emph{the woman}  to sign away her fortune.}\label{persuade_gram}
        \ex[]{They convinced \emph{the politicians} to answer the questions.}\label{convinced}
    \end{xlist}
    }
\end{exe}

If you look at these examples closely, you might end up with the intuition that in the first set, the entire embedded clause is the complement of the verb: we believe or consider the entire proposition. But in the second set, the embedded subject seems to get a thematic role from the matrix verb itself: the woman is persuaded and the politicians get convinced.

We can now combine this intuition with the analytical tools we have and try to use an expletive, embedded subject. See if you can predict which of the two classes of verbs above would allow it.
\begin{exe}
    \ex{
    \begin{xlist}
        \ex[]{I believed/considered/expected there to be several problems}
        \ex[*]{I persuaded/convinced there to be several problems.}\label{persuade_ungram}
    \end{xlist}
    }
\end{exe}
Why are the examples in (\ref{persuade_ungram}) ungrammatical? 
It appears that \emph{there} doesn't denote something that can be persuaded or convinced.
That is to say, the DP that follows \emph{persuade} or \emph{convince} has to denote something animate, and capable of thought and understanding.
But that means that \emph{persuade} and \emph{convince} place \keyword{selectional restrictions} on this DP.
Which means that the DP must be receiving its $\theta$-role from them; that it is one of their \keyword{arguments}.
So in the grammatical examples (\ref{persuade_gram}--\ref{convinced}), \emph{the woman} and \emph{the politicians} are assigned $\theta$-roles by \emph{persuaded} and \emph{convinced}, respectively.

\subsection{The `ECM'  construction}
The situation is different for \emph{believe}, \emph{consider}, and \emph{expect}.
These verbs do not seem to impose any selectional restrictions on the DP that follows them at all. 
For example, if the \emph{infinitival} predicate is one that allows \emph{there} as a subject, it will be ok in this construction:
\begin{exe}
    \ex{
    \begin{xlist}
        \ex[]{There are several problems.}
        \ex[]{They believed/considered/expected there to be several problems.}
    \end{xlist}
    }
\end{exe}
If on the other hand the infinitival predicate is \emph{not} one that allows \emph{there} as a subject, it will not be ok in this construction either:
\begin{exe}
    \ex{
    \begin{xlist}
        \ex[]{\{Joshua/*there\} has a great smile.}\label{joshua-a}
        \ex[]{They believe/consider/expect \{Joshua/*there\} to have a great smile.}\label{believe-case}
    \end{xlist}
    }\label{joshuathere}
\end{exe}
So we can conclude that in the case of \emph{believe}, \emph{consider}, and \emph{expect}, the following DP is an argument of the infinitival verb, and \textbf{not} of \emph{believe}, \emph{consider}, and \emph{expect}.
Here is a possible tree that captures this:
\begin{exe}
    \ex{
    \begin{forest}
        [
        \iibar{I}
        [\iibar{D} [\ibar{D} [\obar{D}\\they, name=copy1]]][\ibar{I}
        [\obar{I}\\\lbrack{}\textsc{pres}\rbrack{}][\iibar{V}
        [$\langle$\sout{\iibar{D}}$\rangle$ [$\langle$\sout{they}$\rangle$, roof, name=trace1]][\ibar{V}
        [\obar{V}\\believe][\iibar{I}
        [\iibar{D} [\ibar{D} [\obar{D}\\Joshua, name=copy]]][\ibar{I}
        [\obar{I}\\to][\iibar{V}
        [$\langle$\sout{\iibar{D}}$\rangle$ [$\langle$\sout{Joshua}$\rangle$, roof, name=trace]][\ibar{V}
        [\obar{V}\\have] [\iibar{D} [a great smile, roof]]]]
        ]]]]]
        ]
        \draw[->,dotted] (trace) to[out=south west,in=south] (copy);
        \draw[->,dotted] (trace1) to[out=south west,in=south] (copy1);
    \end{forest}
    }\label{believe-tree}
\end{exe}
What this tree expresses is that \emph{believe} (and the same will be true for \emph{consider} and \emph{expect}) selects for two arguments: a \keyword{clause}---which expresses the proposition that is considered/believed---and the \keyword{agent}/\keyword{experiencer} (in this sentence, \emph{They}). 
The DP \emph{Joshua} is assigned a $\theta$-role by the infinitival verb \emph{have}; it's not selected by the matrix verb \emph{believe}.
So \emph{Joshua} is an argument of \emph{have} here: in very much the same way as is true when \emph{believe} selects for a finite clause:
\begin{exe}
    \ex{I believe that Joshua has a great smile.}\label{josh_smile}
\end{exe}

\paragraph{Extension: Case.} If you want to see how this lines up with the theory of Case we've alluded to above, consider the following. While \emph{Joshua}, the subject of the embedded infinitival clause, is an argument of the verb in the that clause (\emph{have} in (\ref{joshua-a}) \& (\ref{josh_smile})), in (\ref{believe-case}) \& (\ref{believe-tree}) we assume that it gets its accusative \keyword{case} from the verb in the matrix clause (\emph{believe}), because we have seen verbs assign accusative to following DPs but never seen \obar{I} heads assign accusative to preceding DPs.
So in this case the relation between the case-assigning head and the DP that needs case is different from either of the two patterns we have seen before.
Here the verb is assigning case to the specifier of its complement (this is what S\&K call \keyword{Head--Spec licensing}).
Because this type of case assignment is relatively less common, verbs like \emph{believe}, \emph{consider}, \emph{expect} are often called \keyword{Exceptional Case Marking} (ECM) verbs, and this construction is called the ECM construction.

\subsection{The Object Control construction}

Now that leaves us with the question of how to represent the sentences with \emph{persuade} and \emph{convince}, for example:
\begin{exe}
    \ex{
    \begin{xlist}
        \ex[]{They persuaded \emph{the politicians}  to answer the questions.}
        \ex[]{They convinced \emph{their grandmother} to write her memoirs.}
    \end{xlist}
    }
\end{exe}
If \emph{the politicians} and \emph{their grandmother} get their $\theta$-roles from \emph{persuade} and \emph{convince}, we seem to have a problem: \emph{answer} and \emph{write} both have \emph{two} $\theta$-roles to assign.
One will go to \emph{the questions}/\emph{her memoirs}.
But what about the other $\theta$-role that both have (Agent)?
Actually, we already have the crucial ingredient:  \textsc{pro}.
\textsc{pro} can act as the subject in the infinitival clause. 
\begin{exe}
    \ex{
    %\begin{multicols}{2}
    \begin{xlist}
        \ex{Arguments of \emph{persuade}:\\
        \emph{they} (Agent), \emph{the politicians} (Goal)}
    %\columnbreak
        \ex{Arguments of \emph{answer}:\\
        \textsc{pro} (Agent), \emph{the questions} (Theme)}
    \end{xlist}
    %\end{multicols}
    }
    \ex{They$_{[persuade(A)]}$ persuaded the politicians$_{[persuade(P)]}$ \lbrack{}\textsc{pro}$_{[answer(A)]}$ to answer\\ the questions$_{[answer(P)]}$\rbrack{}}
\end{exe}
The difference between these cases and cases like \emph{try} is that with \emph{try} it is the \textit{subject} of the matrix verb that \keyword{controls} \textsc{pro}, while with \emph{order} it is the object:
\begin{exe}
    \ex{
    \begin{xlist}
        \ex[]{\textit{They} tried [\textsc{pro} to answer the questions].}
        \ex[]{They persuaded \emph{them} [\textsc{pro} to answer the questions].}
    \end{xlist}
    }
\end{exe}
And what is structure for a sentence with \emph{convince} or \emph{persuade}?
Assuming, as we have been doing, that all of the arguments of a verb get their $\theta$-roles within its projection (its VP), we need room in the VP projected by \emph{convince/persuade} for \emph{three} arguments: the Agent (the person doing the persuading), the Goal (the person who is being persuaded), and the clause expressing the proposition (the state of affairs that has to be brought about).
The immediately obvious solution is to propose a structure along the lines of (\ref{ternary_tree}), where importantly \emph{the politicians} is co-indexed with \textsc{pro}:
\begin{exe}
    \ex{
    \small\begin{forest}
        [
        \iibar{I}
        [\iibar{D} [they, roof, name=copy1]][\ibar{I}
        [\obar{I}\\\lbrack{}\textsc{past}\rbrack{}][\iibar{V}
        [$\langle$\sout{\iibar{D}}$\rangle$ [$\langle$\sout{they}$\rangle$, roof, name=trace1]][\ibar{V}
        [\obar{V}\\persuaded][\iibar{D}\textsubscript{\emph{i}} [the politicians, roof]][\iibar{I}
        [\iibar{D}\textsubscript{\emph{i}} [\textsc{pro}, roof, name=copy]][\ibar{I}
        [\obar{I}\\to][\iibar{V}
        [$\langle$\sout{\iibar{D}}$\rangle$ [$\langle$\sout{\textsc{pro}}$\rangle$, roof, name=trace]][\ibar{V}
        [\obar{V}\\answer][\iibar{D} [the questions, roof]]]
        ]]]]]]
        ]
        \draw[->,dotted] (trace) to[out=south west,in=south] (copy);
        \draw[->,dotted] (trace1) to[out=south west,in=south] (copy1);
    \end{forest}
    }\label{ternary_tree}
\end{exe}

\paragraph{Extension: three-place predicates.} One problem with this tree is that V$'$ has three daughters (\keyword{ternary branching}), where we've implicitly stuck to a \keyword{binary branching} format until now.
You may have seen related problems in several tutorials.
There is in fact a way to provide a binary branching structure for all of these constructions---the trees that are given in S\&K rely on this.
We therefore face the same issues that we faced with \keyword{double object} verbs like \emph{give}, which similarly have three arguments: Agent (person doing the giving), Goal (person receiving the gift), Patient (object being transferred).
We saw that this is not so straightforward to achieve when Merge always only combines \textit{two} elements at a time, resulting in binary-branching trees.
This was one reason for postulating the abstract \keyword{cause} morpheme (also often referred to as `little \emph{v}') that provided an extra position. We can do the same here:
\begin{exe}
    \ex{
    \small\begin{forest}
        [
        \iibar{I}
        [\iibar{D} [they, roof, name=copy1]][\ibar{I}
        [\obar{I}\\\lbrack{}\textsc{past}\rbrack{}][\iibar{\emph{v}}
        [$\langle$\sout{\iibar{D}}$\rangle$ [$\langle$\sout{they}$\rangle$, roof, name=trace1]][\ibar{\emph{v}}
        [\obar{\emph{v}}
        [\obar{V}\\persuaded, name=copyv][\obar{\emph{v}}\\\lbrack{}\textsc{cause}\rbrack{}]][\iibar{V}
        [\iibar{D}\textsubscript{\emph{i}} [the politicians, roof]][\ibar{V}
        [$\langle$\sout{\obar{V}}$\rangle$\\$\langle$\sout{persuade}$\rangle$, name=tracev] [\iibar{I}
        [\iibar{D}\textsubscript{\emph{i}} [\textsc{pro}, name=copy]][\ibar{I}
        [\obar{I}\\to][\iibar{V}
        [$\langle$\sout{\iibar{D}}$\rangle$ [$\langle$\textsc{pro}$\rangle$, roof, name=trace]][\ibar{V}
        [\obar{V}\\answer][\iibar{D} [the questions, roof]]]
        ]]]]]]
        ]]]
        \draw[->,dotted] (trace) to[out=south west,in=south] (copy);
        \draw[->,dotted] (trace1) to[out=south west,in=south] (copy1);
        \draw[->,dotted] (tracev) to[out=south west,in=south] (copyv);
    \end{forest}
    }
\end{exe}
The main point to be clear about is which DP is an argument of which verb.
Differences in this respect are the main reason we propose a different structure in this case from the structure proposed for \emph{believe} in (\ref{believe-tree}).

\section{Are infinitival clauses always IPs?}
\hfill{}\textbf{Skills:} %~\ref{object_control}, 
\ref{raising_control}

In these notes so far, the infinitival clause has always appeared as an IP.
We know that it has to be at least this big, because it contains \emph{to}, which we think is an instance of the category \obar{I}.
But is it possible that it might be larger, in particular that it might be a CP?
Since in English we generally don't see/hear an overt complementizer in these clauses, if the infinitival clause is in fact a CP, it must be that the complementizer is phonologically/phonetically null.
You'll see from the suggested reading that there may be reasons to think that \keyword{control} infinitivals may be CPs, while \keyword{raising} infinitivals are just IPs (the most straightforward reason is that, in some languages, an overt complementizer is included in the relevant sentences).
However, at this point this is not a central issue for us. 

\newpage
\section{Summary \& Beyond the Course}

We now have an analysis for two different cases:
\begin{itemize}
\item \keyword{Raising} verbs that take an infinitival complement whose subject \keyword{raises} into the matrix clause:
\begin{exe}
    \ex Joshua seems [\sout{Joshua} to \sout{Joshua} enjoy classical music]
\end{exe}
\item \keyword{Subject control} verbs that take an infinitival complement whose subject is \textsc{pro}, controlled by the matrix \textit{subject}:
\begin{exe}
    \ex Joshua tries [\textsc{pro} to \sout{\textsc{pro}} enjoy classical music]
\end{exe}
\end{itemize}
There are other types of infinitival clauses that have been much discussed in the literature and that are touched on above.
In particular, there are examples where there doesn't seem to be any ``missing'' subject.
Examples include:
\begin{exe}
    \ex{I believe Joshua to enjoy classical music.}
    \label{beleive_to}
    \ex{I persuaded Joshua to listen to classical music.}
    \label{persuade_to}
\end{exe}
(\ref{beleive_to}) is an example of what is called \keyword{ECM} (Exceptional Case Marking), also occasionally called ``AcI'' as an abbreviation for the Latin term ``Accusativus cum Infinitivo'' (`accusative with infinitive').
(\ref{persuade_to}) is an example of \keyword{Object Control}.

\subsection{Unanswered Questions}
This topic leaves a number of questions unanswered and we will not have time to address them all in this course.
% You might also choose one of these questions as the basis for a macro-skills assessment.

As mentioned above, we have been assuming that all these embedded infinitival clauses are IPs.
But we have seen that many embedded clauses actually have more structure than that: they are often CPs.
Are any of these infinitival clauses also CPs?
What consequences would this have for the theory developed over the syntax block?

We've proposed that there is a phonetically null pronoun (\textsc{pro}) that occurs in some of these infinitival clauses.
But what prevents it from occurring all over the place?
In object position, for example?
Or as the subject of a finite clause?
Can we propose a theory consistent reason for the limited distribution of \textsc{pro}?

You may want to read further in the suggested reading and beyond. We will not be spending a great deal of time on Case, but you might want to have a look at Chapter 8 of S\&K so that you can better understand the references to case theory.

%\section*{Exercises}
%
%Good exercises to make sure you have understood the material are the following in the online SK text: 9.1A, 9.1B, 9.1D, 9.3, 9.5

% \subsection{Further Reading}

% The suggested reading---apart from these notes---is the two specified chapters in S\&K (Chapter 8 and 9).
% We will not be spending a great deal of time on Chapter 8 (on case), but you should read it so that you can understand references to case in Chapter 9 (and these notes).
% If you have questions about the reading, bring them to class or tutorials.

\end{document}