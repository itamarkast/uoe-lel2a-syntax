\documentclass{article}
\usepackage{xr-hyper} %Adds referencing between handouts and the Skills.tex document to avoid typos (req. latexmkrc)
\externaldocument{Skills} %where to look for labels
\usepackage[hidelinks]{hyperref} %links and URLS
\usepackage[linguistics]{forest} %needs tikz, draws trees
\usepackage[margin=1in]{geometry} %page layout
\usepackage{graphicx} % Required for inserting images
\usepackage[T1]{fontenc} %Make sure to be able to get accented characters etc
\usepackage[utf8]{inputenc}
\usepackage[normalem]{ulem} %adds strikethrough and other commands
\setlength{\parindent}{0pt}%don't indent paragraphs...
\setlength{\parskip}{1ex plus 0.5ex minus 0.2ex} 
\usepackage{multicol} %adds columns
\usepackage{gb4e} %for formatting examples, works with leipzig and multicol
\primebars %setting for gb4e, adds bars for X-bar notation, allows switch between bar or %'%
\noautomath
\usepackage{tabto}
\usepackage{amssymb}
\usepackage{fancyhdr}
\usepackage{setspace}
\usepackage{pifont} %allows dingbats to be called (for the "crosses" and "ticks" defined below)
\usepackage{tipa} % IK


\usepackage{leipzig}%primarily used for the abbreviations

\usepackage[backend=biber,
            style=unified,
            natbib,
            maxcitenames=3,
            maxbibnames=99]{biblatex}
\addbibresource{references.bib}
\usepackage{attrib}%allows authors next to quote environments

\makeatletter
\def\@maketitle{%I guessed from the commenting out of the author below that you don't want an author, this just gets rid of the space associated with the author field
  \newpage
  \null
%  \vskip 2em%
  \begin{center}%
  \let \footnote \thanks
    {\LARGE {\@title}\par}
%    \vskip 1.5em%
%    {\large
%      \lineskip .5em%
%      \begin{tabular}[t]{c}%
%        \@author
%      \end{tabular}\par}%
    \vskip 1em%
    {\large \@date}%
  \end{center}%
  \par
%  \vskip 1.5em
}
\makeatother

\title{LEL2A: Syntax}
%\author{Instructor: Itamar Kastner}
\date{Semester 1, 2025--26}%changed to current academic year

\newcommand*{\sqb}[1]{\lbrack{#1}\rbrack}
\newcommand*{\fn}[1]{\footnote{#1}}
\newcommand{\keyword}[1]{\textsc{#1}}
\newcommand{\cmark}{\ding{51}}
\newcommand{\xmark}{\ding{55}}
\newcommand{\subtitle}[1]{\maketitle\begin{center}{\Large #1}\end{center}}
\newcommand\blue[1]{\textcolor{blue}{#1}} % Itamar is lazy (I am Itamar)
\makeatletter
\newcommand*{\addFileDependency}[1]{% argument=file name and extension
\typeout{(#1)}% latexmk will find this if $recorder=0
% however, in that case, it will ignore #1 if it is a .aux or 
% .pdf file etc and it exists! If it doesn't exist, it will appear 
% in the list of dependents regardless)
%
% Write the following if you want it to appear in \listfiles 
% --- although not really necessary and latexmk doesn't use this
%
\@addtofilelist{#1}
%
% latexmk will find this message if #1 doesn't exist (yet)
\IfFileExists{#1}{}{\typeout{No file #1.}}
}\makeatother

\newcommand*{\myexternaldocument}[1]{%
\externaldocument{#1}%
\addFileDependency{#1.tex}%
\addFileDependency{#1.aux}%
}
\myexternaldocument{Skills} %also necessary for cross referencing, to reference other documents duplicate with name of document

\begin{document}
\maketitle
\subtitle{Topic 8 Course Notes: Passives}
\hfill{}\textbf{Skills:}~\ref{passives}

% In our last few topics we saw that in a clause containing a verb like \emph{seem} and a non-finite complement clause to this verb, the subject of the non-finite complement clause moves to the subject position of the clause containing seem, a process we called \keyword{raising}; another name for this sort of movement is \keyword{A-movement}. We then saw that this kind of movement to subject position (Specifier of IP) happens in a number of constructions. In this topic we will discuss one more construction, called the \keyword{passive}, in which a constituent moves to the subject position of the clause. For further reading and detail, there will also be sections dealing with \keyword{locality} restrictions on movement to subject position, in that the DP moving to subject position must be close enough to it in a sense to be made specific, and on additional investigations of the subject requirement.

\section{Properties of the passive}
\hfill{}\textbf{Skill:}~\ref{passives}

Many languages (though not all) have a construction known as the \keyword{passive}. In English, the alternation between an active clause and its passive counterpart looks like in the following pairs (where parentheses show optionality):
\begin{exe}
\begin{multicols}{2}
    \ex{
    \begin{xlist}
        \ex{Pavel invited Itamar to the party.}
        \ex{Itamar was invited to the party (by Pavel).}
    \end{xlist}
    }\label{passive_examples_1}
    \ex{
    \begin{xlist}
        \ex{Joan is feeding the elephants.}
        \ex{The elephants are being fed (by Joan).}
    \end{xlist}
    }\label{passive_examples_2}
\end{multicols}
\end{exe}

We can identify a number of systematic similarities differences between the active and passive clauses. Let's list them here, and then go through them one at a time:
\begin{enumerate}
    \item The active and passive clauses share selectional restrictions. This also means that the Theme of the active clause remains the Theme of the passive clause.
    \item The object of the active clause becomes the subject of the passive clause.
    \item The Agent can be left out altogether, or reintroduced in a \emph{by}-phrase.
    \item The verb and auxiliaries (if any) receive special morphological marking.
\end{enumerate}

    \subsection{Selectional restrictions}
We have seen that a transitive verb combines with two syntactic arguments, a subject that frequently realizes an Agent role and an object that frequently realizes a Theme role.
Moreover, we have seen that (at least in English) there is a particular order in which the verb combines with these two arguments.
First the verb combines with the object within the VP, then the subject combines with the VP as a whole.

We saw previously that a verb can impose selectional restrictions on its arguments.
A verb like \emph{invite}, for example, wants its direct object to refer to a human, while a verb like \emph{feed} only goes together felicitously with objects that refer to things that can take food.
It turns out that the selectional restrictions on the object in an active clause are exactly the same as the selectional restrictions on the subject in the passive counterpart of that clause.
This is illustrated in (\ref{passive_selection_1}--\ref{passive_selection_2}):
\begin{exe}
    \ex{
    \begin{xlist}
        \ex[]{Itamar invited his parents / \#only public transport / \#three bottles of whisky.}
        \ex[]{His parents / \#Only public transport / \#Three bottles of whisky were invited, too.}
    \end{xlist}
    }\label{passive_selection_1}
    \ex{
    \begin{xlist}
        \ex[]{Joan feeds  her cat /\#her bike / \#her shoe.}
        \ex[]{Her cat / \#her bike / \#her shoe  is regularly fed by Joan.}
    \end{xlist}
    }\label{passive_selection_2}
\end{exe}

This means that we're probably dealing with the same verb, and potentially the same elementary tree, in both the active and passive versions. The Theme of the active clause remains the Theme even in the passive clause.

    \subsection{Object becomes subject} % raising
Back in Topic 3a we discussed cases in which syntactic functions (like subject and object) don't align with thematic roles (like Agent and Theme). The passive examples show us that the subject is a Theme, just like the original Theme object of the active sentence. We can tell that it's the subject with our usual tool of agreement:
\ea
    \ea The child was invited to the party.
    \ex The children were invited to the party.
    \z
\z

% This is starting to look a bit like an instance of raising, where we had an object Theme which remains the Theme semantically but is now the subject. Why this happens is something we'll need to return to after we consider the other properties of the passive.

    \subsection{The Agent and \emph{by}-phrases}
While the Theme is preserved in the passive, the status of the Agent is less obvious. We can leave it out completely:
\ea Her parents were invited.
\z

But we can also re-introduce it in a PP headed by the preposition \emph{by}, often simply called a \keyword{\emph{by}-phrase}:
\ea Her parents were invited \textbf{by the First Minister}.
\z

There are two things worth keeping in mind for our analysis. The first is that because the Agent is now optional, we don't want to include it in our elementary tree for the verb - we don't have a way in our theory to chuck out phrases that we don't want anymore during the derivation!

The second is that the \emph{by}-phrase is limited to Agents, i.e.~animate beings that can do something more or less on purpose. Other Causers are incompatible with the \emph{by}-phrase and require some other kind of wording. And other Causers are also incompatible with some verbs, too.

Some verbs require an Agent subject, others don't, but \emph{by}-phrases do:
\ea
    \ea[]{The First Minister invited her parents.}
    \ex[*]{The wind invited her parents.}
    \z
\ex
    \ea[]{The first minister knocked over the glass.}
    \ex[]{The wind knocked over the glass.}
    \z
\ex
    \ea[]{Her parents were invited by the First Minister.}
    \ex[*]{Her parents were invited by the wind.}
    \z
\z

    \subsection{Passive morphology}
In English, the passive is formed by using a combination of the \textbf{past participle} of the main verb and the auxiliary \uline{\emph{be}}, plus any other auxiliaries which might also be involved:
\ea
    \ea Her parents \uline{were} \textbf{invited}.
    \ex Her parents will have \uline{been} \textbf{invited}.
    \z
\z

While we would want to capture the systematic aspects of the morphology in our theory, that's not something we'll be able to develop in LEL2A.

\section{Formal analysis}
\hfill{}\textbf{Skill:}~\ref{passives}

Let's recap what we've established so far.
\ea Properties of the passive:
    \ea The active and passive clauses share selectional restrictions.
    \ex The object of the active clause becomes the subject of the passive clause.
    \ex The Agent can be left out altogether, or reintroduced in a \emph{by}-phrase.
    \ex The verb and auxiliaries (if any) receive special morphological marking.
    \z
\z

Our solution will be to treat passives another case where the subject (Spec,IP) will be a constituent DP that moves there from another elementary tree, this time from the object position (complement of \obar{V}). The Agent will not be introduced in the Specifier of VP, but it can be added as a modifier \emph{by}-phrase PP (adjunction at the V-bar level).

Let's put this into practice. Assume that the constituent in the subject position of the passive clause actually is the object argument of the verb.
What happens in a passive, then, is that this object argument is moved to the subject position in the clause:
\begin{exe}
    \ex{
    \begin{forest}
        [
        \iibar{I}
        [\iibar{D} [her parents, roof, name=copy]][\ibar{I}
        [\obar{I}\\were][\iibar{V} [\phantom{X} ]
        [\ibar{V}
        [\obar{V}\\invited][\iibar{D}\\t, name=trace]]]]
        ]
        \draw[->,dotted] (trace) to[out=south west,in=south] (copy);
    \end{forest}
    }\label{passive-tree}
\end{exe}
\vspace{-2em}
As with all traces left by moved constituents, the trace in (\ref{passive-tree}) shares all properties with its antecedent (\emph{her parents} in this case).
The trace can therefore function as the object of the verb \emph{invited} and satisfy the selectional restrictions of that verb.
Note that no other element can occupy the object position, which can be taken as an indication that this position is indeed filled:
\begin{exe}
    \ex[*]{Her parents were invited her cousins.}\label{her_parents}
\end{exe}

The DP \emph{her parents} in (\ref{her_parents}) now functions as the grammatical subject of the clause (which lets it trigger the correct agreement, as we saw above).
% This is shown, amongst other things, by the fact that it shows agreement with the finite verb:
% \begin{exe}
%     \ex{
%     \begin{xlist}
%         \ex[]{Her father was/*were invited.}
%         \ex[]{Her parents *was/were invited.}
%     \end{xlist}
%     }
% \end{exe}

When the Agent does appear, it's as an adjunct:
\begin{exe}
    \ex{
    \begin{forest}
    [\iibar{I}
        [\iibar{D} [her parents, roof, name=copy]]
        [\ibar{I}
            [\obar{I}\\were]
            [\iibar{V}
                [{\phantom{X}} ]
                [\ibar{V}
                    [\ibar{V}
                        [\obar{V}\\invited][\iibar{D}\\t, name=trace]
                    ]
                    [\iibar{P}
                        [\phantom{X} ]
                        [\ibar{P}
                            [P\\by]
                            [DP [the First Minister, roof]]
                        ]
                    ]
                ]
            ]
        ]            
    ]
        \draw[->,dotted] (trace) to[out=south west,in=south west] (copy);
    \end{forest}
    }\label{passive-tree}
\end{exe}

Looking back at our list, all that's left is to account for the morphology:
\ea Properties of the passive:
    \ea The active and passive clauses share selectional restrictions.
    \ex The object of the active clause becomes the subject of the passive clause.
    \ex The Agent can be left out altogether, or reintroduced in a \emph{by}-phrase.
    \ex The verb and auxiliaries (if any) receive special morphological marking.
    \z
\z

The auxiliary appears in \obar{I}; we might ask whether the passive auxiliary is of the same category as other auxiliaries (more elaborate syntactic research has suggested that it is its own category). We then need to specify that the verb appears in the participial form. We don't have a mechanism for this, so we can assume some kind of selection between \obar{I} and \obar{V}, or just leave this for a more advanced course in morphology.

\section*{Further reading}

If you are interested in finding out more about the passive, a good place to start is the classic article \citet{jaeggli_passive_1986}.

\printbibliography
\end{document}