\documentclass{article}
\usepackage{xr-hyper} %Adds referencing between handouts and the Skills.tex document to avoid typos (req. latexmkrc)
\externaldocument{Skills} %where to look for labels
\usepackage[hidelinks]{hyperref} %links and URLS
\usepackage[linguistics]{forest} %needs tikz, draws trees
\usepackage[margin=1in]{geometry} %page layout
\usepackage{graphicx} % Required for inserting images
\usepackage[T1]{fontenc} %Make sure to be able to get accented characters etc
\usepackage[utf8]{inputenc}
\usepackage[normalem]{ulem} %adds strikethrough and other commands
\setlength{\parindent}{0pt}%don't indent paragraphs...
\setlength{\parskip}{1ex plus 0.5ex minus 0.2ex} 
\usepackage{multicol} %adds columns
\usepackage{gb4e} %for formatting examples, works with leipzig and multicol
\primebars %setting for gb4e, adds bars for X-bar notation, allows switch between bar or %'%
\noautomath
\usepackage{tabto}
\usepackage{amssymb}
\usepackage{fancyhdr}
\usepackage{setspace}
\usepackage{pifont} %allows dingbats to be called (for the "crosses" and "ticks" defined below)
\usepackage{tipa} % IK


\usepackage{leipzig}%primarily used for the abbreviations

\usepackage[backend=biber,
            style=unified,
            natbib,
            maxcitenames=3,
            maxbibnames=99]{biblatex}
\addbibresource{references.bib}
\usepackage{attrib}%allows authors next to quote environments

\makeatletter
\def\@maketitle{%I guessed from the commenting out of the author below that you don't want an author, this just gets rid of the space associated with the author field
  \newpage
  \null
%  \vskip 2em%
  \begin{center}%
  \let \footnote \thanks
    {\LARGE {\@title}\par}
%    \vskip 1.5em%
%    {\large
%      \lineskip .5em%
%      \begin{tabular}[t]{c}%
%        \@author
%      \end{tabular}\par}%
    \vskip 1em%
    {\large \@date}%
  \end{center}%
  \par
%  \vskip 1.5em
}
\makeatother

\title{LEL2A: Syntax}
%\author{Instructor: Itamar Kastner}
\date{Semester 1, 2025--26}%changed to current academic year

\newcommand*{\sqb}[1]{\lbrack{#1}\rbrack}
\newcommand*{\fn}[1]{\footnote{#1}}
\newcommand{\keyword}[1]{\textsc{#1}}
\newcommand{\cmark}{\ding{51}}
\newcommand{\xmark}{\ding{55}}
\newcommand{\subtitle}[1]{\maketitle\begin{center}{\Large #1}\end{center}}
\newcommand\blue[1]{\textcolor{blue}{#1}} % Itamar is lazy (I am Itamar)
\makeatletter
\newcommand*{\addFileDependency}[1]{% argument=file name and extension
\typeout{(#1)}% latexmk will find this if $recorder=0
% however, in that case, it will ignore #1 if it is a .aux or 
% .pdf file etc and it exists! If it doesn't exist, it will appear 
% in the list of dependents regardless)
%
% Write the following if you want it to appear in \listfiles 
% --- although not really necessary and latexmk doesn't use this
%
\@addtofilelist{#1}
%
% latexmk will find this message if #1 doesn't exist (yet)
\IfFileExists{#1}{}{\typeout{No file #1.}}
}\makeatother

\newcommand*{\myexternaldocument}[1]{%
\externaldocument{#1}%
\addFileDependency{#1.tex}%
\addFileDependency{#1.aux}%
}
\myexternaldocument{Skills} %also necessary for cross referencing, to reference other documents duplicate with name of document

\begin{document}
\maketitle
\subtitle{Topic 9 Course Notes: Non-finite clauses Part 1\\
To-infinitives, Raising \& Control}
\hfill{}\textbf{Skills:}~\ref{nonfin_to},
\ref{raising_control}

\section{Clauses are not necessarily finite}
\hfill{}\textbf{Skill:}~\ref{nonfin_to}

By now we've seen a lot of finite clauses.
Here are some more---each of the following examples has a \keyword{matrix clause} (also called a \keyword{root clause}) that contains a \keyword{subordinate clause} (also called an \keyword{embedded clause}).
\begin{exe}
    \ex[]{
    \begin{xlist}
        \ex[]{I expect \lbrack{}that I will be exonerated\rbrack{}.}
        \ex[]{They hope \lbrack{}that they will make a difference\rbrack{}.}
        \ex[]{It seems \lbrack{}that you've been living two lives\rbrack{}.}
        \ex[]{It is likely \lbrack{}that public sector pay will fall furthest in the poorest parts of the UK\rbrack{}.}
        \ex[]{Many people wonder \lbrack{}where they can get advice about credit card debt\rbrack{}.}
    \end{xlist}
    }
\end{exe}

What exactly is a clause?
A first pass at a semantic definition (although it will need to be improved):
\begin{exe}
    \ex[]{A clause is a syntactic constituent that expresses a proposition.}
\end{exe}

Syntacticians distinguish between \keyword{sentences} and \keyword{clauses}:
(\ref{clause_sentence_A}) is a clause and it is also a sentence. On the other hand, (\ref{clause_sentence_B}) is a single sentence, but it consists of two clauses (one nested inside the other):
\begin{exe}
    \ex[]{
    \begin{xlist}
        \ex[]{{[}She hated cats].}
        \label{clause_sentence_A}
        \ex[]{{[}Marjory said [that she hated cats]].}
        \label{clause_sentence_B}
    \end{xlist}
    }
\end{exe}

As it turns out, we find plenty of clauses that are \keyword{nonfinite}.
Here we're going to focus on some cases of \textsc{infinitival} clauses, where the \textsc{infinitive} is one non-finite form (other non-finite forms in English include present and past participles).%
\footnote{Terminology note: the grammatical terms here are \textsc{infinitive} (n) and \textsc{infinitival} (adj),
not ``infinite''.}
In the cases in (\ref{inf_examples}), the verb occurs with \emph{to} rather than person/number/tense inflection or a modal:
\begin{exe}
    \ex[]{
    \begin{xlist}
        \ex[]{I expect \lbrack{}to be exonerated\rbrack{}.}
        \ex[]{They hope \lbrack{}to make a difference\rbrack{}.}
        \ex[]{You seem \lbrack{}to have been living two lives\rbrack{}.}
        \ex[]{Public sector pay is likely \lbrack{}to fall furthest in the poorest parts of the UK\rbrack{}.}
        \ex[]{Many people wonder \lbrack{}where to get advice about credit card debt\rbrack{}.}
    \end{xlist}
    }
    \label{inf_examples}
\end{exe}

It isn't always the case that a speaker has a choice between finite and nonfinite clauses.
We've already seen that \keyword{root} clauses are nearly always finite:
\begin{exe}
    \ex[]{
    \begin{xlist}
        \ex[]{So how do you think the trial will end?}
        \ex[]{I will be exonerated.}
        \ex[*]{To be exonerated.}
        \ex[*]{To have been released.}
    \end{xlist}
    }
\end{exe}
Some verbs and adjectives that take clausal complements also only take \textit{finite} ones:
\begin{exe}
    \ex[]{
    \begin{xlist}
        \ex[]{I think \lbrack{}that I will go\rbrack{}.}
        \ex[*]{I think \lbrack{}to go\rbrack{}.}
    \end{xlist}
    }
    \ex[]{
    \begin{xlist}
        \ex[]{They state \lbrack{}that they will make a difference\rbrack{}.}
        \ex[*]{They state \lbrack{}to make a difference\rbrack{}.}
    \end{xlist}
    }
    \ex[]{
    \begin{xlist}
        \ex[]{It is probable \lbrack{}that public sector pay will fall furthest in the poorest parts of the UK\rbrack{}.}
        \ex[*]{Public sector pay is probable \lbrack{}to fall furthest in the poorest parts of the UK\rbrack{}.}
    \end{xlist}
    }
\end{exe}

Some other verbs and adjectives show the opposite pattern: they only take \textit{nonfinite} (and in fact for these examples, specifically \emph{infinitival}) complements:
\begin{exe}
    \ex[]{
    \begin{xlist}
        \ex[*]{I try \lbrack{}that I complete my tasks on time\rbrack{}.}
        \ex[]{I try \lbrack{}to complete my tasks on time\rbrack{}.}
    \end{xlist}
    }
    \ex[]{
    \begin{xlist}
        \ex[*]{It tends \lbrack{}that climate change deniers believe in conspiracy theories\rbrack{}.}
        \ex[]{Climate change deniers tend \lbrack{}to believe in conspiracy theories\rbrack{}.}
    \end{xlist}
    }
\end{exe}

What we're going to look at now is the syntax of nonfinite clauses built around \emph{to}-infinitives; we'll refer to these as \keyword{infinitival clauses}.

These infinitival clauses raise quite a few questions.
These include:
\begin{itemize}
    \item What is the syntactic status of the \emph{to} that appears in them? 
    \item Why do there seem to be arguments missing from the infinitival clauses?
\end{itemize}

\section{The status of \emph{to} with infinitives}
\hfill{}\textbf{Skill:}~\ref{nonfin_to}

Infinitival \emph{to} has the same spelling and pronunciation as the preposition \emph{to} from which it is historically derived.
However, infinitival \emph{to} does not behave like a preposition.\footnote{The classic paper on this is \citet{pullum_syncategorematicity_1982}; a more recent paper that aims to further defend the claims made there is \citet{levine_auxiliaries_2012}.}

\subsection{Modification}
PPs can be \emph{modified} by \emph{right} or \emph{straight}, given an appropriate meaning.
The \emph{to} in \emph{to}-infinitives can't.
\begin{exe}
    \ex[]{
    \begin{xlist}
        \ex[]{He went (right) to the edge of the cliff.}
        \ex[]{She walked (straight) to her supervisor.}
    \end{xlist}
    }
    \ex[]{
    \begin{xlist}
        \ex[]{He went (*right) to get some cash from the machine.}
        \ex[]{I try (*right) to complete my tasks on time.}
    \end{xlist}
    }
\end{exe}

\subsection{Ellipsis}
PPs don't generally allow ellipsis of their complements.
The \emph{to} in \emph{to}-infinitives does.
\begin{exe}
    \ex[]{
    \begin{xlist}
        \ex[]{Do you want to go to the cinema?}
        \ex[]{Yes, I really want to go to *(the cinema).}
    \end{xlist}
    }
    \ex[]{
    \begin{xlist}
        \ex[]{Do you expect to make a difference?}
        \ex[]{Yes, I expect to (make a difference).}
    \end{xlist}
    }
\end{exe}

In fact, the ellipsis-licensing properties of \emph{to} look instead like those of a modal:
\begin{exe}
    \ex[]{
    \begin{xlist}
        \ex[]{Are you thinking of going to the cinema?}
        \ex[]{Yes, I think I will (go to the cinema).}
    \end{xlist}
    }
\end{exe}

\subsection{Position}
Infinitival \emph{to} also appears in the same kind of \emph{position} as a modal, namely between the subject (when there is one---see below) and the nonfinite verb:
\begin{exe}
    \ex[]{
    \begin{xlist}
        \ex[]{I hope that \lbrack{}you \emph{will stay} away\rbrack{}.}
        \ex[]{I would very much prefer \lbrack{}for you \emph{to stay} away\rbrack{}.}
    \end{xlist}
    }
    \ex[]{
    \begin{xlist}
        \ex[]{I expected that \lbrack{}the audience \emph{would clap} at that point\rbrack{}.}
        \ex[]{I expected \lbrack{}the audience \emph{to clap} at that point\rbrack{}.}
    \end{xlist}
    }
\end{exe}

This makes sense if we have tense features on \obar{I}, because then \emph{to} simply marks non-finite tense, just like e.g.~\emph{will} marks future tense. We'll go ahead and assume the nonfinite \emph{to} is indeed of category \obar{I}. Otherwise these sentences are identical in their structure:
\begin{exe}
    \ex[]{
    \begin{multicols}{2}
    \begin{xlist}
        \ex[]{I expected that \dots\\
        \small\begin{forest}
        [\iibar{I}
        [\iibar{D}, name=copy [\ibar{D} [\obar{D}\\the][\iibar{N} [\ibar{N} [\obar{N}\\audience]]]]][\ibar{I}
        [\obar{I}\\would][\iibar{V} 
        [$\langle$\sout{\iibar{D}}$\rangle$ [$\langle$\sout{the audience}$\rangle$, roof, name=trace]][\ibar{V} [\ibar{V} 
        [\obar{V}\\clap]][\iibar{P} [at that point, roof]]]]]
        ]
       \draw[->,dotted] (trace) to[out=south,in=south] ([yshift=-14em] copy.south);
        \end{forest}}
        \ex[]{I expected \dots\\
        \small\begin{forest}
        [\iibar{I}
        [\iibar{D}, name=copy [\ibar{D} [\obar{D}\\the][\iibar{N} [\ibar{N} [\obar{N}\\audience]]]]][\ibar{I}
        [\obar{I}\\to][\iibar{V} 
        [$\langle$\sout{\iibar{D}}$\rangle$ [$\langle$\sout{the audience}$\rangle$, roof, name=trace]][\ibar{V} [\ibar{V} 
        [\obar{V}\\clap]][\iibar{P} [at that point, roof]]]]]
        ]
        \draw[->,dotted] (trace) to[out=south,in=south] ([yshift=-14em] copy.south);
        \end{forest}}
    \end{xlist}
    \end{multicols}
    }
\end{exe}

\section{Missing subjects of infinitival clauses}
\hfill{}\textbf{Skill:}~\ref{raising_control}

One other notable feature of most of the infinitival clauses exemplified above is that they appear to be ``missing'' one of the arguments of the infinitival verb:
\begin{exe}
    \ex[]{
    \begin{xlist}
        \ex[]{I believe \lbrack{}that I will complete this task\rbrack{}.}
        \ex[*]{I believe \lbrack{}that will complete this task\rbrack{}.}
    \end{xlist}
    }
    \ex[]{
    \begin{xlist}
        \ex[*]{I try \lbrack{}I/me to complete this task\rbrack{}.}
        \ex[]{I try \lbrack{}to complete this task\rbrack{}.}
    \end{xlist}
    }
    \ex[]{
    \begin{xlist}
        \ex[]{I expect \lbrack{}that I will complete this task\rbrack{}.}
        \ex[*]{I expect \lbrack{}that will complete this task\rbrack{}.}
        \ex[]{I expect \lbrack{}to complete this task\rbrack{}.}
    \end{xlist}
    }
\end{exe}

And it's not just that \emph{some} argument is missing: the missing argument in these examples is always the \textit{subject}.

How can we deal with this?

    \subsection{Approaches to the subjects of infinitival clauses}
\hfill{}\textbf{Skill:}~\ref{raising_control}

What might seem the simplest solution is just to suppose that in a nonfinite IP there is one fewer argument than in a finite IP.
So something along the lines of (\ref{null_subject}): 
\begin{exe}
    \ex[]{\small
    \begin{forest}
    baseline,
    for tree={parent anchor=south,child anchor=north,l=7ex,s sep=10pt},
        [\iibar{I}
        [\iibar{D} [\ibar{D} [\obar{D}\\they, name=copy]]][\ibar{I}
        [\obar{I}\\\lbrack{}\textsc{past}\rbrack{}][\iibar{V}
        [$\langle$\sout{\iibar{D}}$\rangle$ [$\langle$\sout{they}$\rangle$, roof, name=trace]][\ibar{V}
        [\obar{V}\\tried][\iibar{I}
        [\ibar{I}
        [\obar{I}\\to][\iibar{V}
        [\ibar{V}
        [\obar{V}\\complete][\iibar{D} [this task, roof]]]]]]]]]
        ]
        \draw[->,dotted] (trace) to[out=south,in=south] (copy);
    \end{forest}
    }
    \label{null_subject}
\end{exe}

But \emph{complete} is a transitive verb: it has two \textbf{$\theta$-roles} (abbreviation for `thematic roles') to assign---that is to say, it has two syntactic arguments, so selects for two DPs.
So at the very least we will need additional assumptions in order to explain why in these cases it is OK for the selectional requirements of a head not to be met, while in others it is not.
For this reason (among others), syntacticians have proposed two alternative solutions, for two different kinds of predicates.

\subsection{Raising: A-Movement of the subject}

Here is one solution: We have already seen that DPs may \keyword{move} to a higher position in the structure.
So far we've seen this in two cases. First, for subjects which originate in the Specifier of VP, as arguments of the verb, and then move to become the specifier of IP. And second, for the object of a passive verb, which gets \textsc{promoted} to subject.
Similarly, we may hypothesize that in some cases a DP may move to the specifier of a \textit{higher IP}. 

The structure for (\ref{seem_raising}) would then be something like (\ref{seem_raising_tree}). We're using a different matrix predicate for this example.
\begin{exe}
    \ex[]{
    \begin{xlist}
        \ex[]{You seem to enjoy those old movies.}
        \label{seem_raising}
        \ex[]{\small
        \begin{forest}
            [\iibar{I}
            [\iibar{D} [\ibar{D} [\obar{D}\\you, name=copy]]][\ibar{I}
            [\obar{I}\\\lbrack{}\textsc{pres}\rbrack{}][\iibar{V}
            [\ibar{V}
            [\obar{V}\\seem][\iibar{I}
            [$\langle$\sout{\iibar{D}}$\rangle$ [$\langle$\sout{you}$\rangle$, roof, name=trace1]][\ibar{I}
            [\obar{I}\\to][\iibar{V}
            [$\langle$\sout{\iibar{D}}$\rangle$ [$\langle$\sout{you}$\rangle$, roof, name=trace]][\ibar{V}
            [\obar{V}\\enjoy][\iibar{D} [old movies, roof]]]]]]]]]
            ]
            \draw[->,dotted] (trace) to[out=south west,in=south] (trace1);
            \draw[->,dotted] (trace1) to[out=south west,in=south] (copy);
        \end{forest}
        }
        \label{seem_raising_tree}
    \end{xlist}
    }
\end{exe}

Here the innovation is that after the subject \emph{you} moves from the embedded Spec,VP (where it did the enjoying) to the embedded Spec,IP (to satisfy the subject requirement), it then moves again to the specifier of the matrix IP (to satisfy the subject requirement of the matrix clause!).

We can call this a (subject-to-subject) \keyword{raising} analysis (because the subject of the embedded clause `raises' into the matrix clause, to become its subject).
Note that this is not very different to what we have proposed happens with modals like \emph{may}, \emph{might}, etc.

\subsection{Control: Some infinitival clauses contain a phonetically null pronoun in subject position, ``\textsc{PRO}''}

Here is another possibility, for a different set of predicates: We can hypothesize that there is a \keyword{null pronoun} that can function as the subject of an infinitive.
Just like any other pronoun, it can be the argument of a verb or adjective, and receive a $\theta$-role.  

This null element has been called \textsc{pro}.\footnote{This is always written in \textsc{small caps} or CAPS.
In fact sometimes people call this `big \textsc{pro}'. }
So then a first attempt at revising the structure in (\ref{null_subject}) would be as in (\ref{control}):
\begin{exe}
    \ex[]{\small
    \begin{forest}
        [\iibar{I}
        [\iibar{D}$_i$ [\ibar{D} [\obar{D}\\they, name=copy]]][\ibar{I}
        [\obar{I}\\\lbrack{}\textsc{past}\rbrack{}][\iibar{V}
        [$\langle$\sout{\iibar{D}}$\rangle$ [$\langle$\sout{they}$\rangle$, roof, name=trace]][\ibar{V}
        [\obar{V}\\tried][\iibar{I}
        [\iibar{D}$_i$ [\ibar{D} [\obar{D}\\\textsc{pro}, name=copy1]]][\ibar{I}
        [\obar{I}\\to][\iibar{V}
        [$\langle$\sout{\iibar{D}$_i$}$\rangle$ [$\langle$\sout{\textsc{pro}}$\rangle$, roof, name=trace1]][\ibar{V}
        [\obar{V}\\complete][\iibar{D} [this task, roof]]]]]]]]]
        ]
        \draw[->,dotted] (trace) to[out=south,in=south] (copy);
        \draw[->,dotted] (trace1) to[out=south,in=south] (copy1);
    \end{forest}
    }
    \label{control}
\end{exe}
We then need to require that \textsc{pro} be interpreted as \keyword{bound} by the matrix subject.\footnote{I have used subscript $_i$ to indicate this \keyword{binding} relationship.
In this case, \emph{bound by} appears to amount to \emph{corefers with}.
We will see later, however, that coreference isn't exactly the right concept here (it is more accurate to think that \textsc{pro} is `referentially dependent' on its binder).
You will often see the same subscripts used on traces of movement, which is is not an accident: \textsc{pro} and traces both need to know what their \keyword{antecedent} (the constituent which determines their identity) is, so they are both dependent elements.}
That is to say, in (\ref{control}) it is \emph{they} who do the trying, but also \emph{they} who will be the ones completing the task (if successful).
Another term for this particular relation, specifically as it applies to \textsc{pro}, is to say that \textsc{pro} is \keyword{controlled} by the matrix subject.
In consequence, this kind of analysis has come to be called a \keyword{control} analysis. To summarise: in a control structure there are two separate DPs (the ``lexical'' DP and \textsc{pro} in the nonfinite clause), which are coreferential.\footnote{In the tree above I have represented \textsc{pro} as moving to the subject position in the infinitival clause.
Since \textsc{pro} is silent, it is not a straightforward empirical matter to decide whether it is in [Spec,VP] or [Spec,IP].
There is one theoretical argument which favours \textsc{pro} in [Spec,IP], though: it would allow us to claim that every clause (even nonfinite ones) has a subject.}

\section{Distinguishing control from raising}

It turns out that in fact we need \emph{both} of these two solutions---\keyword{control} and \keyword{raising}---because the infinitival clauses that we have been looking at have different properties. 
\begin{itemize}
    \item Some are consistent with the control analysis---where there is a silent pronoun \textsc{pro};
    \item Others are consistent with the raising analysis---where the subject of the infinitival clause moves into the matrix clause.
\end{itemize}
The crucial observation is that on the evidence we have seen in simple cases, \emph{a single DP is only ever assigned one $\theta$-role}.
So, for example, (\ref{mary_admired}) cannot mean that \emph{Mary} admired herself:
\begin{exe}
    \ex[*]{Mary admired.}
    \label{mary_admired}
\end{exe}
If \emph{Mary} could appear first in complement position, getting the patient $\theta$-role of \emph{admire}, and could then move into the other argument position, getting the agent $\theta$-role, that interpretation ought to be possible:
\begin{exe}
    \ex[]{\small
    \begin{forest}
    baseline,
    for tree={parent anchor=south,child anchor=north,l=7ex,s sep=10pt},
        [\iibar{I}
        [\iibar{D} [Mary, roof, name=copy]][\ibar{I}
        [\obar{I}\\\lbrack{}\textsc{past}\rbrack{}][\iibar{V}
        [$\langle$\sout{\iibar{D}}$\rangle$ [$\langle$\sout{Mary}$\rangle$, roof, name=trace1]][\ibar{V}
        [\obar{V}\\admired][$\langle$\sout{\iibar{D}}$\rangle$ [$\langle$\sout{Mary}$\rangle$, roof, name=trace]]]]]
        ]
        \draw[->,dotted] (trace) to[out=south west,in=south] (trace1);
        \draw[->,dotted] (trace1) to[out=south west,in=south] (copy);
    \end{forest}
    }
    \label{mary_admired_mary}
\end{exe}
So we need to rule this out.
This has been done by adopting the following assumption, which is one half of what has been called the \keyword{$\theta$-Criterion}:\footnote{The other half is that every $\theta$-role has to be assigned to exactly one DP.}
\begin{exe}
    \ex[]{A non-expletive DP must receive \textbf{precisely one} $\theta$-role (an expletive DP cannot receive a $\theta$-role.)}
    \label{proto_theta_criterion}
\end{exe}

The structure in in (\ref{mary_admired_mary}) would violate the assumption in (\ref{proto_theta_criterion}) (and hence the $\theta$-criterion.
So that is how we rule out (\ref{mary_admired}) with the reading \emph{Mary admired herself}.
\begin{multicols}{2}
Control:\\
\small
\begin{forest}
baseline,
    for tree={parent anchor=south,child anchor=north,l=7ex,s sep=10pt},
    [\iibar{I}
        [\iibar{D}$_i$ [\ibar{D} [\obar{D}\\they, name=copy1]]][\ibar{I}
        [\obar{I}\\\lbrack{}\textsc{past}\rbrack{}][\iibar{V}
        [$\langle$\sout{\iibar{D}}$\rangle$ [$\langle$\sout{they}$\rangle$, roof, name=trace1]][\ibar{V}
        [\obar{V}\\tried][\iibar{I}
        [\iibar{D}$_i$ [\ibar{D} [\obar{D}\\\textsc{pro}, name=copy]]][\ibar{I}
        [\obar{I}\\to][\iibar{V}
        [$\langle$\sout{\iibar{D}$_i$}$\rangle$ [$\langle$\sout{\textsc{pro}}$\rangle$, roof, name=trace]][\ibar{V}
        [\obar{V}\\complete][\iibar{D} [this task, roof]]]]]]]]]
        ]
        \draw[->,dotted] (trace) to[out=south west,in=south] (copy);
        \draw[->,dotted] (trace1) to[out=south west,in=south] (copy1);
\end{forest}
%\hfill
\normalsize Raising:\\
\small
\begin{forest}
baseline,
    for tree={parent anchor=south,child anchor=north,l=7ex,s sep=10pt},
    [\iibar{I}
        [\iibar{D} [\ibar{D} [\obar{D}\\you, name=copy]]][\ibar{I}
        [\obar{I}\\\lbrack{}\textsc{pres}\rbrack{}][\iibar{V}
        [\ibar{V}
        [\obar{V}\\seem][\iibar{I}
        [$\langle$\sout{\iibar{D}}$\rangle$ [$\langle$\sout{you}$\rangle$, roof, name=trace1]][\ibar{I}
        [\obar{I}\\to][\iibar{V}
        [$\langle$\sout{\iibar{D}}$\rangle$ [$\langle$\sout{you}$\rangle$, roof, name=trace]][\ibar{V}
        [\obar{V}\\enjoy][\iibar{D} [old movies, roof]]]]]]]]]
    ]
        \draw[->,dotted] (trace) to[out=south west,in=south] (trace1);
        \draw[->,dotted] (trace1) to[out=south west,in=south] (copy);
\end{forest}
\end{multicols}

\normalsize
    \subsection{Diagnostics}
Now, if we go back to the \keyword{control} analysis and the \keyword{raising} analysis, we see that they actually make different claims about the number of $\theta$-roles that must be available, and exactly what head is selecting what argument:

In the \keyword{control} structure there are \emph{three} distinct DPs:  \emph{they}, \textsc{pro}, and \emph{this task}.
It is true that the first two both have to refer to the same individual(s), but as far as the syntax is concerned they are two separate DPs, just as is true of the two instances of \emph{I} in (\ref{identity_not_syntax_A}), or \emph{Joan} and \emph{she} in (\ref{identity_not_syntax_B}), or \emph{every employee} and \emph{they} in (\ref{identity_not_syntax_C}): 
\begin{exe}
    \ex[]{
    \begin{xlist}
        \ex[]{I think that I will bake a cake.}
        \label{identity_not_syntax_A}
        \ex[]{Joan thinks that she may bake a cake.}
        \label{identity_not_syntax_B}
        \ex[]{Every employee thinks that they are underpaid.}
        \label{identity_not_syntax_C}
    \end{xlist}
    }
\end{exe}
So, given our assumptions, if there are three distinct DPs in \emph{They tried to complete this task} (\emph{I, \textsc{pro}, this task}), there must be three $\theta$-roles: one for each DP.
One must be assigned by \emph{try}, and two by \emph{complete}.

In the \keyword{raising} structure there are only \textit{two} distinct DPs: \emph{you} and \emph{those old movies}.
The first one has moved, so it has occupied two different positions, and it leaves behind traces (or copies), just like the other instances of movement we have seen.
Crucially, we don't consider that a trace/copy is distinct from the moved DP in the relevant sense.
So there must only be two $\theta$-roles.
\emph{Enjoy} is a transitive verb (it assigns two $\theta$-roles), so if this is the right structure, it must be that \emph{seem} does not assign a $\theta$-role to \emph{any} DP.

Thus, if we think that a verb or adjective taking an infinitival complement assigns a $\theta$-role to the DP that appears to be its subject, that verb or adjective must be appearing in a \keyword{control} structure---or there will not be enough DPs to get each of the $\theta$-roles.
But if we think that a verb or adjective taking an infinitival complement does \emph{not} assign a $\theta$-role to the DP that appears to be its subject, it must be appearing in a \keyword{raising} structure---or there will be too many DPs for the number of $\theta$-roles available.

    \subsubsection{Diagnostic \#1: Expletive subjects}
How can we tell whether a given verb is used in a control or raising construction? If raising verbs don't require their own agent---which is why they need a lower DP to become their subject---then we should be able to get expletive subjects. This is correct: raising verbs allow expletive subjects, but control verbs do not.

\ea Raising:
    \ea[]{There seem to be many linguists in the room.}
    \ex[]{It appears to have been raining.}
    \z
\ex Control:
    \ea[*]{There hope to be bears finding fish.}
    \ex[*]{It tried the task to have been completed.}
    \z
\z

    \subsubsection{Diagnostic \#2: Agent-modifying adverbs}
If the subject of a control verb is a true Agent, then we should be able to ``find'' it in the semantics by referencing it with an adverb. And what's more, this adverb might conflict with an adverb in the embedded clause, modifying the embedded event. But there is no matrix agent that can be referenced with raising verbs (separately from that of the embedded clause).

\ea Control:
    \ea[]{They diligently tried to complete the task.}
    \ex[]{They diligently tried to complete the task sloppily.}
    \ex[]{The bears secretly hoped to find fish.}
    \ex[]{The bears secretly hope to find fish publicly.}
    \z
\ex Raising:
    \ea[*]{You diligently seem to enjoy old movies.}
    \ex[]{The bear appears to have eaten the fish excitedly.}
    \ex[*]{The bear excitedly appears to have eaten the fish.}
    \z
\z

    \subsubsection{Diagnostic \#3: Idiomatic readings}
Lastly, remember that one of our original pieces of evidence for the VPISH was that the interpretation of idioms is \keyword{local}, in that it's determined within a small syntactic domain (in that case the VP). See if you can figure out how the same logic is used here, and why idiomatic reading might be preserved in raising but not in control.

\ea Raising:
    \ea[]{The cat appears to be out of the bag. [The secret seems to have been revealed]}
    \ex[]{The shit seems to have finally hit the fan. [The crisis has now reached its point of no return]}
    \z
\ex Control:
    \ea[\#]{The cat tried to be out of the bag. [Literal meaning only]}
    \ex[\#]{The shit needs to finally hit the fan. [No reading on which the crisis needs to reach a point of no return]}
    \z
\z

\section{Extension: more on A-movement}
The type of grammatical subject we see in sentences with a raising verb (and in passives, as discussed in the last set of notes) is called a \keyword{derived} subject. It starts its syntactic life as an object or as the subject of a lower clause and is then `promoted' to a higher subject position.

The type of object-to-subject movement we see in passives must be distinguished from other types of movement, such as the movement process that puts questioned phrases in the first position of a clause in English, called \keyword{wh-movement}.
In contrast to movement that makes a phrase the grammatical subject of a clause, Wh-movement does not change the grammatical function of the moved phrase: just like \emph{her parents} in (\ref{marya}), \emph{whose parents} is the direct object of \emph{invite} in (\ref{maryb}), not the subject.
It is just in a different position than the usual one for objects, namely in the first position in the clause, in front of the subject position which is occupied by \emph{Mary} in both (\ref{marya}) and (\ref{maryb}).
(You can see that \emph{Mary} is still the subject in (\ref{maryb}), not \emph{whose parents}, because it is the former rather than the latter that determines the agreement on the finite verb.)
\begin{exe}
    \ex{
    \begin{xlist}
        \ex[]{Mary has invited her parents.}\label{marya}
        \ex[]{Whose parents$_{i}$ has Mary invited \emph{t}$_i$?}\label{maryb}
    \end{xlist}
    }
\end{exe}
The type of movement we see in passives and in raising constructions, which moves a constituent to the subject position of the clause, is called \keyword{A-movement}.
This term is used because the type of position to which this process moves a constituent is called an \keyword{A-position}, for historical reasons that need not concern us here.
In contrast, the type of movement that does not change the grammatical function of the moving constituent, such as \keyword{Wh-movement}, is called \keyword{A$’$-movement} (pronounced `A-bar movement'), and the type of position that this movement targets is called an \keyword{A$'$-position} (pronounced `A-bar position').
We will see in the next topic which position in the clause is the landing site for Wh-movement.

In the examples of passive above, such as (\ref{passive-tree}), there was A-movement from direct object position to subject position.
In some contexts, another constituent than the direct object is `promoted' to subject by this movement.
This happens, for instance, in the passive of a \keyword{double object} construction.
The double object construction is a sentence type containing a ditransitive verb, in which both the direct object and the indirect object are expressed by Noun Phrases, as in (\ref{igave}) for example.\footnote{Again this raises the question of the structure for a double object construction is, particularly since we have been assuming that all phrase structure trees are only binary branching.
See the item for the tutorial exercises for Topic 2 in Learn for some notes about ditransitives and further reading about this, if you are interested. }
In many varieties of English, it is the indirect object that is chosen to be promoted to the subject position in the passive counterpart of the double object construction, rather than the direct object, see (\ref{marywas}--\ref{thosebooks}):
\begin{exe}
    \ex[]{
    \begin{xlist}
        \ex[]{I gave Mary those books.}\label{igave}
        \ex[]{Mary$_i$ was given \emph{t}$_i$ those books.}\label{marywas}
        \ex[*]{Those books$_i$ were given Mary \emph{t}$_i$.}\label{thosebooks}
    \end{xlist}
    }
\end{exe}

\section{Extension: Locality}
% \hfill{}\textbf{Skills:}~\ref{passives},
% \ref{locality_constraints}

As it turns out, it is not possible to move a constituent across arbitrarily long distances by A-movement.
In particular, it turns out that:
\begin{exe}
    \ex[]{
    \begin{xlist}
        \ex[]{You cannot move a constituent to a subject position across another subject position.}\label{generalisation_A}
        \ex[]{You cannot move a constituent to a subject position X if there is another constituent that could also move to X and that is closer to X.}\label{generalisation_B}
    \end{xlist}
    }
\end{exe}
The restriction in (\ref{generalisation_A}) is known as the impossibility of \keyword{superraising}.
Its effect is shown by an example like (\ref{locality_D}), where movement of \emph{Sean} to the subject position in the highest clause skips the subject position of the intermediate clause, which is filled by \emph{it}.
This can be compared to the grammatical examples in (\ref{locality_B}) and (\ref{locality_C}), where movement to a higher subject position does not skip another subject (in (\ref{locality_C}) this is because there first is A-movement to the subject position of the passive intermediate clause, and then to the subject position of the highest clause, which contains a raising verb).
\begin{exe}
    \ex[]{
    \begin{xlist}
        \ex[]{It seems [that it was believed [that Sean plays the piano]].}\label{locality_A}
        \ex[]{It seems [that Sean$_i$ was believed [\emph{t}$_i$ to play the piano]].}\label{locality_B}
        \ex[]{Sean$_i$ seems [\emph{t}$_i$ to have been believed [\emph{t}$_i$ to play the piano]].}\label{locality_C}
        \ex[*]{Sean$_i$ seems [that it was believed [\emph{t}$_i$ to play the piano]].}\label{locality_D}
    \end{xlist}
    }\label{locality}
\end{exe}

The restriction in (\ref{generalisation_B}) is known as \keyword{superiority}.
Superiority is discussed most often in connection to A$'$-movement (on which more in a later lecture), but A-movement is subject to it, too.
Consider (\ref{superiority_example}).
There is evidence (that we cannot discuss here) that in a sentence like (\ref{superiority_example_A}) the direct object DP is in a higher position in the structure than the indirect object PP.
It is therefore closer to the subject position.
Superiority then demands that in the passive counterpart to (\ref{superiority_example_A}) it is the direct object argument that moves to subject position, rather than the indirect object argument.
This is correct, as (\ref{superiority_example_B}--\ref{superiority_example_D}) show.
Possibly, (\ref{superiority_example_C}) is ungrammatical for an independent reason, as English does not like PPs as subject very much in any case, but this does not hold for (\ref{superiority_example_D}) (note that the `preposition stranding' that occurs in (\ref{superiority_example_D}) is perfectly fine in other contexts in English, such as  \emph{Paula is never listened to}.
Therefore, that cannot be the source of the ungrammaticality of this example). 
\begin{exe}
    \ex{
    \begin{xlist}
        \ex[]{Mary showed some books to Paula.}\label{superiority_example_A}
        \ex[]{\lbrack{}Some books\rbrack{}$_i$ were shown \emph{t}$_i$ to Paula by Mary.}\label{superiority_example_B}
        \ex[*]{\lbrack{}To Paula\rbrack{}$_i$ was shown some books \emph{t}$_i$ by Mary.}\label{superiority_example_C}
        \ex[*]{\lbrack{}Paula\rbrack{}$_i$ was shown some books to \emph{t}$_i$ by Mary.}\label{superiority_example_D}
    \end{xlist}
    }\label{superiority_example}
\end{exe}

\section{Extension: Burzio’s generalization}
% \hfill{}\textbf{Skill:}~\ref{passives}

We have now seen a number of instances of A-movement in which a constituent is moved to the subject position of a finite clause, which we have assumed to be the spec-IP position.
So far, however, we have been silent on the question why A-movement takes place.
One area where people have a looked for an answer to this is \keyword{Case theory}. 

In languages with visible case morphology, we see that the different arguments of a verb can carry different case endings.
Typically, subjects show up in what is called a \keyword{nominative} case form, direct objects in an \keyword{accusative} case form, and indirect objects in a \keyword{dative} case form.\footnote{Notice, however, that in the course of its history English has lost the distinction between accusative and dative case, even on pronouns, which appear in the accusative case even when they are functioning as indirect objects.}
Apparently, if some DP receives a thematic role of the verb, it carries a particular case.
Case is therefore sometimes said to make the thematic role of an DP ‘visible’ for the interpretative component of the grammar (the semantic component):
\begin{exe}
    \ex[]{
    \begin{xlist}
        \ex[]{\keyword{Visibility condition}\\
		A DP that receives a thematic role from a verb must be assigned case.}
    \end{xlist}
    }\label{visibility_condition}
\end{exe}
There is no reason to assume that in languages without overt case morphology (such as---with the exception of the pronominal system---modern English) the requirement in (\ref{visibility_condition}) is just void.
The difference with languages showing morphological case is that the case-marking must involve abstract, nonvisible, case.
To indicate the difference with morphological case, this abstract case usually gets the distinction of being written with a capital C, i.e.\ as `Case'.

There is a tendency that languages without morphological case have a somewhat stricter word order than languages that have it (though it should be emphasised that this is, indeed, a tendency).
This may indicate that there are conditions on the assignment of abstract Case, such that a particular Case can only be assigned in a particular structural configuration.
The following restrictions on structural Case assignment have been proposed:
\begin{exe}
    \ex[]{
    \begin{xlist}
        \ex[]{Verbs and prepositions can assign abstract structural Case, but nouns and adjectives cannot.}\label{generalisations_2A}
        \ex[]{Verbs and prepositions can assign structural Accusative to a DP that they govern (where, roughly, V or P governs a DP if the DP is lower in the structure than V or P and there is no other Case-assigner intervening between the V or P and the DP)}\label{generalisations_2B}
        \ex[]{The inflection I on finite verbs can assign structural Nominative to the element in the associated spec-IP position.}\label{generalisations_2C}
        \end{xlist}
    }
\end{exe}
Returning now to the question of why there is A-movement, let us first consider what passivization does to a verb.
How does making a passive participle out of a verb affect the properties of this verb?
For one, this process must involve suppression of the assignment of the regular external theta-role of the verb, as the subject that gets this role in the active clause does not appear as an argument in the passive (but at most inside an optional PP adjunct).
Suppose now that another effect is that the Case-assigning capabilities of the verb are lost under passive participle formation.
In that case, if the internal argument would remain in the complement position to the verb, it would remain Caseless---which would violate the Visibility condition in (\ref{visibility_condition}).
Given that passive clauses do contain a verb that carries finite inflection (namely the finite auxiliary verb), Nominative Case is available in the spec-IP position.
Therefore, the object moves to spec-IP to receive this Case and thereby satisfy (\ref{visibility_condition}).

If so, we know what a defining characteristic of raising verbs must be.
Despite being active verbs, they, too, must have deficient Case-properties: they are not able to assign Accusative to the subject of their non-finite complement clause.
Nor can the subject of a non-finite clause get Nominative, because finite inflection is needed to assign that.
Like the object in a passive, then, the subject in the complement to a verb like \emph{seem} raises to the spec-IP position of the finite main clause in order to get Case (namely Nominative) there.

We see that, if this approach to the rationale of A-movement is correct, it must be the case in general that verbs that do not assign a thematic role to their subject position (such as passive participles and raising verbs) lack the capacity to assign Accusative Case to a constituent in (or inside) their complement position.
This correlation is known as \keyword{Burzio’s Generalization}, after the Italian syntactician Luigi Burzio.
\begin{exe}
    \ex[]{\keyword{Burzio’s Generalization}\\
		If a verb does not assign a thematic role to its subject position, it does not assign Accusative Case to its complement position, and vice versa.}\label{burzios_gen}
\end{exe}
It should be noted that the account outlined above is not wholly unproblematic, however.
People, including Burzio himself, have pointed out that (\ref{burzios_gen}) is quite stipulative, as it is not so clear why a verb’s capacity to assign abstract Case to its complement position should be related to its assigning a thematic role to its subject position or not.
Also, consider what happens with other elements that cannot assign abstract Case to their complements, namely adjectives and nouns (see (\ref{generalisations_2A})).
That these don't assign abstract Case to their complement is shown by the ungrammaticality of examples like those in (\ref{n_a_dont_assign_case}), where an N and an A, respectively, take a DP complement.
(Apparently, there is no abstract Case counterpart to the genitive case we typically see showing up on complements of N and A in languages that have morphological case.)
\begin{exe}
    \ex[]{
    \begin{xlist}
        \ex[*]{the destruction the city}
        \ex[*]{afraid dogs}
    \end{xlist}
    }\label{n_a_dont_assign_case}
\end{exe}
As and Ns also do not carry finite inflection, so any potential subject position in the NP or AP will not receive Nominative either.
Does this mean that As and Ns cannot select for a DP argument in languages without morphological case at all? Clearly not.
But what happens is that a meaningless preposition \emph{of} is plugged into the structure, as in (\ref{p_saves_the_day}).
The only function of this P seems to be to assign abstract Case to the DP in the complement.
\begin{exe}
    \ex[]{
    \begin{xlist}
        \ex[]{the destruction of the city}
        \ex[]{afraid of dogs}
    \end{xlist}
    }\label{p_saves_the_day}
\end{exe}
However, if this preposition is available to save DPs in the complement of an N or A from being without Case, and if lack of Case is the crucial property of complements to passive and raising verbs, we may wonder why it is not possible to use this preposition in passives and with raising verbs as well (rather than having the complement undergo A-movement).
That this is indeed impossible is shown by the examples in (\ref{p_passives}):
\begin{exe}
    \ex[]{
    \begin{xlist}
        \ex[*]{It has been invited of Mary.}
        \ex[*]{It seems of John to have left.}
    \end{xlist}
    }\label{p_passives}
\end{exe}
Instead of appealing to Case, there are other accounts (including one by Burzio) which make use of the fact that, quite independently of A-movement, we already need a principle in English that states that every clause must have a subject, namely the Subject requirement we have encountered before.
Perhaps that requirement is what triggers A-movement to subject position, too, but we will not pursue this here any further.

\printbibliography
\end{document}