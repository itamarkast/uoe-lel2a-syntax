\documentclass{article}
\usepackage{xr-hyper} %Adds referencing between handouts and the Skills.tex document to avoid typos (req. latexmkrc)
\externaldocument{Skills} %where to look for labels
\usepackage[hidelinks]{hyperref} %links and URLS
\usepackage[linguistics]{forest} %needs tikz, draws trees
\usepackage[margin=1in]{geometry} %page layout
\usepackage{graphicx} % Required for inserting images
\usepackage[T1]{fontenc} %Make sure to be able to get accented characters etc
\usepackage[utf8]{inputenc}
\usepackage[normalem]{ulem} %adds strikethrough and other commands
\setlength{\parindent}{0pt}%don't indent paragraphs...
\setlength{\parskip}{1ex plus 0.5ex minus 0.2ex} 
\usepackage{multicol} %adds columns
\usepackage{gb4e} %for formatting examples, works with leipzig and multicol
\primebars %setting for gb4e, adds bars for X-bar notation, allows switch between bar or %'%
\noautomath
\usepackage{tabto}
\usepackage{amssymb}
\usepackage{fancyhdr}
\usepackage{setspace}
\usepackage{pifont} %allows dingbats to be called (for the "crosses" and "ticks" defined below)
\usepackage{tipa} % IK


\usepackage{leipzig}%primarily used for the abbreviations

\usepackage[backend=biber,
            style=unified,
            natbib,
            maxcitenames=3,
            maxbibnames=99]{biblatex}
\addbibresource{references.bib}
\usepackage{attrib}%allows authors next to quote environments

\makeatletter
\def\@maketitle{%I guessed from the commenting out of the author below that you don't want an author, this just gets rid of the space associated with the author field
  \newpage
  \null
%  \vskip 2em%
  \begin{center}%
  \let \footnote \thanks
    {\LARGE {\@title}\par}
%    \vskip 1.5em%
%    {\large
%      \lineskip .5em%
%      \begin{tabular}[t]{c}%
%        \@author
%      \end{tabular}\par}%
    \vskip 1em%
    {\large \@date}%
  \end{center}%
  \par
%  \vskip 1.5em
}
\makeatother

\title{LEL2A: Syntax}
%\author{Instructor: Itamar Kastner}
\date{Semester 1, 2024-25}%changed to current academic year

\newcommand*{\sqb}[1]{\lbrack{#1}\rbrack}
\newcommand*{\fn}[1]{\footnote{#1}}
\newcommand{\keyword}[1]{\textsc{#1}}
\newcommand{\cmark}{\ding{51}}
\newcommand{\xmark}{\ding{55}}
\newcommand{\subtitle}[1]{\maketitle\begin{center}{\Large #1}\end{center}}
\makeatletter
\newcommand*{\addFileDependency}[1]{% argument=file name and extension
\typeout{(#1)}% latexmk will find this if $recorder=0
% however, in that case, it will ignore #1 if it is a .aux or 
% .pdf file etc and it exists! If it doesn't exist, it will appear 
% in the list of dependents regardless)
%
% Write the following if you want it to appear in \listfiles 
% --- although not really necessary and latexmk doesn't use this
%
\@addtofilelist{#1}
%
% latexmk will find this message if #1 doesn't exist (yet)
\IfFileExists{#1}{}{\typeout{No file #1.}}
}\makeatother

\newcommand*{\myexternaldocument}[1]{%
\externaldocument{#1}%
\addFileDependency{#1.tex}%
\addFileDependency{#1.aux}%
}
\myexternaldocument{Skills} %also necessary for cross referencing, to reference other documents duplicate with name of document

\begin{document}
\pagestyle{empty}
\maketitle
\subtitle{Tutorial Week 8: Topics 10, 11 \& 12}
% \section*{Question 1}%Source: Intro Syntax Assessment 2 Question 4
% \hfill{} \ref{A_movement} $\Box$,
% \ref{object_control} $\Box$,
% \ref{locality_constraints} $\Box$

% Given the examples in (\ref{believe_consider}):
% \begin{exe}
%     \ex{
%     \begin{xlist}
%         \ex{I believe those people to know the answer.}
%         \label{believe}
%         \ex{I consider their proposal to be ridiculous.}
%     \end{xlist}
%     }
%     \label{believe_consider}
% \end{exe}
% Here are two hypotheses, \ref{hyp_A} \& \ref{hyp_B}, about the behaviour of \emph{believe} and \emph{consider}:
% \begin{xlistA}
%     \ex[]{%\textbf{Hypothesis A}\\
%     These verbs have two arguments: a \iibar{D} subject argument and an infinitival \iibar{I} object argument.
%     These verbs are able to assign \keyword{accusative} case in an \textsc{exceptional} way, to the \iibar{D} occupying the specifier position of that infinitival \iibar{I} complement.
%     So the structure of (\ref{believe}) is roughly:\\ I believe \lbrack{}\textsubscript{\iibar{I}} \lbrack{}\textsubscript{\iibar{D}}those people\rbrack{}\textsubscript{\Acc{}} to know the answer\rbrack{}.
%     }
%     \label{hyp_A}
%     \ex[]{%\textbf{Hypothesis B}\\
%     These verbs have three arguments: a \iibar{D} subject argument, a \iibar{D} object argument, and an infinitival \iibar{I} second object argument.
%     The subject of the infinitival \iibar{I} is a \keyword{pro} whose reference is controlled by the \keyword{accusative} object in the main clause.
%     So the structure of (\ref{believe}) is roughly:\\ I believe \lbrack{}\textsubscript{\iibar{D}} those people\rbrack{}\textsubscript{\Acc{}} \lbrack{}\textsubscript{\iibar{I}} \textsc{pro} to know the answer\rbrack{}.
%     }
%     \label{hyp_B}
% \end{xlistA}
% Some of the following examples are consistent with either hypothesis, and some may support one over the other.
% Which of the examples can be used as an argument to support one hypothesis over the other, and which are “neutral” between the two hypotheses?
% \begin{exe}
%     \ex{
%     \begin{xlist}
%         \ex[*]{I believe they to know the answer.}
%         \ex[]{I believe there to be a good solution.}
%         \ex[]{I believe the answer to be known by those people.}
%         \label{A_case_tree}
%         \ex[]{They are believed to know the answer.}
%         \ex[]{There is believed to be a good solution.}
%         \ex[]{I believe the cat to be out of the bag. (with the interpretation: \emph{I believe that the secret has been exposed})}
%     \end{xlist}
%     }
% \end{exe}
% Write a short response to justify your answer.
% In your answer, draw two possible trees for (\ref{A_case_tree})---one implementing each hypothesis---and provide an accompanying rationale for choosing one over the other.

\section*{Question 1}%Source: LEL2A Week 8 tutorial \& IS Assessment 2
% \hfill{} \ref{raising_control} $\Box$,
% \ref{passives} $\Box$,
% \ref{wh_movement} $\Box$,
% \ref{i_to_c_movement} $\Box$

Instructions:
\begin{itemize}
    \item Provide an analysis (syntactic tree) for each of the following sentences, consistent with the course and represent this in tree form.
    \item You should represent all relevant forms of movement discussed in the course.
    \item Each sentence is tagged with the main Syntax Skills it relies on.
    \item These sentences are complex, synthesising everything we've worked on so far. But this also means that you should be able to separate the parts (skills) you're comfortable with from the ones you're not yet proficient in. In other words, do give all three a try!
    \item If you're still ``missing'' a Syntax Skill from a previous week and believe your tree demonstrates it, include a very brief note (1--2 sentences) after your tree.
    \item If you want to add some explanations about any of your trees, please keep them short too (e.g.~2--3 sentences or a constituency test).
\end{itemize}

The sentences!
\begin{exe}
    \ex[]{Whose bag did you try to steal? \label{whose_bag} \hfill \ref{np_dp} $\Box$, \ref{raising_control} $\Box$, \ref{passives} $\Box$, \ref{wh_movement} $\Box$}
    \ex[]{Dr Ali seems to forget which bag her instruments were put in.\label{which_bag} \hfill \ref{raising_control} $\Box$, \ref{A_movement} $\Box$, \ref{passives} $\Box$, \ref{wh_movement} $\Box$}
    \ex[]{Which stories about the pandemic did the historian consider writing an analysis of? \\
        \label{which_stories} \hfill \ref{np_structure} $\Box$, \ref{raising_control} $\Box$, \ref{A_movement} $\Box$, \ref{wh_movement} $\Box$}
\end{exe}


\section*{Question 2 (for discussion)}

What was the most interesting thing you learned in the LEL2A syntax block?

\end{document}