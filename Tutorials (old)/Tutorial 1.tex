\documentclass{article}
\usepackage{xr-hyper} %Adds referencing between handouts and the Skills.tex document to avoid typos (req. latexmkrc)
\externaldocument{Skills} %where to look for labels
\usepackage[hidelinks]{hyperref} %links and URLS
\usepackage[linguistics]{forest} %needs tikz, draws trees
\usepackage[margin=1in]{geometry} %page layout
\usepackage{graphicx} % Required for inserting images
\usepackage[T1]{fontenc} %Make sure to be able to get accented characters etc
\usepackage[utf8]{inputenc}
\usepackage[normalem]{ulem} %adds strikethrough and other commands
\setlength{\parindent}{0pt}%don't indent paragraphs...
\setlength{\parskip}{1ex plus 0.5ex minus 0.2ex} 
\usepackage{multicol} %adds columns
\usepackage{gb4e} %for formatting examples, works with leipzig and multicol
\primebars %setting for gb4e, adds bars for X-bar notation, allows switch between bar or %'%
\noautomath
\usepackage{tabto}
\usepackage{amssymb}
\usepackage{fancyhdr}
\usepackage{setspace}
\usepackage{pifont} %allows dingbats to be called (for the "crosses" and "ticks" defined below)
\usepackage{tipa} % IK


\usepackage{leipzig}%primarily used for the abbreviations

\usepackage[backend=biber,
            style=unified,
            natbib,
            maxcitenames=3,
            maxbibnames=99]{biblatex}
\addbibresource{references.bib}
\usepackage{attrib}%allows authors next to quote environments

\makeatletter
\def\@maketitle{%I guessed from the commenting out of the author below that you don't want an author, this just gets rid of the space associated with the author field
  \newpage
  \null
%  \vskip 2em%
  \begin{center}%
  \let \footnote \thanks
    {\LARGE {\@title}\par}
%    \vskip 1.5em%
%    {\large
%      \lineskip .5em%
%      \begin{tabular}[t]{c}%
%        \@author
%      \end{tabular}\par}%
    \vskip 1em%
    {\large \@date}%
  \end{center}%
  \par
%  \vskip 1.5em
}
\makeatother

\title{LEL2A: Syntax}
%\author{Instructor: Itamar Kastner}
\date{Semester 1, 2024-25}%changed to current academic year

\newcommand*{\sqb}[1]{\lbrack{#1}\rbrack}
\newcommand*{\fn}[1]{\footnote{#1}}
\newcommand{\keyword}[1]{\textsc{#1}}
\newcommand{\cmark}{\ding{51}}
\newcommand{\xmark}{\ding{55}}
\newcommand{\subtitle}[1]{\maketitle\begin{center}{\Large #1}\end{center}}
\makeatletter
\newcommand*{\addFileDependency}[1]{% argument=file name and extension
\typeout{(#1)}% latexmk will find this if $recorder=0
% however, in that case, it will ignore #1 if it is a .aux or 
% .pdf file etc and it exists! If it doesn't exist, it will appear 
% in the list of dependents regardless)
%
% Write the following if you want it to appear in \listfiles 
% --- although not really necessary and latexmk doesn't use this
%
\@addtofilelist{#1}
%
% latexmk will find this message if #1 doesn't exist (yet)
\IfFileExists{#1}{}{\typeout{No file #1.}}
}\makeatother

\newcommand*{\myexternaldocument}[1]{%
\externaldocument{#1}%
\addFileDependency{#1.tex}%
\addFileDependency{#1.aux}%
}
\myexternaldocument{Skills} %also necessary for cross referencing, to reference other documents duplicate with name of document

\begin{document}
\pagestyle{empty}
\maketitle
\subtitle{Tutorial Week 5: Topics 1, 2 \& 3}

\paragraph{Explanation on accumulating Syntax Skills:} Let me stress that the point of the syntax block is to learn how to think like a syntactician, learn about English, and try to have fun. But marks are involved, and here's how we mark Syntax Skills: generally speaking, it's enough to show them once. This means that if we have questions Q1A and Q2B in a given tutorial sheet (for example), both of which target skill 6b, then getting one of them right is enough. That said, it's best to attempt all questions.

Some tutorial questions are primarily \textbf{for discussion}. This means that they can count towards your skill tally, but don't worry if you don't get them right -- their goal is to allow for extended discussion in the tutorial itself and some will go beyond the discussion in class.

(Please remember that we're developing this approach to tutorials and marking together, so your feedback is welcome -- keep it specific and try to give it directly to Itamar.)

\section*{Question 1}
\hfill{}
%\ref{whatissyntaxA} $\Box$,
\ref{whatissyntaxB} $\Box$ 
%\ref{whatissyntaxC} $\Box$

\paragraph{Q1A} Given the following list of constituents, visually represent the structure of the sentence \emph{My cousin gave the dog a juicy bone}.
\begin{itemize}
    \item my cousin
    \item the dog
    \item a juicy bone
    \item gave the dog a juicy bone
\end{itemize}
Don’t worry too much about whether your representation is \emph{right} or not in other aspects, just as long as it
shows those four strings as constituents. For example, it doesn't need to have any labels for the constituents (like VP or NP).

\hfill{}\ref{constituencytestVP} $\Box$ %, \ref{diagnosticlimits} $\Box$

\paragraph{For discussion} Above, you were told that \emph{gave the dog a juicy bone} was a constituent.
However, if we apply an \emph{it-cleft} test for constituency to this string we get the following results:
\begin{exe}
    \ex[*]{It is gave the dog a juicy bone that my cousin}
\end{exe}
What does this mean for the representation you gave above?
Can you use other diagnostics to resolve this issue (one is enough but if you can give more that's even better)?

\section*{Question 2}
\hfill{}
\ref{constituencytestNP} $\Box$,
\ref{constituencytestVP} $\Box$,
\ref{constituencytestother} $\Box$
%, M3 $\Box$, M4 $\Box$

\paragraph{Q2A }Which of the following trees is the best representation of the structure for the sentence \emph{She placed that vase on the table}, based on what you can determine about the constituent
structure of the sentence? For each of the trees that you didn't choose in (\ref{Q2_trees}), what is the evidence against it?

\begin{exe}
\ex{
\begin{xlist}
    \ex{
    \small \begin{forest}
for tree={s=2mm, l-=5mm},
	[S [NP[Pronoun\\\emph{she}]] [VP [V\\\emph{placed}] [NP [Det\\\emph{that}] [NP [N\\\emph{vase}] [PP [P\\\emph{on}] [NP [Det\\\emph{the}] [N\\\emph{table}]]]]]]]
    \end{forest}}
    \ex{
    \small \begin{forest}
for tree={s=2mm, l-=5mm},
	[S [NP[Pronoun\\\emph{she}]] [VP [V\\\emph{placed}]
    [NP 
        [Det\\\emph{that}]
        [N\\\emph{vase}]
        [P\\\emph{on}]
        [NP [Det\\\emph{the}] [N\\\emph{table}]]]]]
    \end{forest}}
\ex{
    \small \begin{forest}
for tree={s=2mm, l-=5mm},
	[S [NP[Pronoun\\\emph{she}]] [VP [V\\\emph{placed}] [NP [Det\\\emph{that}] [NP [N\\\emph{vase}]]] [PP [P\\\emph{on}] [NP [Det\\\emph{the}] [N\\\emph{table}]]]]]
    \end{forest}}
    \end{xlist}}
    \label{Q2_trees}
\end{exe}

\newpage
\section*{Question 3}
\hfill{}
\ref{syn_sem_roles} $\Box$,
\ref{sem_roles} $\Box$


As you know, some verbs in English can appear either with two syntactic arguments, or just one:
\begin{exe}
    \ex{
    \begin{xlist}
        \ex{This film shocks.\tabto{10em} This film shocks a lot of people.}
        \ex{The donkey kicked.\tabto{10em} The donkey kicked the door.}
        \ex{The canary drank.\tabto{10em} The canary drank some water.}
    \end{xlist}}
    \label{semantic_rolesA}
    \ex{
    \begin{xlist}
        \ex{The branch broke.\tabto{10em} The bear broke the branch.}
        \ex{The cup dropped.\tabto{10em} The child dropped the bag.}
        \ex{The lake thawed.\tabto{10em} The sun thawed the ice.}
    \end{xlist}}
    \label{semantic_rolesB}
\end{exe}

Thinking of the ways in which participant roles (like AGENT or THEME) are mapped onto syntactic functions (like SUBJECT or OBJECT):

\paragraph{Q3A} What is the difference between the verbs in (\ref{semantic_rolesA}) and those in (\ref{semantic_rolesB})? Describe it in terms of the syntactic and semantic arguments of the predicates.

\paragraph{For discussion:} Can you think of other English verbs that behave like the verbs in (\ref{semantic_rolesA})? And like the verbs in (\ref{semantic_rolesB})?

\paragraph{For discussion:} If you speak or are familiar with another language, are there verbs in that language that behave to some extent (or exactly) like the verbs in (\ref{semantic_rolesB})? If you find a pattern that is similar but not identical, can you describe it?

\end{document}