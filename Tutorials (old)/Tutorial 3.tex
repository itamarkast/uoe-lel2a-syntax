\documentclass{article}
\usepackage{xr-hyper} %Adds referencing between handouts and the Skills.tex document to avoid typos (req. latexmkrc)
\externaldocument{Skills} %where to look for labels
\usepackage[hidelinks]{hyperref} %links and URLS
\usepackage[linguistics]{forest} %needs tikz, draws trees
\usepackage[margin=1in]{geometry} %page layout
\usepackage{graphicx} % Required for inserting images
\usepackage[T1]{fontenc} %Make sure to be able to get accented characters etc
\usepackage[utf8]{inputenc}
\usepackage[normalem]{ulem} %adds strikethrough and other commands
\setlength{\parindent}{0pt}%don't indent paragraphs...
\setlength{\parskip}{1ex plus 0.5ex minus 0.2ex} 
\usepackage{multicol} %adds columns
\usepackage{gb4e} %for formatting examples, works with leipzig and multicol
\primebars %setting for gb4e, adds bars for X-bar notation, allows switch between bar or %'%
\noautomath
\usepackage{tabto}
\usepackage{amssymb}
\usepackage{fancyhdr}
\usepackage{setspace}
\usepackage{pifont} %allows dingbats to be called (for the "crosses" and "ticks" defined below)
\usepackage{tipa} % IK


\usepackage{leipzig}%primarily used for the abbreviations

\usepackage[backend=biber,
            style=unified,
            natbib,
            maxcitenames=3,
            maxbibnames=99]{biblatex}
\addbibresource{references.bib}
\usepackage{attrib}%allows authors next to quote environments

\makeatletter
\def\@maketitle{%I guessed from the commenting out of the author below that you don't want an author, this just gets rid of the space associated with the author field
  \newpage
  \null
%  \vskip 2em%
  \begin{center}%
  \let \footnote \thanks
    {\LARGE {\@title}\par}
%    \vskip 1.5em%
%    {\large
%      \lineskip .5em%
%      \begin{tabular}[t]{c}%
%        \@author
%      \end{tabular}\par}%
    \vskip 1em%
    {\large \@date}%
  \end{center}%
  \par
%  \vskip 1.5em
}
\makeatother

\title{LEL2A: Syntax}
%\author{Instructor: Itamar Kastner}
\date{Semester 1, 2024-25}%changed to current academic year

\newcommand*{\sqb}[1]{\lbrack{#1}\rbrack}
\newcommand*{\fn}[1]{\footnote{#1}}
\newcommand{\keyword}[1]{\textsc{#1}}
\newcommand{\cmark}{\ding{51}}
\newcommand{\xmark}{\ding{55}}
\newcommand{\subtitle}[1]{\maketitle\begin{center}{\Large #1}\end{center}}
\makeatletter
\newcommand*{\addFileDependency}[1]{% argument=file name and extension
\typeout{(#1)}% latexmk will find this if $recorder=0
% however, in that case, it will ignore #1 if it is a .aux or 
% .pdf file etc and it exists! If it doesn't exist, it will appear 
% in the list of dependents regardless)
%
% Write the following if you want it to appear in \listfiles 
% --- although not really necessary and latexmk doesn't use this
%
\@addtofilelist{#1}
%
% latexmk will find this message if #1 doesn't exist (yet)
\IfFileExists{#1}{}{\typeout{No file #1.}}
}\makeatother

\newcommand*{\myexternaldocument}[1]{%
\externaldocument{#1}%
\addFileDependency{#1.tex}%
\addFileDependency{#1.aux}%
}
\myexternaldocument{Skills} %also necessary for cross referencing, to reference other documents duplicate with name of document

\begin{document}
\pagestyle{empty}
\maketitle
\subtitle{Tutorial Week 7: Topics 7, 8 \& 9}

\paragraph{Explanation on accumulating Syntax Skills:} Let me stress that the point of the syntax block is to learn how to think like a syntactician, learn about English, and try to have fun. But marks are involved, and here's how we mark Syntax Skills: generally speaking, it's enough to show them once. This means that if we have questions Q1A and Q2B in a given tutorial sheet (for example), both of which target skill 6b, then getting one of them right is enough. That said, it's best to attempt all questions.

Some tutorial questions are primarily \textbf{for discussion}. This means that they can count towards your skill tally, but don't worry if you don't get them right -- their goal is to allow for extended discussion in the tutorial itself and some will go beyond the discussion in class.

(Please remember that we're developing this approach to tutorials and marking together, so your feedback is welcome -- keep it specific and try to give it directly to Itamar.)

\section*{Question 1}%Source: Intro Syntax Tutorial 5, Q1 \& LEL2A Week 5 Tutorial
\hfill{} \ref{np_dp} $\Box$,
\ref{np_structure} $\Box$,
\ref{adjp_pp} $\Box$,
\ref{xp_structure} $\Box$%,

\paragraph{Q1A} In (\ref{good_wine_V}), the predicate \emph{appreciate}, has two arguments, \emph{he} \& \emph{good wine}.
Is the argument \emph{good wine} a complement or adjunct of the verb?
Provide a relevant test to justify your answer.
\paragraph{Q1B} Represent this analysis schematically. You do not need to consider the external argument, \emph{he}, at this stage.

\begin{exe}
    \ex{
    \begin{xlist}
        \ex He appreciates good wine
        \label{good_wine_V}
        \ex His \emph{appreciation of good wine}
        \label{good_wine_N}
        \ex He is \emph{appreciative of good wine}
        \label{good_wine_A}
    \end{xlist}
    }
    \label{good_wine}
\end{exe}

Questions Q1A--Q1B recapped material from previous Topics. If you need them to catch up on some Syntax Skills, note which ones you've just demonstrated.


\paragraph{Q1C \ref{np_dp} $\Box$,
\ref{np_structure} $\Box$} If we assume the same structural relationship as in Q1B for (\ref{good_wine_N}), what changes would you need to make to represent the \emph{italicised} section schematically?

\paragraph{For discussion \ref{adjp_pp} $\Box$,
\ref{xp_structure} $\Box$} If we assume the same structural relationship as in Q1B--Q1C for (\ref{good_wine_A}), what changes would you need to make to represent the \emph{italicised} section schematically?


\section*{Question 2}
\hfill \ref{np_dp} $\Box$,
\ref{np_structure} $\Box$

\paragraph{Q2A} What structure would you assign to the phrases in (\ref{np_xbar})?
Represent the full string given.
Provide relevant tests to justify the choices you have made, such as between representing a string as an adjunct or as a complement. Get at least two out of these three sentences correct. We haven't talked about the structure of proper nouns like \emph{Ipanema} in~(\ref{np_xbar}a) is, so you can propose your own; some discussion can be found in the supplementary readings for chapter 5 of S\&K.
\begin{exe}
    \ex{
    \begin{xlist}
        \ex the girl from Ipanema
        %\ex that new book about the referendum
        \ex the children's teacher's car 
        \ex his utter dependence on alcohol
    \end{xlist} 
    }
  \label{np_xbar}
\end{exe}


\section*{Question 3}%IntroSyntax06-tutorial Q3
\hfill{} 
\ref{xp_structure} $\Box$,
\ref{nonfin_to} $\Box$,
\ref{raising_control} $\Box$%,

We have seen that different instances of \emph{to}, namely prepositional \emph{to} and infinitival \emph{to}, behave differently.
Most of the examples looked at in the course notes were verbal, but we might ask if there are there ``raising adjectives" that would share properties with ``raising verbs" like \emph{seem}. 
\begin{exe}
    \ex[]{
    \begin{xlist}
        \ex[]{Jun is likely to go.}\label{junislikely}
        \ex[]{Jun being the winner is likely to me.}
        \ex[]{Jun is eager to go.}
        \ex[]{Jun is easy to annoy.}\label{juniseasy1}
        \ex[*]{Jun is easy to go.}\label{juniseasy2}
        \ex[]{The range of difficulty is easy to hard.}
        % \ex[]{Jun is fit to burst}
    \end{xlist}
    }
\end{exe}

\paragraph{Q3A} For each of the examples above, is the use of \emph{to} prepositional or infinitival?

We will now consider \emph{likely, eager} and \emph{easy} in turn.

\paragraph{Q3B} Is \emph{likely} a control or a raising predicate? Use at least two diagnostics, and briefly discuss any uncertainties that arise.

\paragraph{For discussion} What about \emph{eager} and \emph{easy}? What properties do they have that are different from both the ``control" and ``raising" predicates we've seen so far?

\end{document}