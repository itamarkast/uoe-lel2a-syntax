\documentclass{beamer}
\usepackage{hyperref} %links and URLS
\usepackage{graphicx} % Required for inserting images
\usepackage[linguistics]{forest} %calls tikz, draws trees
\usepackage{multicol} %adds columns
\usepackage{gb4e} %for formatting examples, works with leipzig and multicol
\primebars %setting for gb4e, adds bars for X-bar notation, allows switch between bar or %'%
\noautomath
\usepackage{tipa} % IK
\usepackage[T1]{fontenc} %Make sure to be able to get accented characters etc
\usepackage[utf8]{inputenc}
\usepackage[normalem]{ulem} %adds strikethrough and other commands
\usepackage{setspace}
\usepackage{pifont} %allows dingbats to be called (for the "crosses" and "ticks" defined below)

\title{LEL2A: Syntax}
%\author{Instructor: Itamar Kastner}
\date{Semester 1, 2024-25}%changed to current academic year


\setbeamertemplate{footline}[frame number]
\setbeamertemplate{blocks}[rounded][shadow=true]
\setbeamertemplate{navigation symbols}{}
\usecolortheme[RGB={0,51,102}]{structure}
\setbeamercolor{alerted text}{fg=red}

\newcommand{\cmark}{\ding{51}}
\newcommand{\xmark}{\ding{55}}
\newcommand\trace{\rule[-0.5ex]{0.5cm}{.4pt}}



\subtitle{Topic 1 class: What is Syntax?}

\begin{document}
\maketitle

\frame{
    \begin{itemize}
        \item Welcome to the syntax block!
        \item Much to discuss: content, skills, assessment.
        \item But first, let's establish some continuity with the morphology block.
    \end{itemize}
}

\frame{
    \url{https://www.facebook.com/RoyalArmouries/videos/serving-curatorial-realness-frmuseum-history-royalarmouries-genz-armsandarmour/717412370571448/}

    \bigskip
    What did you notice about the linguistics of this video?

    \bigskip
    Did you understand everything?
}

\frame{
    \begin{itemize}
        \item Let's talk linguistics!
            \begin{itemize}
                \item Constituency
                \item Hierarchical structure
                \item Syntactic and lexical ambiguity
            \end{itemize}
        \item Join the Wooclap at \textbf{SYN2A}.
        \item What's a constituent?
        % \item Second question: what isn't a constituent?
    \end{itemize}
}

\frame{
    How many constituents are in the following example (including the entire clause)?
    \ea Anna read a book.
    \z

\pause 

    I counted 7:
        \begin{enumerate}
            \item Anna
            \item read
            \item a
            \item book
            \item a book
            \item read a book
            \item Anna read a book
        \end{enumerate}
}

\frame{
    And how many here? Also four words.
    \ea The students love linguistics.
    \z

\pause

    I counted 7 again:
        \begin{enumerate}
            \item The
            \item students
            \item The students
            \item love
            \item linguistics
            \item love linguistics
            \item The students love linguistics
        \end{enumerate}
}
    
\frame{
    What do you think, do these two sentences have the same structure?

    \ea Anna read a book. \hfill (7 constituents)
    \ex The students love linguistics. \hfill (7 constituents)
    \z
}

\frame{
    What do you think, do these two sentences have the same structure?
    
    \ea {[}Anna] [read] [[a] [book]].
    \ex {[[}The] [students]] [love] [linguistics].
    \z
}

\frame{
    \begin{itemize}
        \item We have different ways of representing \emph{structure}.
        \item The two main ones are bracket notation and trees.
        \item For example, we can start with this:
    \end{itemize}

\bigskip
    \begin{multicols}{2}
    {[}Anna] [read] [[a] [book]]

    \begin{forest}
        [{Anna read a book}
            [Anna ]
            [read ]
            [{a book}
                [a ]
                [book ]
            ]
        ]
    \end{forest}
    
    {[[}The] [students]] [love] [linguistics]

    \begin{forest}
        [{The students love linguistics}
            [{The students}
                [The ]
                [students ]
            ]
            [love ]
            [linguistics ]
        ]
    \end{forest}
    \end{multicols}
    
}

\frame{

    \begin{itemize}
        \item Another way to think about constituency: questions.
        \item When thinking about questions in syntax, it's useful to view this as ``before'' and ``after''.
    \end{itemize}
    
    What's a possible rule for forming yes/no questions in English?

    \ea The student \alert{\textbf{is}} happy.
    \ex \alert{\textbf{Is}} the student \trace{} happy?
    \z
}
    % \ea
    %     \ea {[}The child \alert{\textbf{has}} a ball].
    %     \ex Does [the child \alert{\textbf{have}} a ball]?
    %     \z
    % \z


% \frame{
%     What are the \emph{possible} rules for forming wh-questions in English?

%     \ea
%         \ea Who is watching what video?
%         \ex What video are you watching?
%         \ex What video is who watching?
%         \z
%     \z
% }

\frame{
    Now take:
    \ea[]{{[}The student who is in McEwan Hall] is happy.}
    \z

\bigskip
    What's the \emph{grammatical} yes-no question? What's the \emph{ungrammatical} one that conforms to a simple linear rule?

\pause
    \ea
        \ea[*]{Is [the student who \trace{} in McEwan Hall] is happy?}
        \ex[]{Is [the student who is in McEwan Hall] \trace{} happy?}
        \z
    \z    
}

\frame{
    Syntactic ambiguity: how many readings (meanings)? What do they depend on?
    
    \begin{center}
    \includegraphics[width=0.4\textwidth]{Images/memorialbench.jpg}
    \end{center}

    What's an example of lexical ambiguity?    
}

\frame{\frametitle{Summary}

    \begin{block}{Topic 1 skill}
    \begin{itemize}
        \item [1a] Use a hierarchical structure to represent constituency.
    \end{itemize}
    \end{block}

    We also discussed:
    \begin{itemize}
        \item What constituents are.
        \item Hierarchial structure (questions).
        \item Syntactic and lexical ambiguity.
    \end{itemize}

    Research Skills:
    \begin{itemize}
    \item[M1]{State a hypothesis clearly}
    \item[M2]{Present data in support of a hypothesis}
    \item[M3]{Define the conditions under which a hypothesis would be refuted}
    \item[M4]{Present data to refute a hypothesis}
    \item[M5]{Consider the limitations of a diagnostic test when interpreting the results}
    \end{itemize}
}

\frame{\frametitle{Assessment: Skills-based grading}
    You have a total of three attempts to show that you've acquired any of the Syntax Skills: two in Learn exercises and one in the tutorial submission.\\
    (Learning adjustments $\Rightarrow$ Three attempts for Learn exercises)

    \bigskip
    A suggested routine for each Topic (lecture):
    \begin{enumerate}
        \item Watch the video and/or read the lecture notes before the lecture.
        \item Come to the lecture.
        \item Open an exercise and see if you want to give it a go.
        \item Submit the tutorial sheet at the end of the week (before Monday 9am).
        \item Try your first or second Learn attempt once you feel ready (the only deadline is the end of the semester).
    \end{enumerate}

    \bigskip
    \textbf{Office hours:} right after the Tue and Fri lectures, DSB 2.23 (Tue 3-4, Fri 2-3), or whenever is convenient for us.
}

\end{document}