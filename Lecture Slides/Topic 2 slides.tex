\documentclass{beamer}
\usepackage{hyperref} %links and URLS
\usepackage{graphicx} % Required for inserting images
\usepackage[linguistics]{forest} %calls tikz, draws trees
\usepackage{multicol} %adds columns
\usepackage{gb4e} %for formatting examples, works with leipzig and multicol
\primebars %setting for gb4e, adds bars for X-bar notation, allows switch between bar or %'%
\noautomath
\usepackage{tipa} % IK
\usepackage[T1]{fontenc} %Make sure to be able to get accented characters etc
\usepackage[utf8]{inputenc}
\usepackage[normalem]{ulem} %adds strikethrough and other commands
\usepackage{setspace}
\usepackage{pifont} %allows dingbats to be called (for the "crosses" and "ticks" defined below)

\title{LEL2A: Syntax}
%\author{Instructor: Itamar Kastner}
\date{Semester 1, 2025--26}%changed to current academic year


\setbeamertemplate{footline}[frame number]
\setbeamertemplate{blocks}[rounded][shadow=true]
\setbeamertemplate{navigation symbols}{}
\usecolortheme[RGB={0,51,102}]{structure}
\setbeamercolor{alerted text}{fg=red}

\newcommand{\cmark}{\ding{51}}
\newcommand{\xmark}{\ding{55}}
\newcommand\trace{\rule[-0.5ex]{0.5cm}{.4pt}}

\subtitle{Topic 2 Course Notes: Constituency}

\begin{document}
\maketitle

\frame{\frametitle{Recap}
    \begin{block}{Topic 1 skill}
    \begin{itemize}
        \item [1a] Use a hierarchical structure to represent constituency.
    \end{itemize}
    \end{block}

    We also discussed:
    \begin{itemize}
        \item What constituents are.
        \item Hierarchial structure (questions).
        \item Syntactic and lexical ambiguity.
    \end{itemize}
}

\frame{\frametitle{Topic 2}
   \begin{block}{Topic 2 skills}
    \begin{itemize}
        \item[2a] Perform an appropriate constituency test for an NP.
        \item[2b] Perform an appropriate constituency test for a VP.
        \item[2c] Perform an appropriate constituency test for PP/AP/AdvP.
    \end{itemize}
    \end{block}

    Research Skills:
    \begin{itemize}
    \item[M1]{State a hypothesis clearly}
    \item[M2]{Present data in support of a hypothesis}
    \item[M3]{Define the conditions under which a hypothesis would be refuted}
    \item[M4]{Present data to refute a hypothesis}
    \item[M5]{Consider the limitations of a diagnostic test when interpreting the results}
    \end{itemize}
}
    
\frame{
    Constituency tests:
    
    \bigskip
    Wooclap your questions and confusions (\textbf{SYN2A})
}

\frame{\frametitle{Substitution, NP}
    \begin{exe}
    \ex[]{
    \begin{xlist}
        \ex[]{{[}Grandmothers] wear hats.}
        \ex[]{{[}Scottish grandmothers] wear hats.}
        \ex[]{{[}Grandmothers from Ayrshire] wear hats.}
        \ex[]{{[}Scottish grandmothers from Ayrshire] wear hats.}
        \ex[*]{{[}Scottish] wear hats.}
        \ex[*]{{[}From Ayrshire] wear hats}
        \ex[*]{{[}Scottish from Ayrshire] wear hats.}
    \end{xlist}
    }
\end{exe}
}

\frame{\frametitle{Substitution, PP}
    \begin{exe}
    \ex[]{
    \begin{xlist}
        \ex[]{She went \lbrack{}\textsubscript{PP} to the port of Rotterdam\rbrack{}.}
        \ex[]{She went [there].}
        \ex[]{I have never seen that \lbrack{}\textsubscript{PP} in a place like Edinburgh\rbrack{}.}
        \ex[]{I have never seen that [here].}
    \end{xlist}
    }
\end{exe}

    What counts as substitution: stay minimal, \emph{here/there/then}. Resist the temptation of phrases such as \emph{like that}.
}

\frame{\frametitle{Substitution, AdjP}

    \begin{exe}
    \ex[]{Luna seems [ill] and Hermione seems [so], too.}
    \end{exe}

    \begin{exe}
    \ex[]{
    \begin{xlist}
        \ex[]{Harry is [rather weird], and Ron is [so], too.}
        \label{ap_subA}
        \ex[*]{Harry is rather [weird], and Ron is very [so], too.}
        \label{ap_subB}
    \end{xlist}
    }
    \end{exe}
}


\frame{\frametitle{Substitution, VP}
    \begin{exe}
    \ex[]{
    \begin{xlist}
        \ex[]{Mary will \lbrack{}\textsubscript{VP} rent some films tonight\rbrack{} and Bill will \emph{do so} as well.}
        \ex[]{Shane has \lbrack{}\textsubscript{VP} given money to that charity\rbrack{} and Jane has \emph{done so}, too.}
        \ex[]{To \lbrack{}\textsubscript{VP} work in the garden\rbrack{} is a nice thing to do, and to \emph{do so} in the morning is especially relaxing.}
    \end{xlist}
    } \label{vp-do-so}
\end{exe}
}

\frame{
    Substitutions:

    \begin{center}
        \begin{tabular}{ll}
        Constituent type & Substitution \\\hline
        NP & pronoun \\
        PP & \emph{here/there/then} \\
        AdjP & \emph{so (too)} \\
        VP & \emph{do so} \\
        \end{tabular}
    \end{center}
}


\frame{\frametitle{Movement}
    A moved element is a constituent. In questions, the \emph{wh}-questioned element, or the short (``fragment'') answer, is a constituent.

    \bigskip
    Find the constituents, using movement/questions, in:

    \ea The dogs will run around the room happily.
    \z
}


\frame{\frametitle{Clefts}
    In clefts, the clefted element is a constituent.

    \bigskip
    Take a few moments, think of a sentence to test with clefts, and pop it in Wooclap.
}

\frame{
    Beware false negatives!

    \ea[]{\textbf{Cats} are hungry.}
    \ex[]{\textbf{Cats without tails} are hungry.}
    \ex[]{\textbf{They} are hungry.}
    \ex[*]{\textbf{They without tails} are hungry.}
    \z

\pause 

    \ea[]{Who is hungry? (cats/cats without tails)}
    \ex[]{It is cats that are hungry.}
    \ex[]{It is cats without tales that are hungry.}
    \z
}

\frame{\frametitle{Summary}

    \begin{block}{Topic 2 skills}
    \begin{itemize}
        \item[2a] Perform an appropriate constituency test for an NP.
        \item[2b] Perform an appropriate constituency test for a VP.
        \item[2c] Perform an appropriate constituency test for PP/AP/AdvP.
    \end{itemize}
    \end{block}

    We also discussed:
    \begin{itemize}
        \item False negatives.
    \end{itemize}

    Research Skills:
    \begin{itemize}
    \item[M1]{State a hypothesis clearly}
    \item[M2]{Present data in support of a hypothesis}
    \item[M3]{Define the conditions under which a hypothesis would be refuted}
    \item[M4]{Present data to refute a hypothesis}
    \item[M5]{Consider the limitations of a diagnostic test when interpreting the results}
    \end{itemize}
}
\end{document}