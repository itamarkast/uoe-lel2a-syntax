\documentclass{beamer}
\usepackage{hyperref} %links and URLS
\usepackage{graphicx} % Required for inserting images
\usepackage[linguistics]{forest} %calls tikz, draws trees
\usepackage{multicol} %adds columns
\usepackage{gb4e} %for formatting examples, works with leipzig and multicol
\primebars %setting for gb4e, adds bars for X-bar notation, allows switch between bar or %'%
\noautomath
\usepackage{tipa} % IK
\usepackage[T1]{fontenc} %Make sure to be able to get accented characters etc
\usepackage[utf8]{inputenc}
\usepackage[normalem]{ulem} %adds strikethrough and other commands
\usepackage{setspace}
\usepackage{pifont} %allows dingbats to be called (for the "crosses" and "ticks" defined below)

\title{LEL2A: Syntax}
%\author{Instructor: Itamar Kastner}
\date{Semester 1, 2024-25}%changed to current academic year


\setbeamertemplate{footline}[frame number]
\setbeamertemplate{blocks}[rounded][shadow=true]
\setbeamertemplate{navigation symbols}{}
\usecolortheme[RGB={0,51,102}]{structure}
\setbeamercolor{alerted text}{fg=red}

\newcommand{\cmark}{\ding{51}}
\newcommand{\xmark}{\ding{55}}
\newcommand\trace{\rule[-0.5ex]{0.5cm}{.4pt}}

\subtitle{Topic 6 Course Notes: The X-Bar Schema, Part 2}

\begin{document}
\maketitle

\frame{\frametitle{Recap}
      \begin{block}{Topic 5 skills}
    \begin{itemize}
        \item[5a]{Represent heads and phrases, identified via constituency tests, in a manner consistent with the \emph{projection principle}.}
        \item[5b]{Use X-bar representations to distinguish adjunction from complementation.}
    \end{itemize}
    \end{block}

    We also discussed:
    \begin{itemize}
        \item The projection I (for modals and auxiliaries).
        \item Adjunction side (left/right).
        % \item Movement of the subject from Spec,VP to Spec,IP.
    \end{itemize}
}

\frame{\frametitle{Topic 6}
   \begin{block}{Topic 6 skills}
    \begin{itemize}
        \item[6a]{Use constituency tests to identify and justify projection of different functional heads in a representation, such as I, C, and NEG.}
        \item[6b]{Implement a \emph{VP-internal subject hypothesis} analysis.}
    \end{itemize}
    \end{block}

    Research Skills:
    \begin{itemize}
    \item[M1]{State a hypothesis clearly}
    \item[M2]{Present data in support of a hypothesis}
    \item[M3]{Define the conditions under which a hypothesis would be refuted}
    \item[M4]{Present data to refute a hypothesis}
    \item[M5]{Consider the limitations of a diagnostic test when interpreting the results}
    \end{itemize}
}


\frame{
    Let's first consider some ways in which modals/auxiliaries differ from main/lexical verbs.

\bigskip

    Auxiliaries invert with the subject to form a question; lexical verbs don't.

    \ea
        \ea[]{I have seen a dog.}
        \ex[]{Have you \trace{} seen a dog?}
        \ex[*]{Did you have seen a dog?}
        \z
    \ex
        \ea[]{You saw a dog.}
        \ex[*]{Saw you \trace{} a dog?}
        \ex[]{Did you see a dog?}
        \z
    \z
}

\frame{

    Let's try another one. What difference does this minimal pair sow?
    
    \ea[]{Dogs never steal biscuits, but cats \textbf{do}.}
    \ex[]{Dogs have never stolen biscuits, but cats \textbf{have}.}
    % \ex[*]{\trace{}\trace{}\trace{}\trace{}\trace{}\trace{}\trace{}\trace{}\trace{}}
    \z
}

\frame{ \small

    1.~Sentences with auxiliaries show selection by V:\\
            \begin{forest}
                [VP
                    [\textcolor{blue}{NP} [cats, roof]]
                    [V'
                        [V\\\textcolor{blue}{steal}]
                        [\textcolor{blue}{NP} [biscuits, roof]]
                    ]
                ]
            \end{forest}
    
    2.~Sentences with auxiliaries also show agreement with I:\\
        \begin{columns}
            \begin{column}{0.5\textwidth}
                \begin{center}
                \begin{forest}
                [IP
                    [NP [the \textcolor{red}{cats}, roof]]
                    [I'
                        [I\\\textcolor{red}{have} ]
                        [VP [stolen the biscuits, roof]]
                    ]
                ]
                \end{forest}
                \end{center}
            \end{column}
            \begin{column}{0.5\textwidth}
                \begin{forest}
                [IP
                    [NP [the \textcolor{red}{cat}, roof]]
                    [I'
                        [I\\\textcolor{red}{has} ]
                        [VP [stolen the biscuits, roof]]
                    ]
                ]
            \end{forest}
            \end{column}
        \end{columns}
}

\frame{
    The VP-Internal Subject Hypothesis (VPISH):

% \pause
    \begin{enumerate}
        \item The subject starts off as the Specifier of VP (selected by V).
        \item It then \textbf{moves} to the Specifier of IP (agreeing with I).
    \end{enumerate}
}

\frame{ \centering
    \emph{The cats have sneakily stolen the biscuits}:

\pause
    \begin{forest}
    [IP
        [NP [the cats, roof, name=specIP]]
        [I'
            [I\\have]
            [VP
                [NP [\sout{the cats}, roof, name=specVP]]
                [V'
                    [AdvP [sneakily, roof]]
                    [V'
                        [V\\stolen ]
                        [NP [the biscuits, roof]]
                    ]
                ]
            ]
        ]
    ]
    \draw[->] (specVP) to[out=south west,in=south] (specIP);
    \end{forest}
}

\frame{\frametitle{Idioms}
    What's the relevance of these?

    \ea The cat \{is / was / will be / may be\} out of the bag.
    \ex \dots{}
    \z
}

\frame{\frametitle{Quantifiers}
    What's the relevance of these?

    \ea The workers have all left.
    \ex \dots{}
    \z
}

\frame{ \centering \scriptsize
    \emph{We think that [the cats have stealthily stolen the biscuits]}:

\vfill
    \begin{forest}
        [IP [the cats have stealthily \sout{the cats} stolen the biscuits, roof]]
    \end{forest}
}

\frame{ \centering \scriptsize
    \emph{We think that [the cats have stealthily stolen the biscuits]}:

    \begin{forest}
    [CP
        [\phantom{} ]
        [C'
            [C]
            [IP
                [NP [we, roof, name=specIP] ]
                [I'
                    [I]
                    [VP
                        [NP [\sout{we}, roof, name=specVP]]
                        [V'
                            [V\\think]
                            [CP
                                [\phantom{}]
                                [C'
                                    [C\\that]
                                    [IP [the cats have stealthily \sout{the cats} stolen the biscuits, roof]]
                                ]
                            ]
                        ]
                    ]
                ]
            ]
        ]
    ]
    \draw[->] (specVP) to[out=south,in=south] (specIP);
    \end{forest}
}


\frame{
    Our \textbf{clausal spine}:
    
    CP > IP > VP

    Where does negation fit in?

\pause

\bigskip
    \ea I think \uline{that} Cats \textcolor{blue}{might} \textcolor{red}{not} \textbf{like} dogs.
    \z

    \uline{CP} > \textcolor{blue}{IP} > \textcolor{red}{NegP} > \textbf{VP}
}


\frame{\frametitle{Summary}
    \begin{block}{Topic 6 skills}
    \begin{itemize}
        \item[6a]{Use constituency tests to identify and justify projection of different functional heads in a representation, such as I, C, and NEG.}
        \item[6b]{Implement a \emph{VP-internal subject hypothesis} analysis.}
    \end{itemize}
    \end{block}

    We also discussed:
    \begin{itemize}
        \item Differences between auxiliaries and lexical verbs.
    \end{itemize}

    Research Skills:
    \begin{itemize} \small
    \item[M1]{State a hypothesis clearly}
    \item[M2]{Present data in support of a hypothesis}
    \item[M3]{Define the conditions under which a hypothesis would be refuted}
    \item[M4]{Present data to refute a hypothesis}
    \item[M5]{Consider the limitations of a diagnostic test when interpreting the results}
    \end{itemize}
}

\frame{\frametitle{Reminder: Weekly routine}
    \alert{Tutorials on Monday!}

\bigskip
    You have a total of three attempts to show that you've acquired any of the Syntax Skills: two in Learn exercises and one in the tutorial submission.

\bigskip
    A suggested routine for each Topic (lecture):
    \begin{enumerate}
        \item Watch the video and read the lecture notes before the lecture.
        \item Come to the lecture.
        \item Open an exercise and see if you want to give it a go.
        \item \alert{Submit the tutorial sheet at the end of the week (before Monday 9am).}
        \item Try your first or second Learn attempt once you feel ready (the only deadline is the end of the semester).
    \end{enumerate}
}

\frame{\frametitle{The puzzles}
    \begin{itemize}
        \item Puzzle 1
        \item Puzzle 2
        \item Puzzle 3
    \end{itemize}
}

\end{document}