\documentclass{beamer}
\usepackage{hyperref} %links and URLS
\usepackage{graphicx} % Required for inserting images
\usepackage[linguistics]{forest} %calls tikz, draws trees
\usepackage{multicol} %adds columns
\usepackage{gb4e} %for formatting examples, works with leipzig and multicol
\primebars %setting for gb4e, adds bars for X-bar notation, allows switch between bar or %'%
\noautomath
\usepackage{tipa} % IK
\usepackage[T1]{fontenc} %Make sure to be able to get accented characters etc
\usepackage[utf8]{inputenc}
\usepackage[normalem]{ulem} %adds strikethrough and other commands
\usepackage{setspace}
\usepackage{pifont} %allows dingbats to be called (for the "crosses" and "ticks" defined below)

\title{LEL2A: Syntax}
%\author{Instructor: Itamar Kastner}
\date{Semester 1, 2024-25}%changed to current academic year


\setbeamertemplate{footline}[frame number]
\setbeamertemplate{blocks}[rounded][shadow=true]
\setbeamertemplate{navigation symbols}{}
\usecolortheme[RGB={0,51,102}]{structure}
\setbeamercolor{alerted text}{fg=red}

\newcommand{\cmark}{\ding{51}}
\newcommand{\xmark}{\ding{55}}
\newcommand\trace{\rule[-0.5ex]{0.5cm}{.4pt}}

\subtitle{Topic 4 Course Notes: Predicates and arguments part 2}

\begin{document}
\maketitle

\frame{\frametitle{Recap}

    \begin{block}{Topic 3 skills}
    \begin{itemize}
        \item[3a]{Identify divergences in the mapping between semantic and syntactic predicates \& arguments.}
        \item[3b]{Identify arguments that bear semantic roles (such as agent, theme/patient, goal, \& location) and any arguments that do not.}
    \end{itemize}
    \end{block}

    We also discussed:
    \begin{itemize}
        \item Agreement.
        \item Expletive subjects.
    \end{itemize}
}

\frame{ \centering
    \includegraphics[width=0.6\textwidth]{Images/agreement-attraction.jpg}
}

\frame{\frametitle{Topic 4}
   \begin{block}{Topic 4 skills}
    \begin{itemize}
        \item[4a]{Relate differences in types of verbal complements to c-selection properties of the verb.}
        \item[4b]{Perform an appropriate diagnostic to distinguish a complement from an adjunct.}
    \end{itemize}
    \end{block}

    Research Skills:
    \begin{itemize}
    \item[M1]{State a hypothesis clearly}
    \item[M2]{Present data in support of a hypothesis}
    \item[M3]{Define the conditions under which a hypothesis would be refuted}
    \item[M4]{Present data to refute a hypothesis}
    \item[M5]{Consider the limitations of a diagnostic test when interpreting the results}
    \end{itemize}
}

\frame{
    Give an example of an \textbf{elementary tree}.
}

\frame{ \centering
    \begin{forest}
    [PP
        [P\\about ]
        [NP ]
    ]
    \end{forest}
    \begin{forest}
    [NP
        [N\\dogs ]
    ]
    \end{forest}
    
\bigskip
    \begin{forest}
     [PP
        [P\\about ]
        [NP
           [N\\dogs ]
        ]
    ]   
    \end{forest}

\bigskip
    Selection: P selects NP.
}

\frame{
    C-selection vs semantics:

    How do these examples illustrate the difference between selection and the meaning of the verb?

    \begin{exe}
    \ex[]{
    \begin{xlist}
        \ex[]{Leon was eating (Brussels sprouts).}
        \ex[]{Leon is devouring *(Brussels sprouts).}
        \ex[]{Leon was dining (*Brussels sprouts).}
    \end{xlist}
    }
    \end{exe}

\bigskip   
    Any other examples?
}

\frame{

    A syntactic head selects its argument(s).

\bigskip
    What are modifiers? (Or what are they not?)

}


\frame{

    Diagnostics for arguments vs adjuncts:
    \begin{enumerate}
        \item Adjuncts appear outside of \emph{do-so} substitution.
        \item Order: first complements, then adjuncts.
        \item Co-ordination holds only between arguments or between adjuncts.
        \item Preposition stranding can leave the preposition without an argument, but not without a modifier.
    \end{enumerate}

    \ea
        \ea Bears hunt [fish].
        \ex Bears hunt [in the river].
        \ex Bears rely [on fish].
        \z
    \ex Kim walked [to the shop].
    \z
}

% \frame{
%     \textbf{\emph{Do-so} substitution}

% \bigskip
%     Let's come up with some examples.
% }

% \frame{
%     \textbf{Order}

% \bigskip
%     Let's come up with some examples.
% }

% \frame{
%     \textbf{Co-ordination}

% \bigskip
%     Let's come up with some examples.
% }

% \frame{
%     \textbf{Preposition stranding for arguments}

% \bigskip
%     Let's come up with some examples.    
% }

\frame{\frametitle{Summary}
   \begin{block}{Topic 4 skills}
    \begin{itemize}
        \item[4a]{Relate differences in types of verbal complements to c-selection properties of the verb.}
        \item[4b]{Perform an appropriate diagnostic to distinguish a complement from an adjunct.}
    \end{itemize}
    \end{block}

    We also discussed:
    \begin{itemize}
        \item Elementary trees.
    \end{itemize}
    
    Research Skills:
    \begin{itemize}
    \item[M1]{State a hypothesis clearly}
    \item[M2]{Present data in support of a hypothesis}
    \item[M3]{Define the conditions under which a hypothesis would be refuted}
    \item[M4]{Present data to refute a hypothesis}
    \item[M5]{Consider the limitations of a diagnostic test when interpreting the results}
    \end{itemize}
}

\frame{\frametitle{Skills-based grading}
    \begin{enumerate}
        \item Feedback:
            \begin{itemize}
                \item Timely
                \item Actionable
            \end{itemize}
        \item Contract grading:
            \begin{itemize}
                \item Eliminate guesswork around marks
                \item Reduce stress around marks
            \end{itemize}
        \item Learn exercises: individual skills
            \begin{itemize}
                \item Check-in on Friday
            \end{itemize}
        \item Tutorials: synthesizing skills
            \begin{itemize}
                \item Check-in next week
            \end{itemize}
    \end{enumerate}

\bigskip
    \ding{228} Rubric will require 2 fewer skills once updated next week.
}

\end{document}