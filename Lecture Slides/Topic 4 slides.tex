\documentclass{beamer}
\usepackage{hyperref} %links and URLS
\usepackage{graphicx} % Required for inserting images
\usepackage[linguistics]{forest} %calls tikz, draws trees
\usepackage{multicol} %adds columns
\usepackage{gb4e} %for formatting examples, works with leipzig and multicol
\primebars %setting for gb4e, adds bars for X-bar notation, allows switch between bar or %'%
\noautomath
\usepackage{tipa} % IK
\usepackage[T1]{fontenc} %Make sure to be able to get accented characters etc
\usepackage[utf8]{inputenc}
\usepackage[normalem]{ulem} %adds strikethrough and other commands
\usepackage{setspace}
\usepackage{pifont} %allows dingbats to be called (for the "crosses" and "ticks" defined below)

\title{LEL2A: Syntax}
%\author{Instructor: Itamar Kastner}
\date{Semester 1, 2024-25}%changed to current academic year


\setbeamertemplate{footline}[frame number]
\setbeamertemplate{blocks}[rounded][shadow=true]
\setbeamertemplate{navigation symbols}{}
\usecolortheme[RGB={0,51,102}]{structure}
\setbeamercolor{alerted text}{fg=red}

\newcommand{\cmark}{\ding{51}}
\newcommand{\xmark}{\ding{55}}
\newcommand\trace{\rule[-0.5ex]{0.5cm}{.4pt}}

\subtitle{Topic 4: Predicates and arguments part 2}

\begin{document}

\maketitle

\frame{\frametitle{Recap}

    \begin{block}{Topic 3 skills}
    \begin{itemize}
        \item[3a]{Identify divergences in the mapping between semantic and syntactic predicates \& arguments.}
        \item[3b]{Identify arguments that bear semantic roles (such as agent, theme/patient, goal, \& location) and any arguments that do not.}
    \end{itemize}
    \end{block}

    We also discussed:
    \begin{itemize}
        \item Agreement.
        \item Expletive subjects.
    \end{itemize}
}

\frame{\frametitle{Topic 4}
   \begin{block}{Topic 4 skills}
    \begin{itemize}
        \item[4a]{Relate differences in types of verbal complements to c-selection properties of the verb.}
        \item[4b]{Perform an appropriate diagnostic to distinguish a complement from an adjunct.}
    \end{itemize}
    \end{block}

    Research Skills:
    \begin{itemize}
    \item[M1]{State a hypothesis clearly}
    \item[M2]{Present data in support of a hypothesis}
    \item[M3]{Define the conditions under which a hypothesis would be refuted}
    \item[M4]{Present data to refute a hypothesis}
    \item[M5]{Consider the limitations of a diagnostic test when interpreting the results}
    \end{itemize}
}

\frame{ \centering
    \includegraphics[width=0.6\textwidth]{Images/agreement-attraction.jpg}
}

\frame{
    \begin{itemize}
        \item We're going to start developing our \emph{formal theory}.
        \item It might be a bit early to see exactly how this works, but as language scientists we:
            \begin{itemize}
                \item \textbf{Observe} interesting patterns,
                \item \textbf{Generalise} them, i.e.~state them in a way that's true as universally as possible,
                \item and \textbf{analyse} them using a formalism.
            \end{itemize}
    \end{itemize}

\pause
\bigskip
    Example: generalising about Lewis Capaldi.
}

\frame{
    \begin{itemize}
        \item Our goal in this Topic is to understand the relationship between verbs and their arguments.
        \item What can we say about the arguments of \emph{wonder}?
    \end{itemize}

\bigskip
    \ea
        \ea[]{I wonder when Lewis Capaldi will go on tour again.}
        \ex[]{I wonder whether Lewis Capaldi was on \emph{Chicken Shop Date}.}
        \ex[]{I wonder who Lewis Capaldi's favourite writer is.}
        \z
    \ex
        \ea[*]{I wonder Lewis Capaldi's biggest hit.}
        \ex[*]{I wonder Lewis Capaldi's age.}
        \ex[*]{I wonder Lewis Capaldi to the shop.}
        \ex[*]{I wonder Lewis Capaldi pet.}
        \z
    \ex
        \ea[*]{Yesterday wondered when Lewis Capaldi will go on tour again.}
        \ex[*]{{[}That musicians are famous] wondered what Lewis Capaldi's biggest hit is.}
        \z
    \z
}

\frame{
    C-selection and elementary trees:
    \begin{itemize}
        \item The verb \emph{wonder} \textbf{selects for} an NP subject (Agent) and a clause object (Theme).
        \item Not much else matters!
        \item The clause could be future or past, the agent could be singular or plural, etc
        \item So: c-selection is about \textbf{syntactic category} and is \textbf{local}.
    \end{itemize}

    \begin{center}
    \begin{forest}
        [(VP?)
            [NP\\(Agent) ]
            [(VP?)
                [V\\\emph{wonder} ]
                [S\\(Theme)]
            ]
        ]
    \end{forest}
    \end{center}

    (We can express this using bracket notation too, it just gets messy)
}

\frame{ \centering
    These elementary trees, or lexical entries, can then be combined.

\bigskip

    \begin{forest}
    [PP
        [P\\about ]
        [NP ]
    ]
    \end{forest}
    \begin{forest}
    [NP
        [N\\dogs ]
    ]
    \end{forest}
    
\bigskip
    \begin{forest}
     [PP
        [P\\about ]
        [NP
           [N\\dogs ]
        ]
    ]   
    \end{forest}

\bigskip
    Selection: P selects NP.
}

\frame{
    C-selection vs semantics (consolidation exercise):

    How do these examples illustrate the difference between selection and the meaning of the verb?

    \begin{exe}
    \ex[]{
    \begin{xlist}
        \ex[]{Leon was eating (Brussels sprouts).}
        \ex[]{Leon is devouring *(Brussels sprouts).}
        \ex[]{Leon was dining (*Brussels sprouts).}
    \end{xlist}
    }
    \end{exe}

\bigskip   
    Any other examples?
}

\frame{

    A syntactic head selects its argument or arguments. \hfill [4a]

\bigskip
    Now: what are modifiers? (Or what are they not?) \hfill [4b]

\bigskip
\bigskip
\pause
    \centering
    \begin{tabular}{lll}
        & Role  & Formalisation \\\hline
    Obligatory:  &   Argument  & Complement  \\
    Optional:    &   Modifier & Adjunct\\
    \end{tabular}
}

\frame{

    Diagnostics for arguments vs adjuncts:
    \begin{enumerate}
        \item Adjuncts appear outside of \emph{do-so} substitution.
        \item Order: first complements, then adjuncts.
        \item Co-ordination holds only between arguments or between adjuncts.
        \item Preposition stranding can leave the preposition without an argument, but not without a modifier.
    \end{enumerate}

\bigskip
    Try it! Are the phrases in brackets arguments (complements) or modifiers (adjuncts)?
    \ea
        \ea Bears hunt \alert{[fish]} (quickly).
        \ex Bears hunt \alert{[in the river]}.
        \ex Bears rely \alert{[on fish]}.
        \z
    \ex Kim walked \alert{[to the shop]}.
    \z

\bigskip
    In Topic 5 (next) we'll put some order into all these trees, arguments and adjuncts.
}

% \frame{
%     \textbf{\emph{Do-so} substitution}

% \bigskip
%     Let's come up with some examples.
% }

% \frame{
%     \textbf{Order}

% \bigskip
%     Let's come up with some examples.
% }

% \frame{
%     \textbf{Co-ordination}

% \bigskip
%     Let's come up with some examples.
% }

% \frame{
%     \textbf{Preposition stranding for arguments}

% \bigskip
%     Let's come up with some examples.    
% }

\frame{\frametitle{Summary}
   \begin{block}{Topic 4 skills}
    \begin{itemize}
        \item[4a]{Relate differences in types of verbal complements to c-selection properties of the verb.}
        \item[4b]{Perform an appropriate diagnostic to distinguish a complement from an adjunct.}
    \end{itemize}
    \end{block}

    We also discussed:
    \begin{itemize}
        \item Elementary trees.
    \end{itemize}
    
    Research Skills:
    \begin{itemize}
    \item[M1]{State a hypothesis clearly}
    \item[M2]{Present data in support of a hypothesis}
    \item[M3]{Define the conditions under which a hypothesis would be refuted}
    \item[M4]{Present data to refute a hypothesis}
    \item[M5]{Consider the limitations of a diagnostic test when interpreting the results}
    \end{itemize}
}

% \frame{\frametitle{Skills-based grading}
%     \begin{enumerate}
%         \item Feedback is:
%             \begin{itemize}
%                 \item Timely
%                 \item Actionable
%             \end{itemize}
%         \item Contract grading:
%             \begin{itemize}
%                 \item Eliminate guesswork around marks
%                 \item Reduce stress around marks
%             \end{itemize}
%         \item Learn exercises: individual skills
%             \begin{itemize}
%                 \item Check-in on Thursday
%             \end{itemize}
%         \item Tutorials: synthesizing skills
%             \begin{itemize}
%                 \item Check-in next week
%             \end{itemize}
%     \end{enumerate}

% % \bigskip
%     % \ding{228} Rubric will require 2 fewer skills once updated next week.
% }

\end{document}