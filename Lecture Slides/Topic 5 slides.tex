\documentclass{beamer}
\usepackage{hyperref} %links and URLS
\usepackage{graphicx} % Required for inserting images
\usepackage[linguistics]{forest} %calls tikz, draws trees
\usepackage{multicol} %adds columns
\usepackage{gb4e} %for formatting examples, works with leipzig and multicol
\primebars %setting for gb4e, adds bars for X-bar notation, allows switch between bar or %'%
\noautomath
\usepackage{tipa} % IK
\usepackage[T1]{fontenc} %Make sure to be able to get accented characters etc
\usepackage[utf8]{inputenc}
\usepackage[normalem]{ulem} %adds strikethrough and other commands
\usepackage{setspace}
\usepackage{pifont} %allows dingbats to be called (for the "crosses" and "ticks" defined below)

\title{LEL2A: Syntax}
%\author{Instructor: Itamar Kastner}
\date{Semester 1, 2024-25}%changed to current academic year


\setbeamertemplate{footline}[frame number]
\setbeamertemplate{blocks}[rounded][shadow=true]
\setbeamertemplate{navigation symbols}{}
\usecolortheme[RGB={0,51,102}]{structure}
\setbeamercolor{alerted text}{fg=red}

\newcommand{\cmark}{\ding{51}}
\newcommand{\xmark}{\ding{55}}
\newcommand\trace{\rule[-0.5ex]{0.5cm}{.4pt}}

\subtitle{Topic 5: The X-Bar Schema}

\begin{document}
\maketitle

\frame{\frametitle{Recap}
   \begin{block}{Topic 4 skills}
    \begin{itemize}
        \item[4a]{Relate differences in types of verbal complements to c-selection properties of the verb.}
        \item[4b]{Perform an appropriate diagnostic to distinguish a complement from an adjunct.}
    \end{itemize}
    \end{block}

    We also discussed:
    \begin{itemize}
        \item Elementary trees.
    \end{itemize}
}

\frame{\frametitle{Topic 5}
   \begin{block}{Topic 5 skills}
    \begin{itemize}
        \item[5a]{Represent heads and phrases, identified via constituency tests, in a manner consistent with the \emph{projection principle}.}
        \item[5b]{Use X-bar representations to distinguish adjunction from complementation.}
    \end{itemize}
    \end{block}

    Research Skills:
    \begin{itemize}
    \item[M1]{State a hypothesis clearly}
    \item[M2]{Present data in support of a hypothesis}
    \item[M3]{Define the conditions under which a hypothesis would be refuted}
    \item[M4]{Present data to refute a hypothesis}
    \item[M5]{Consider the limitations of a diagnostic test when interpreting the results}
    \end{itemize}
}

\frame{\frametitle{The X-bar schema}
    So far we've established that:
        \begin{itemize}
            \item Predicates (verbs) always take a certain set of arguments (noun phrases, clauses, etc). \hfill [4a]
            \item These arguments have different semantic roles (Agent, Patient, etc). \hfill [3b]
            \item We can also add a potentially unlimited number of modifiers (adverbs, etc). \hfill [4b]
        \end{itemize}

\bigskip
    So: how do we put arguments [5a] and adjuncts [5b] into a precise formal system?

\bigskip
    (Later in this block, and then definitely in LEL2D and Hons courses: what \emph{predictions} does the formal system make?)
}

\frame{\frametitle{The X-bar schema}
    \begin{enumerate}
        \item A given head might select for its argument.
        \item A given head might also need an additional argument, e.g.~the subject/Agent of a verb.
        \item The X-bar schema: head, complement, specifier.
        \item Modifiers extend the projection.
    \end{enumerate}

\bigskip
    Important:
    \begin{itemize}
        \item We've already said all of these things; now we'll get more precise with the terminology and formalism.
        \item Take your time with the formal aspects; they're just one way to represent our understanding and intuitions.
    \end{itemize}
}

\frame{\frametitle{The X-bar schema}
    \begin{itemize}
        \item On the left: the schema (for a verb with subject and object).
        \item Then two examples of it in action.
        \item Pick out the: heads, arguments, complements, specifiers.
    \end{itemize}

\bigskip
    \centering
    \begin{forest} for tree={calign=fixed edge angles},
        [\iibar{V}
        [\iibar{N}\\Subject] [\ibar{V}
        [\obar{V}] [\iibar{N}\\Object]]
        ]
    \end{forest}
    \begin{forest} for tree={calign=fixed edge angles},
        [\iibar{V}
        [\iibar{N}\\Artists] [\ibar{V}
        [\obar{V}\\hate] [\iibar{N}\\Spotify]]
        ]
    \end{forest}
    \begin{forest} for tree={calign=fixed edge angles},
        [\iibar{V}
        [\iibar{N}\\Spotify] [\ibar{V}
        [\obar{V}\\depends] [\iibar{P}\\{on artists}]]
        ]
    \end{forest}
}

\frame{\frametitle{The X-bar schema}
\centering
    More general form:
    
\bigskip
    \begin{forest} for tree={calign=fixed edge angles},
        [\iibar{V}
        [specifier] [\ibar{V}
        [\obar{V}] [complement]]
        ]
    \end{forest}
}

\frame{\frametitle{The X-bar schema}
\centering

    Even more general form:
    
\bigskip
    \begin{forest} for tree={calign=fixed edge angles},
        [\iibar{X}
        [specifier] [\ibar{X}
        [\obar{X}] [complement]]
        ]
    \end{forest}
}

% \frame{
%     The specifier and complement are both phrases.
    
%     \bigskip
%     Why?
% }

\frame{
    The specifier and complement are both phrases.
    
\bigskip
    Why?

\bigskip
    Because we think in terms of \textbf{constituents}, satisfying the projection principle.
}

\frame{

    Is this a good X-bar representation?

    \begin{center}
        \begin{forest}
            [VP
                [NP
                    [\ibar{N} [N \\dugs ]]
                ]
                [V'
                    [V\\like ]
                    [NP
                        [\ibar{N} [N\\claps ]]
                    ]
                ]
            ]
        \end{forest}
    \end{center}
}

\frame{

    % Is this a good X-bar representation?

    \begin{center}
        \begin{forest}
            [VP
                [NP
                    [\ibar{N} [N \\dugs ]]
                ]
                [V' 
                    [V\\like ]
                    [NP
                        [\ibar{N} [N\\claps ]]
                    ]
                ]
            ]
        \end{forest}
        {\Large =}
        \begin{forest}
            [VP
                [NP [dogs,roof]]
                [V' 
                    [V\\like ]
                    [NP [pats,roof]]
                ]
            ]
        \end{forest}
        {\Large =}
        \begin{forest}
            [VP [dogs like pats,roof]]
        \end{forest}
    \end{center}
}

\frame{
 Is this a good X-bar representation?

    \begin{center}
        \begin{forest}
            [PP
                [P\\in]
                [N\\Edinburgh]
            ]
        \end{forest}
    \end{center}
}

\frame{
 More accurate:

    \begin{center}
        \begin{forest}
            [PP
                [P\\in]
                [\alert{NP}\\Edinburgh]
            ]
        \end{forest}
    \end{center}
}

% \frame{
%     Any other suggested phrases/trees?
% }

\frame{
    \begin{itemize}
        \item These are technical points: ``do things this way''.
        \item \textbf{But} they're supposed to show our understanding of syntactic relationships.
        \item Try to keep both of these aspects in mind.
    \end{itemize}

\bigskip
    Some things we haven't talked about yet: adjectives, adverbs, more kinds of verbs, \dots{}
}

\frame{
    Let's talk about adverbs.

\bigskip
    Adjectives will wait a bit till we develop the structure of NPs a bit more.

\bigskip
    Recap: what's the difference between adverbs and arguments?

% \pause

    \begin{itemize}
        \item Optional.
        \item Pattern differently (we have diagnostics for them).
    \end{itemize}
}

\frame{
    Modifiers in our theory: \textbf{adjunction} to the \textbf{bar-level}.

    \ea Dogs really like pats.
    \ex Students draw trees diligently.
    \z
}

\frame{
    And there are a couple of twists in the following examples (see the lecture notes).

    Auxiliary verb:
    \ea I might draw trees slowly.
    \z
\bigskip

    Where does each of the two modifiers adjoin?
    \ea Dogs really like pats behind the ear.
    \z
}


\frame{\frametitle{Summary}
   \begin{block}{Topic 5 skills}
    \begin{itemize}
        \item[5a]{Represent heads and phrases, identified via constituency tests, in a manner consistent with the \emph{projection principle}.}
        \item[5b]{Use X-bar representations to distinguish adjunction from complementation.}
    \end{itemize}
    \end{block}

    We also discussed:
    \begin{itemize}
        \item The projection I (for modals and auxiliaries).
        \item Adjunction side (left/right).
        % \item Movement of the subject from Spec,VP to Spec,IP.
    \end{itemize}

    Research Skills:
    \begin{itemize} \small
    \item[M1]{State a hypothesis clearly}
    \item[M2]{Present data in support of a hypothesis}
    \item[M3]{Define the conditions under which a hypothesis would be refuted}
    \item[M4]{Present data to refute a hypothesis}
    \item[M5]{Consider the limitations of a diagnostic test when interpreting the results}
    \end{itemize}

}
\end{document}