\documentclass{beamer}
\usepackage{hyperref} %links and URLS
\usepackage{graphicx} % Required for inserting images
\usepackage[linguistics]{forest} %calls tikz, draws trees
\usepackage{multicol} %adds columns
\usepackage{gb4e} %for formatting examples, works with leipzig and multicol
\primebars %setting for gb4e, adds bars for X-bar notation, allows switch between bar or %'%
\noautomath
\usepackage{tipa} % IK
\usepackage[T1]{fontenc} %Make sure to be able to get accented characters etc
\usepackage[utf8]{inputenc}
\usepackage[normalem]{ulem} %adds strikethrough and other commands
\usepackage{setspace}
\usepackage{pifont} %allows dingbats to be called (for the "crosses" and "ticks" defined below)

\title{LEL2A: Syntax}
%\author{Instructor: Itamar Kastner}
\date{Semester 1, 2025--26}%changed to current academic year


\setbeamertemplate{footline}[frame number]
\setbeamertemplate{blocks}[rounded][shadow=true]
\setbeamertemplate{navigation symbols}{}
\usecolortheme[RGB={0,51,102}]{structure}
\setbeamercolor{alerted text}{fg=red}

\newcommand{\cmark}{\ding{51}}
\newcommand{\xmark}{\ding{55}}
\newcommand\trace{\rule[-0.5ex]{0.5cm}{.4pt}}

\subtitle{Topic 3 Course Notes: Predicates and arguments part 1}

\begin{document}
\maketitle

\frame{\frametitle{Recap}
       \begin{block}{Topic 2 skills}
    \begin{itemize}
        \item[2a] Perform an appropriate constituency test for an NP.
        \item[2b] Perform an appropriate constituency test for a VP.
        \item[2c] Perform an appropriate constituency test for PP/AP/AdvP.
    \end{itemize}
    \end{block}

    We also discussed:
    \begin{itemize}
        \item False negatives.
    \end{itemize}
}

\frame{\frametitle{Topic 3}
   \begin{block}{Topic 3 skills}
    \begin{itemize}
        \item[3a]{Identify divergences in the mapping between semantic and syntactic predicates \& arguments.}
        \item[3b]{Identify arguments that bear semantic roles (such as agent, theme/patient, goal, \& location) and any arguments that do not.}
    \end{itemize}
    \end{block}

    Research Skills:
    \begin{itemize}
    \item[M1]{State a hypothesis clearly}
    \item[M2]{Present data in support of a hypothesis}
    \item[M3]{Define the conditions under which a hypothesis would be refuted}
    \item[M4]{Present data to refute a hypothesis}
    \item[M5]{Consider the limitations of a diagnostic test when interpreting the results}
    \end{itemize}
}
 

\frame{

    Give some examples of:
    
    \begin{enumerate}
        \item One-place predicate
        \item Two-place predicate
        \item Three-place predicate
        \item Zero-place predicate
    \end{enumerate}
}

\frame{
    How do we identify the subject?

    \bigskip
    Agreement:
    \ea 
        \ea The children are(/*is) dancing.
        \ex The key to the offices is(/*are) on the table.
        \ex It is(/*are) raining buckets.
        \z
    \z
}

\frame{
    Some basic mappings:

\bigskip
    \begin{center}
    \begin{tabular}{ccc}
        Subject & $\Leftrightarrow$ & Agent\\
        Object & $\Leftrightarrow$ & Theme/Patient\\
        Indirect Object & $\Leftrightarrow$ & Goal\\
    \end{tabular}
    \end{center}

    \bigskip
    What are some violations of these mappings?
}

\frame{
    \begin{itemize}
    \item Some verbs require objects:
        \begin{itemize}
            \item \emph{devour}
            \item \emph{remove}
        \end{itemize}
    \item Some let you drop them:
        \begin{itemize}
            \item \emph{eat}
            \item \emph{know}
        \end{itemize}
    \item Some never have them:
        \begin{itemize}
            \item \emph{appear}
            \item \emph{sleep}
        \end{itemize}
    \end{itemize}
    
    \bigskip
    (And specialist vocabulary: \emph{Do you even lift?})
}


\frame{
    What about subjects? What about agents?

    \ea
        \ea[]{\textbf{It} is my favourite topic.}
        \ex[]{\textbf{What} is my favourite topic?}
        \z
    \ex
        \ea[]{\textbf{It} is obvious that subjects are special.}
        \ex[*]{\textbf{What} is obvious that subject are special?}
        \z
    \z

\bigskip
    \textbf{There are} other ``expletive'' or ``dummy'' subject in English.
}

\frame{\frametitle{Summary}

    \begin{block}{Topic 3 skills}
    \begin{itemize}
        \item[3a]{Identify divergences in the mapping between semantic and syntactic predicates \& arguments.}
        \item[3b]{Identify arguments that bear semantic roles (such as agent, theme/patient, goal, \& location) and any arguments that do not.}
    \end{itemize}
    \end{block}

    We also discussed:
    \begin{itemize}
        \item Agreement.
        \item Expletive subjects.
    \end{itemize}
    
    Research Skills:
    \begin{itemize}
    \item[M1]{State a hypothesis clearly}
    \item[M2]{Present data in support of a hypothesis}
    \item[M3]{Define the conditions under which a hypothesis would be refuted}
    \item[M4]{Present data to refute a hypothesis}
    \item[M5]{Consider the limitations of a diagnostic test when interpreting the results}
    \end{itemize}
}

\frame{\frametitle{Reminder: Weekly routine}
    \alert{Tutorials on Monday!}

\bigskip
    You have a total of three attempts to show that you've acquired any of the Syntax Skills: two in Learn exercises and one in the tutorial submission.

\bigskip
    A suggested routine for each Topic (lecture):
    \begin{enumerate}
        \item Watch the video and read the lecture notes before the lecture.
        \item Come to the lecture.
        \item Open an exercise and see if you want to give it a go.
        \item \alert{Submit the tutorial sheet at the end of the week (before Monday 9am).}
        \item Try your first or second Learn attempt once you feel ready (the only deadline is the end of the semester).
    \end{enumerate}
}

\frame{\frametitle{The puzzles}
    \begin{itemize}
        \item Puzzle 1
        \item Puzzle 2
        \item Puzzle 3
    \end{itemize}
}

\end{document}