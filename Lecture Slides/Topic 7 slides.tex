\documentclass{beamer}
\usepackage{hyperref} %links and URLS
\usepackage{graphicx} % Required for inserting images
\usepackage[linguistics]{forest} %calls tikz, draws trees
\usepackage{multicol} %adds columns
\usepackage{gb4e} %for formatting examples, works with leipzig and multicol
\primebars %setting for gb4e, adds bars for X-bar notation, allows switch between bar or %'%
\noautomath
\usepackage{tipa} % IK
\usepackage[T1]{fontenc} %Make sure to be able to get accented characters etc
\usepackage[utf8]{inputenc}
\usepackage[normalem]{ulem} %adds strikethrough and other commands
\usepackage{setspace}
\usepackage{pifont} %allows dingbats to be called (for the "crosses" and "ticks" defined below)

\title{LEL2A: Syntax}
%\author{Instructor: Itamar Kastner}
\date{Semester 1, 2025--26}%changed to current academic year


\setbeamertemplate{footline}[frame number]
\setbeamertemplate{blocks}[rounded][shadow=true]
\setbeamertemplate{navigation symbols}{}
\usecolortheme[RGB={0,51,102}]{structure}
\setbeamercolor{alerted text}{fg=red}

\newcommand{\cmark}{\ding{51}}
\newcommand{\xmark}{\ding{55}}
\newcommand\trace{\rule[-0.5ex]{0.5cm}{.4pt}}

\subtitle{Topic 7: Nonverbal XPs (DP)}

\begin{document}
\maketitle

\frame{\frametitle{Recap}
      \begin{block}{Topic 6 skills}
    \begin{itemize}
        \item[6a]{Use constituency tests to identify and justify projection of different functional heads in a representation, such as I, C, and NEG.}
        \item[6b]{Implement a \emph{VP-internal subject hypothesis} analysis.}
    \end{itemize}
    \end{block}
}

\frame{\frametitle{Topic 7}
   \begin{block}{Topic 7 skills}
    \begin{itemize}
        \item[7a]{Implement an analysis of complex noun phrases that is consistent with the framework developed.}
        \item[7b]{Identify elements as complements or adjuncts of a noun in a principled way.}
    \end{itemize}
    \end{block}

    Research Skills:
    \begin{itemize}
    \item[M1]{State a hypothesis clearly}
    \item[M2]{Present data in support of a hypothesis}
    \item[M3]{Define the conditions under which a hypothesis would be refuted}
    \item[M4]{Present data to refute a hypothesis}
    \item[M5]{Consider the limitations of a diagnostic test when interpreting the results}
    \end{itemize}
}

\frame{

    So far:
    \begin{enumerate}
        \item Verbs are the lexical core of a clause.
        \item But sometimes there are other morphemes around them them, like auxiliaries and agreement.
        \item So we've hypothesised the head (projection) \obar{I}.
    \end{enumerate}
    
    \begin{center}
    \begin{forest}
        [IP
            [\phantom{a} ]
            [\ibar{I}
                [\obar{I} ]
                [VP ]
            ]
        ]
    \end{forest}
    \end{center}
    
\bigskip
    Today:
    \begin{itemize}
        \item The same for nouns. \hfill [7a]
        \item And then we can make sense of the difference between arguments and adjuncts in the same way. \hfill [7b]
    \end{itemize}
}

\frame{\frametitle{The DP hypothesis}

    \begin{center}
    \begin{forest}
        [DP
            [\phantom{a} ]
            [\ibar{D}
                [\obar{D} ]
                [NP ]
            ]
        ]
    \end{forest}
    \end{center}

\bigskip
    (We aren't going to motivate this too much --- see the course notes)
}


\frame{
    \centering
 \begin{forest}
    [\iibar{I}
        [\dots{} ]
        [\ibar{I}
            [\obar{I} ]
            [VP
                [\dots{} ]
                [\ibar{V}
                    [\obar{V} ]
                    [{NP/PP/CP\\\dots{}} ]
                ]
            ]
        ]
    ]
    \end{forest}
    \begin{forest}
    [\iibar{D}
        [\dots{} ]
        [\ibar{D}
            [\obar{D} ]
            [NP
                [\dots{} ]
                [\ibar{N}
                    [\obar{N} ]
                    [{NP/PP/CP\\\dots{}} ]
                ]
            ]
        ]
    ]
    \end{forest}
}


\frame{
    Let's list some things worth pointing out about examples like the following:

    \ea
        \ea[]{Taylor \alert{considered} her options.}
        \ex[]{Taylor \alert{considered that} her fans would be upset.}
        \ex[]{Taylor \textcolor{blue}{relied on} her fans.}
        \ex[*]{Taylor \textcolor{blue}{relied at} her audience.}
        \z
\bigskip
    \ex
        \ea[]{Taylor's \alert{consideration} of her options.}
        \ex[]{Taylor expected \alert{consideration} of her options.}
        \ex[?]{Taylor's \alert{consideration that} her fans would be upset.}
        \ex[]{The \alert{consideration} of her options annoyed her.}
        \ex[]{Taylor's \textcolor{blue}{reliance on} her fans.}
        \ex[*]{Taylor's \textcolor{blue}{reliance at} her audience.}
        \z
    \z
}

\frame{
    \begin{itemize}
        \item Parallels between verbal and nominal versions (c-selection).
        \item No subject requirement in nominals (no expletive subjects needed).
        \item The definite article and the possessor are in complementary distribution.
    \end{itemize}
}

\frame{
    What's the difference between Taylor's reliance and Taylor's hat?

    \ea Taylor's reliance on her fans.
    \ex Taylor's hat on the hanger.
    \z

\bigskip
    Which is the argument and which is the adjunct?

\bigskip
    How can we express this in our structures?
}

\frame{
    \begin{columns}
        \begin{column}{0.5\textwidth}
        \begin{center}
            \begin{forest}
                [\iibar{D} 
                    [\iibar{D} [Taylor, roof]]
                    [\ibar{D}
                        [\obar{D}\\'s]
                        [\iibar{N}
                            [\iibar{D} [\sout{Taylor}, roof]]
                            [\ibar{N}
                                [\obar{N}\\reliance ]
                                [\iibar{P} [on her fans, roof]]
                            ]
                        ]
                    ]
                ]
            \end{forest}
        \end{center}
        \end{column}
        \begin{column}{0.5\textwidth}
        \begin{center}
            \begin{forest}
                [\iibar{D}
                    [\iibar{D} [Taylor, roof]]
                    [\ibar{D}
                        [\obar{D}\\'s]
                        [\iibar{N}
                            [\ibar{N}
                                [\ibar{N}
                                    [N\\hat]
                                ]
                                [\iibar{P} [on the hanger, roof]]
                            ]
                        ]
                    ]
                ]
            \end{forest}
        \end{center}
        \end{column}
    \end{columns}
}

\frame{
    Which of these two? What's at stake?

    \begin{columns}
        \begin{column}{0.5\textwidth}
        \begin{center}
            \begin{forest}
                [\iibar{D}
                    [\obar{D}\\the ]
                    [\iibar{N}
                        [\ibar{N}
                            [\ibar{N}
                                [\obar{N}\\day]
                            ]
                            [\iibar{P}
                                [\ibar{P}
                                    [\obar{P}\\ before ]
                                    [AdvP [yesterday, roof]]
                                ]
                            ]
                        ]
                    ]
                ]
            \end{forest}
        \end{center}
        \end{column}
        \begin{column}{0.5\textwidth}
        \begin{center}
            \begin{forest}
                [\iibar{D}
                    [\ibar{D}
                        [\obar{D}\\the]
                        [\iibar{N}
                            [\ibar{N} [\obar{N}\\day]
                            ]
                        ]
                    ]
                    [\iibar{P}
                        [\ibar{P}
                            [\obar{P}\\ before ]
                            [AdvP [yesterday, roof]]
                        ]
                    ]
                ]
            \end{forest}
        \end{center}
        \end{column}
    \end{columns}
}

\frame{
    Let's think through constituency:
    
    \ea
        \ea[]{The day before yesterday}
        \ex[]{The one before yesterday}
        \ex[]{The day before yesterday and the one next week}
        \z
    \z

    \ea
        \ea[]{It was the day before yesterday that I remember.}
        \ex[*]{It was the day that I remember before yesterday.}
        \z
    \z
}

\frame{
    Which of these two? What's at stake?

    \begin{columns}
        \begin{column}{0.5\textwidth}
        \begin{center}
            \begin{forest}
                [\iibar{D}
                    [\obar{D}\\the ]
                    [\iibar{N}
                        [\ibar{N}
                            [\ibar{N}
                                [\obar{N}\\day]
                            ]
                            [\iibar{P}
                                [\ibar{P}
                                    [\obar{P}\\ before ]
                                    [AdvP [yesterday, roof]]
                                ]
                            ]
                        ]
                    ]
                ]
            \end{forest}
        \end{center}
        \end{column}
        \begin{column}{0.5\textwidth}
        \begin{center}
            \begin{forest}
                [\iibar{D}
                    [\ibar{D}
                        [\obar{D}\\the]
                        [\iibar{N}
                            [\ibar{N} [\obar{N}\\day]
                            ]
                        ]
                    ]
                    [\iibar{P}
                        [\ibar{P}
                            [\obar{P}\\ before ]
                            [AdvP [yesterday, roof]]
                        ]
                    ]
                ]
            \end{forest}
        \end{center}
        \end{column}
    \end{columns}
}

\frame{
    Constituency tests identify \emph{day before yesterday} as a constituent: we adjoin the modifier PP to the \ibar{N} level, then combine D with NP.

    \begin{columns}
        \begin{column}{0.5\textwidth}
        \begin{center}
            \begin{forest}
                [\iibar{D}
                    [\textcolor{teal}{\obar{D}}\\\textcolor{teal}{the} ]
                    [\textcolor{red}{\iibar{N}}
                        [\ibar{N}
                            [\ibar{N}
                                [\obar{N}\\\textcolor{blue}{day}]
                            ]
                            [\textcolor{red}{\iibar{P}}
                                [\ibar{P}
                                    [\obar{P}\\ \textcolor{blue}{before} ]
                                    [AdvP [\textcolor{blue}{yesterday}, roof]]
                                ]
                            ]
                        ]
                    ]
                ]
            \end{forest}
        \end{center}
        \end{column}
        \begin{column}{0.5\textwidth}
        \begin{center}
            \begin{forest}
                [\iibar{D}
                    [\ibar{D}
                        [\obar{D}\\the]
                        [\iibar{N}
                            [\ibar{N} [\obar{N}\\day]
                            ]
                        ]
                    ]
                    [\iibar{P}
                        [\ibar{P}
                            [\obar{P}\\ before ]
                            [AdvP [yesterday, roof]]
                        ]
                    ]
                ]
            \end{forest}
        \end{center}
        \end{column}
    \end{columns}
}


% \frame{
%     Argument vs adjunct again:

%     \ea 
%         \ea[]{The professor is proud.}
%         \ex[]{The professor is proud by nature.}
%         \ex[]{The professor is proud of her students.}
%         \z
%     \z

% \bigskip
%     What can and can't we co-ordinate?
%     \ea
%         \ea[]{The professor is proud by nature and \trace{}\trace{}\trace{}.}
%         \ex[*]{The professor is proud by nature and \trace{}\trace{}\trace{}.}
%         \ex[]{The professor is proud of her students and \trace{}\trace{}\trace{}.}
%         \ex[*]{The professor is proud of her students and \trace{}\trace{}\trace{}.}
%         \z
%     \z
% }

\frame{\frametitle{Noun complement clauses}

    \ea
        \ea{I confessed [that I listen to K-pop].}
        \ex{My confession [that I listen to K-pop].}
        \z
    \ex{The hat [that you gave me \trace{} yesterday].}
    \z

    \begin{center}
    \begin{tabular}{l|c|c}
        & Noun complement clause & Relative clause\\\hline
    Argument/modifier? & & \\\hline
    Gap  &   &  \\
    \emph{that/which} & & \\
    \end{tabular}
    \end{center}
}

\frame{
    \centering
    \begin{tabular}{l|c|c}
        & Noun complement clause & Relative clause\\\hline
    Argument/modifier? & Argument & Modifier \\\hline
    Gap  & No  & Yes \\
    \emph{that/which} & \emph{that} & either one \\
    \end{tabular}
}

\frame{\frametitle{Summary}
   \begin{block}{Topic 7 skills}
    \begin{itemize}
        \item[7a]{Implement an analysis of complex noun phrases that is consistent with the framework developed.}
        \item[7b]{Identify elements as complements or adjuncts of a noun in a principled way.}
    \end{itemize}
    \end{block}

    We also discussed:
    \begin{itemize}
        \item Complement clauses vs relative clauses.
        \item Constituency within the DP.
    \end{itemize}

    Research Skills:
    \begin{itemize} \small
    \item[M1]{State a hypothesis clearly}
    \item[M2]{Present data in support of a hypothesis}
    \item[M3]{Define the conditions under which a hypothesis would be refuted}
    \item[M4]{Present data to refute a hypothesis}
    \item[M5]{Consider the limitations of a diagnostic test when interpreting the results}
    \end{itemize}
}
\end{document}