\documentclass{beamer}
\usepackage{hyperref} %links and URLS
\usepackage{graphicx} % Required for inserting images
\usepackage[linguistics]{forest} %calls tikz, draws trees
\usepackage{multicol} %adds columns
\usepackage{gb4e} %for formatting examples, works with leipzig and multicol
\primebars %setting for gb4e, adds bars for X-bar notation, allows switch between bar or %'%
\noautomath
\usepackage{tipa} % IK
\usepackage[T1]{fontenc} %Make sure to be able to get accented characters etc
\usepackage[utf8]{inputenc}
\usepackage[normalem]{ulem} %adds strikethrough and other commands
\usepackage{setspace}
\usepackage{pifont} %allows dingbats to be called (for the "crosses" and "ticks" defined below)

\title{LEL2A: Syntax}
%\author{Instructor: Itamar Kastner}
\date{Semester 1, 2024-25}%changed to current academic year


\setbeamertemplate{footline}[frame number]
\setbeamertemplate{blocks}[rounded][shadow=true]
\setbeamertemplate{navigation symbols}{}
\usecolortheme[RGB={0,51,102}]{structure}
\setbeamercolor{alerted text}{fg=red}

\newcommand{\cmark}{\ding{51}}
\newcommand{\xmark}{\ding{55}}
\newcommand\trace{\rule[-0.5ex]{0.5cm}{.4pt}}

\subtitle{Topic 9 Course Notes: Nonfinite clauses}

\begin{document}
\maketitle

\frame{\frametitle{Recap}
   \begin{block}{Topic 7 skills}
    \begin{itemize}
        \item[7a]{Implement an analysis of complex noun phrases that is consistent with the framework developed.}
        \item[7b]{Identify elements as complements or adjuncts of a noun in a principled way.}
    \end{itemize}
    \end{block}

   \begin{block}{Topic 8 skills}
    \begin{itemize}
        \item[8a]{\sout{Implement an analysis of complex noun phrases that is consistent with the framework developed.}}
        \item[8b]{\sout{Identify elements as complements or adjuncts of a noun in a principled way.}}
    \end{itemize}
    \end{block}
}

\frame{\frametitle{Topic 9}
     \begin{block}{Topic 9 skills}
    \begin{itemize}
        \item[9a]{Identify non-finite clauses headed by infinitival \emph{to} in a principled way.}
        \item[9b]{Distinguish raising structures from control structures on the basis of $\theta$-role assignment.}
    \end{itemize}
    \end{block}

    Research Skills:
    \begin{itemize}
    \item[M1]{State a hypothesis clearly}
    \item[M2]{Present data in support of a hypothesis}
    \item[M3]{Define the conditions under which a hypothesis would be refuted}
    \item[M4]{Present data to refute a hypothesis}
    \item[M5]{Consider the limitations of a diagnostic test when interpreting the results}
    \end{itemize}
}

\frame{
    Which of these \emph{to}-s is a preposition and which is a non-finite marker?
    
    \ea \textbf{To} resort \textbf{to} contrived examples \textbf{to} give \textbf{to} students \textbf{to} analyse brings me \textbf{to} tears.
    \z
}

\frame{
    Which of these \emph{to}-s is a preposition and which is a \alert{non-finite marker}?
    
    \ea \alert{\textbf{To}} resort \textbf{to} contrived examples \alert{\textbf{to}} give \textbf{to} students \alert{\textbf{to}} analyse brings me \textbf{to} tears.
    \z
}

\frame{
Infinitival vs preposition \emph{to}:
    \begin{itemize}
        \item Followed by the non-finite form of the verb or by a DP?
        \item The preposition \emph{to} can be modified by \emph{right}, \emph{straight}, etc.
        \item Moving the VP complement to I is possible, but not the DP complement to P in an adjunct.
    \end{itemize}
}

\frame{
    What are the arguments of the different verbs?

    \ea They wanted to analyse the sentence.
    \ex We seem to have solved the problem.
    \z
}

\frame{
    What are the arguments of the different verbs?

    \ea \uline{They} \alert{wanted} [to \textcolor{blue}{analyse} the sentence]. \hfill [Control]
    \ex We \alert{seem} [to have \textcolor{blue}{solved} the problem]. \hfill [Raising]
    \z
}

\frame{ \centering
    \begin{tabular}{c|c|c}
        &   Raising & Control\\\hline
    Expletive subjects & & \\
    Idiomatic meaning & & \\
    Agent-oriented adverbs & & \\
    \end{tabular}
}

\frame{
    \ea
        \ea[]{Many linguists seem to be in the room.}
        \ex[]{There seem to be many linguists in the room.}
        \z
    \ex
        \ea[]{Many linguists tried to solve the problem.}
        \ex[*]{There tried many linguists to solve the problem.}
        \z
    \z
}

\frame{ \tiny
\begin{multicols}{2}
Control:\\
\begin{forest}
baseline,
    for tree={parent anchor=south,child anchor=north,l=7ex,s sep=10pt},
    [\iibar{I}
        [\iibar{D}$_i$ [\ibar{D} [\obar{D}\\they, name=copy1]]][\ibar{I}
        [\obar{I}\\\lbrack{}\textsc{past}\rbrack{}][\iibar{V}
        [$\langle$\sout{\iibar{D}}$\rangle$ [$\langle$\sout{they}$\rangle$, roof, name=trace1]][\ibar{V}
        [\obar{V}\\tried][\iibar{I}
        [\iibar{D}$_i$ [\ibar{D} [\obar{D}\\\textsc{pro}, name=copy]]][\ibar{I}
        [\obar{I}\\to][\iibar{V}
        [$\langle$\sout{\iibar{D}$_i$}$\rangle$ [$\langle$\sout{\textsc{pro}}$\rangle$, roof, name=trace]][\ibar{V}
        [\obar{V}\\complete][\iibar{D} [this task, roof]]]]]]]]]
        ]
        \draw[->,dotted] (trace) to[out=south west,in=south] (copy);
        \draw[->,dotted] (trace1) to[out=south west,in=south] (copy1);
\end{forest}
Raising:\\
\begin{forest}
baseline,
    for tree={parent anchor=south,child anchor=north,l=7ex,s sep=10pt},
    [\iibar{I}
        [\iibar{D} [\ibar{D} [\obar{D}\\you, name=copy]]][\ibar{I}
        [\obar{I}\\\lbrack{}\textsc{pres}\rbrack{}][\iibar{V}
        [\ibar{V}
        [\obar{V}\\seem][\iibar{I}
        [$\langle$\sout{\iibar{D}}$\rangle$ [$\langle$\sout{you}$\rangle$, roof, name=trace1]][\ibar{I}
        [\obar{I}\\to][\iibar{V}
        [$\langle$\sout{\iibar{D}}$\rangle$ [$\langle$\sout{you}$\rangle$, roof, name=trace]][\ibar{V}
        [\obar{V}\\enjoy][\iibar{D} [old movies, roof]]]]]]]]]
    ]
        \draw[->,dotted] (trace) to[out=south west,in=south] (trace1);
        \draw[->,dotted] (trace1) to[out=south west,in=south] (copy);
\end{forest}
\end{multicols}
}

\frame{ \centering
    \begin{tabular}{c|c|c}
        &   Raising & Control\\\hline
    Expletive subjects & \cmark{} & \xmark{} \\
    Idiomatic meaning & & \\
    Agent-oriented adverbs & & \\
    \end{tabular}
}

\frame{
    \ea[]{The cat appears to be out of the bag.}
    \ex[\#]{The cat tried to be out of the bag.}
    \z
}

\frame{
    Agent-modifying adverbs (\emph{excitedly, diligently}, etc.): what is the prediction? What are some examples?

\bigskip
    \ea[*]{\trace{}\trace{}\trace{}\trace{}\trace{}\trace{}\trace{}\trace{}\trace{}}
    \ex[]{\trace{}\trace{}\trace{}\trace{}\trace{}\trace{}\trace{}\trace{}\trace{}}
    \z
}

\frame{ \centering
    \begin{tabular}{c|c|c}
        &   Raising & Control\\
        & \emph{seem, appear, \dots{}} & \emph{try, want, \dots{}} \\\hline
    Expletive subjects & \cmark{} & \xmark{} \\
    Idiomatic meaning & \cmark{} & \xmark{} \\
    Agent-oriented adverbs & \xmark{} & \cmark{} \\\hline
    Matrix subject & Derived subject & PRO\\
    \end{tabular}
}

\frame{\frametitle{Summary}
     \begin{block}{Topic 9 skills}
    \begin{itemize}
        \item[9a]{Identify non-finite clauses headed by infinitival \emph{to} in a principled way.}
        \item[9b]{Distinguish raising structures from control structures on the basis of $\theta$-role assignment.}
    \end{itemize}
    \end{block}

    % We also discussed:
    % \begin{itemize}
        % \item Complement clauses vs relative clauses.
        % \item Constituency within the DP.
    % \end{itemize}

    Research Skills:
    \begin{itemize} \small
    \item[M1]{State a hypothesis clearly}
    \item[M2]{Present data in support of a hypothesis}
    \item[M3]{Define the conditions under which a hypothesis would be refuted}
    \item[M4]{Present data to refute a hypothesis}
    \item[M5]{Consider the limitations of a diagnostic test when interpreting the results}
    \end{itemize}
}
\end{document}