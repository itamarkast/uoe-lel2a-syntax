\documentclass{beamer}
\usepackage{hyperref} %links and URLS
\usepackage{graphicx} % Required for inserting images
\usepackage[linguistics]{forest} %calls tikz, draws trees
\usepackage{multicol} %adds columns
\usepackage{gb4e} %for formatting examples, works with leipzig and multicol
\primebars %setting for gb4e, adds bars for X-bar notation, allows switch between bar or %'%
\noautomath
\usepackage{tipa} % IK
\usepackage[T1]{fontenc} %Make sure to be able to get accented characters etc
\usepackage[utf8]{inputenc}
\usepackage[normalem]{ulem} %adds strikethrough and other commands
\usepackage{setspace}
\usepackage{pifont} %allows dingbats to be called (for the "crosses" and "ticks" defined below)

\title{LEL2A: Syntax}
%\author{Instructor: Itamar Kastner}
\date{Semester 1, 2024-25}%changed to current academic year


\setbeamertemplate{footline}[frame number]
\setbeamertemplate{blocks}[rounded][shadow=true]
\setbeamertemplate{navigation symbols}{}
\usecolortheme[RGB={0,51,102}]{structure}
\setbeamercolor{alerted text}{fg=red}

\newcommand{\cmark}{\ding{51}}
\newcommand{\xmark}{\ding{55}}
\newcommand\trace{\rule[-0.5ex]{0.5cm}{.4pt}}

\subtitle{Topic 8: Passives}

\begin{document}
\maketitle

\frame{\frametitle{Recap}
      \begin{block}{Topic 7 skills}
        \begin{itemize}
        \item[7a]{Implement an analysis of complex noun phrases that is consistent with the framework developed.}
        \item[7b]{Identify elements as complements or adjuncts of a noun in a principled way.}
    \end{itemize}
    \end{block}
}

\frame{\frametitle{Topic 8}
   \begin{block}{Topic 8 skills}
    \begin{itemize}
        \item[8a]{Account for the behaviour of passive movement, including any relevant restrictions, within the framework developed.}
        % \item[11b]{\sout{Identify relevant locality constraints on A-movement.}}
    \end{itemize}
    \end{block}

    Research Skills:
    \begin{itemize}
    \item[M1]{State a hypothesis clearly}
    \item[M2]{Present data in support of a hypothesis}
    \item[M3]{Define the conditions under which a hypothesis would be refuted}
    \item[M4]{Present data to refute a hypothesis}
    \item[M5]{Consider the limitations of a diagnostic test when interpreting the results}
    \end{itemize}
}

\frame{
    Let's enumerate the differences (and similarities) between the active and passive variants of the same basic clause:

    \bigskip
    \ea
        \ea Billie launched a new perfume.
        \ex A new perfume was launched (by Billie).
        \z
    \z

    \pause
    \begin{enumerate}
        \item Same selectional restrictions.
        \item Promotion of object to derived subject.
        \item Optional Agent (or in \emph{by}-phrase).
        \item Passive auxiliary and participial marking.
    \end{enumerate}
}

\frame{ \tiny
\begin{columns}
    \begin{column}{0.4\textwidth}
\hspace*{-3em}
         \begin{forest}
    [IP
        [DP [A new perfume, roof, name=copy]]
        [I'
            [I\\was]
            [VP
                [\phantom{X} ]
                [V'
                    [V\\launched][DP\\\sout{a new perfume}, name=trace]
                ]
            ]
        ]            
    ]
        \draw[->,dotted] (trace) to[out=south west,in=south west] (copy);
    \end{forest}
    \end{column}
    \begin{column}{0.6\textwidth}
        \begin{forest}
    [IP
        [DP [A new perfume, roof, name=copy]]
        [I'
            [I\\was]
            [VP
                [\phantom{X} ]
                [V'
                    [V'
                        [V\\launched][DP\\\sout{a new perfume}, name=trace]
                    ]
                    [PP
                        [\phantom{X} ]
                        [P'
                            [P\\by]
                            [DP [Billie, roof]]
                        ]
                    ]
                ]
            ]
        ]            
    ]
        \draw[->,dotted] (trace) to[out=south west,in=south west] (copy);
    \end{forest}
    \end{column}
\end{columns}
}

\frame{\frametitle{Extensions: locality}

    \ea
        \ea{It seems [that Sean plays the piano ].}
        \ex{Sean seems [to \trace{} play the piano ].}
        \z
    \ex
        \ea{Sean was believed [to \trace{} play the piano].}
        \ex{It seems [that Sean was believed [to \trace{} play the piano]].}
        \ex{Sean seems [to \trace{} be believed [to \trace{} have played the piano.]]}
        \z
    \ex[*]{Sean seems [that it was believed [ \trace{} to play the piano]].}
    \z
}

\frame{\frametitle{Extensions: locality}
    A \textbf{locality} constraint: we cannot move \alert{a DP} across \textcolor{blue}{\textbf{another DP}}.

    \bigskip
    \ea
        \ea It seems [that Sean plays the piano ].
        \ex Sean seems [to \trace{} play the piano ].
        \z
    \ex
        \ea Sean was believed [to \trace{} play the piano].
        \ex It seems [that Sean was believed [to \trace{} play the piano]].
        \ex Sean seems [to \trace{} be believed [to \trace{} have played the piano.]]
        \z
    \ex[*]{\alert{Sean} seems [that \textcolor{blue}{\textbf{it}} was believed [ \trace{} to play the piano]].}
    \z
}

\frame{\frametitle{Summary}
   \begin{block}{Topic 8 skills}
    \begin{itemize}
          \item[8a]{Account for the behaviour of passive movement, including any relevant restrictions, within the framework developed.}
    \end{itemize}
    \end{block}

    We also discussed:
    \begin{itemize}
        \item Locality in A-movement.
    \end{itemize}

    Research Skills:
    \begin{itemize} \small
    \item[M1]{State a hypothesis clearly}
    \item[M2]{Present data in support of a hypothesis}
    \item[M3]{Define the conditions under which a hypothesis would be refuted}
    \item[M4]{Present data to refute a hypothesis}
    \item[M5]{Consider the limitations of a diagnostic test when interpreting the results}
    \end{itemize}
}
\end{document}
